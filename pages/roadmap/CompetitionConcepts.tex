\chapter{Concepts Behind the Competition}\label{chap:concepts}

A set of conceptual key criteria forms the foundation of the \textsc{\AtHome{}} competition.
These criteria represent a shared understanding of the general concept of the competition.
The specific rules are detailed in the \AtHome{} Rules \& Regulations.

\section{Lean Set of Rules}\label{concept:lean_set_of_rules}

To enable diverse, general, and transmissible approaches in the \AtHome{} competition, the rule set should be as concise as possible
However, it must also be specific enough to eliminate ambiguity and prevent disagreements during the competition.
If discrepancies or multiple interpretations arise, decisions will be made by the \TC{} and on-site referees.

\paragraph*{Note:} Once the test scoresheet is signed the \TC{}'s decision is final and cannot be reversed.

\section{Autonomy \& Mobility}\label{concept:autonomy_and_mobility}

The aim of \AtHome{} is to foster mobile autonomous service robotics and natural human-robot interaction.
Thus, all robots participating in the RoboCup@Home competition must be \emph{mobile} and \emph{autonomous}, meaning that humans are not allowed to directly (remotely) control the robot, including through verbal commands.

\section{Aiming for Applications}\label{concept:aiming_for_applications}

To drive technological advancement and maintain the competition's interest, the scenario and the tests will steadily increase in complexity.
While fundamental abilities are tested, the focus will shift toward real-world applications with higher levels of complexity and uncertainty.
In \AtHome{}, solutions that are practical, reliable, general, cost-effective, and applicable are highly valued.

\section{Social Relevance}\label{concept:social_relevance}

The competition and its tests should produce outcomes that are socially relevant, as the goal is to demonstrate the value of autonomous robotic applications to the public.
This should be achieved by showing applications where robots directly help or assist humans in everyday situations.
Examples include a personal robot assistant, a guide robot for the blind, or robot care for the elderly.

\section{Scientific Value}\label{concept:scientific_value}

\AtHome{} should not only highlight current practical applications, but also encourage the presentation of innovative approaches, even if they are not yet fully applicable or require special configurations.
Therefore, approaches with high scientific value are rewarded.

\section{Time Constraints}\label{concept:time_constraints}

To accommodate many teams and tests, as well as to promote simplicity in setup procedures, the setup time and test duration are very limited.

\section{No Standardized Scenario}\label{concept:no_standardized_scenario}

The competition scenario should be simple, effective, available worldwide, and low-cost.
As uncertainty is a key part of the competition, no standardized scenario will be provided.
Instead, the scenario is expected to reflect typical environments of the country where the competition is held.

It could be a domestic setting like a living room or kitchen, or an office, supermarket, or restaurant. The scenario may change annually to ensure the tests remain relevant and engaging. Additionally, tests may occur in previously unknown environments, such as nearby public spaces.

\section{Attractiveness}\label{concept:attractiveness}

The competition should be appealing to both the audience and the public. Thus, approaches that are highly attractive and original are rewarded.

\section{Community}\label{concept:community}

While teams compete against each other during the competition, members of the \AtHome{} league are expected to collaborate and share knowledge to advance robotic technology collectively.

The \iterm{RoboCup@Home mailing list} and the \RR{} can be used to communicate with other teams and discuss league-specific issues such as rule changes or proposals for new tests.
% Since 2007 there is also the \iterm{RoboCup@Home Wiki} (see~\refsec{sec:at_home_wiki}) which serves as a central place to collect information relevant for the @Home league.
Additionally, every team is expected to share technical, scientific, and team-related information in a Team Description Paper and on their team website.

Finally, all teams are invited to submit papers on related research to the \Symp{}, which accompanies the annual RoboCup World Championship.

\section{Desired Abilities}\label{concept:desired_abilities}

The following are the key technical abilities that the tests in \AtHome{} aim to evaluate

\begin{itemize}
    \item Navigation in dynamic environments
    \item Fast and easy calibration and setup (the ultimate goal is to have a robot up and running out of the box)
    \item Object recognition
    \item Object manipulation
    \item Detection and recognition of humans
    \item Natural human-robot interaction
    \item Speech recognition
    \item Gesture recognition
    \item Robot applications (\AtHome{} is aiming for applications of robots in daily life)
    \item Ambient intelligence, such as communicating with surrounding devices, retrieving information from the internet, etc.
\end{itemize}


% Local Variables:
% TeX-master: "Rulebook"
% End:
