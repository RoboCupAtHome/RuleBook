%% %%%%%%%%%%%%%%%%%%%%%%%%%%%%%%%%%%%%%%%%%%%%%%%%%%%%%%%%%%%%%%%%%%%%%%%%%%%
%%
%%    author(s): RoboCupAtHome Technical Committee(s)
%%  description: Introduction - Awards
%%
%% %%%%%%%%%%%%%%%%%%%%%%%%%%%%%%%%%%%%%%%%%%%%%%%%%%%%%%%%%%%%%%%%%%%%%%%%%%%
\section{Awards}\label{sec:awards}

The \AtHome{} league features the \iterm{awards} described below.
Note that all awards need to be approved by the \RCF{}; based on a decision by the RCF, some of them may not be awarded

\subsection{Winner of the Competition}\label{award:winner}

For each league, there will be 1st, 2nd, and 3rd place award trophies (or only 1st and 2nd place trophies if the number of teams in a league is eight or fewer).

\subsection{Best Human-Robot Interface Award}\label{award:hri}

To honor outstanding human-robot interfaces developed for interacting with robots in \AtHome{}, a special \HRIAward{} may be awarded to one of the participating teams.
Special attention is given to interfaces that are open and available to the \AtHome{} community.

The \AtHome{} \EC{} members nominate a set of candidates for the award and the TC elects the winner.
A TC member whose team is among the nominees is not allowed to vote.
No \HRIAward{} is given if the EC decides there is no outstanding interface and thus no nominees.

\subsection{Best Poster Award}\label{award:poster}

To promote scientific knowledge exchange and reward teams for their efforts in presenting research contributions, all scientific posters from each league are evaluated for the \DSPLPosterAward{} or \OPLPosterAward{}.

Candidate posters must present innovative and state-of-the-art research in a field directly applicable to \AtHome{}, with clear and easy-to-understand results.
In addition to being attractive and well-rated in the \PS{} (see~\refsec{sec:poster_teaser_session}), the research must also have a measurable impact on the team's performance during the competition.

The \AtHome{} \EC{} members nominate a set of candidates for the award and the TC elects the winner.
A \TC{} member whose team is among the nominees is not allowed to vote.

\subsection{Open Challenge Award}\label{award:oc}

To encourage teams to showcase their research to other teams durin the competition off-hours, \AtHome{} grants the \OCAward{} to the team that presents the best open demonstration during the competition.
This award is given only if a team demonstrates innovative research relevant to the global objectives of \AtHome{}.

The \AtHome{} TC members, with a recommendation from the team leaders, nominate a set of candidates for the award. Yhe EC decides whether the award should be granted and then elects the winner.
A \TC{} member is not allowed to nominate their own team without a recommendation from the team leaders.

\subsection{Skill Certificates}\label{award:skill}

The \AtHome{} league features certificates for best demonstrated skills in \NAV{}, \MAN{}, \PerRec{}, and \NLP{}.
A team receives the certificate if it scores at least 75\% of the attainable points for that skill across all tests and challenges. 
For example, if a robot scores manipulation points during the \emph{Help-me-Carry} test to open the door, that will count for the \MAN{} certificate.
Note that the certificate will only be awarded if the team is \emph{not} the overall winner of the competition.

\subsection{Open-source software award}\label{award:oss}

Since Nagoya 2017, RoboCup@Home awards the best contribution to the community by means of an open-source software solution.
To be eligible for the award, the software must be easy to read, well-documented, follow standard design patterns, actively maintained, and meet the IEEE software engineering metrics of scalability, portability, maintainability, fault tolerance, and robustness.
Additionally, the open-source software must be available as a framework-independent standalone library to ensure it can be reused with any software architecture.

Candidates must submit their application to the \TC{} at least one month before the competition via a short paper (maximum 4 pages), in the same format as the \TDP{} (see~\refsec{rule:website_tdp}), including a brief explanation of the approach, comparison with state-of-the-art techniques, statement of the used metrics and software design patterns, and the name of the teams and other collaborators using the software.

The \AtHome{} \EC{} members nominate a set of candidates for the award and the \TC{} elects the winner.
A \TC{} member whose team is among the nominees is not allowed to vote.


