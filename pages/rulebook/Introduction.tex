%% %%%%%%%%%%%%%%%%%%%%%%%%%%%%%%%%%%%%%%%%%%%%%%%%%%%%%%%%%%%%%%%%%%%%%%%%%%%
%%
%%    author(s): RoboCupAtHome Technical Committee(s)
%%  description: Introduction
%%
%% %%%%%%%%%%%%%%%%%%%%%%%%%%%%%%%%%%%%%%%%%%%%%%%%%%%%%%%%%%%%%%%%%%%%%%%%%%%
\chapter{Introduction}
\label{chap:introduction}


\section{RoboCup}

\RoboCup{} is an international joint project to promote AI, robotics, and related fields.
It is an attempt to foster AI and intelligent-robotics research by providing standard problems where a wide range of technologies can be integrated and examined. More information can be found at \url{http://www.robocup.org/}.

\section{RoboCup@Home}

The \textsc{RoboCup@Home} league aims to develop service and assistive robot technology with high relevance for future personal domestic applications.
It is the largest international annual competition for autonomous service robots and is part of the RoboCup initiative.
A set of benchmark tests is used to evaluate the abilities and performance of different robots in a realistic, non-standardized home environment setting.
The focus is on, but is not limited to, the following domains: human-robot interaction and cooperation, navigation and mapping in dynamic environments, computer vision and object recognition under natural light conditions, object manipulation, adaptive behaviors, behavior integration, ambient intelligence, standardization and system integration.
The competition is co-located with the RoboCup symposium.

%% %%%%%%%%%%%%%%%%%%%%%%%%%%%%%%%%%%%%%%%%%%%%%%%%%%%%%%%%%%%%%%%%%%%%%%%%%%%
%%
%%  author(s): RoboCupAtHome Technical Committee(s)
%%  description: Introduction - Organization
%%
%% %%%%%%%%%%%%%%%%%%%%%%%%%%%%%%%%%%%%%%%%%%%%%%%%%%%%%%%%%%%%%%%%%%%%%%%%%%%
\section{Organization}

\AtHome{} is organized into three subcommittees. The current members of the committees are listed at \url{https://athome.robocup.org/committees/}.

\subsection{Executive Committee --- rc-home-ec@lists.robocup.org}
\label{sec:ec}

The \EC{} consists of members of the board of trustees, and representatives of each activity area, and supervises both the TC and OC.
The committee is responsible for the long-term planning of the league and cast the final vote in all decisions within the competition, including those pertaining to any of the committees and any other affair that requires a qualified vote.
Additionally, the EC must act when any of the committees fail, as it responds to the Trustee board and directs the league accordingly.

\subsection{Technical Committee --- rc-home-tc@lists.robocup.org}
\label{sec:tc}

The \TC{} is responsible for the rules of the league; its main focus is writing the rulebook and refereeing.
The members of the EC are always members of the TC as well.

\subsection{Organizing Committee --- rc-home-oc@lists.robocup.org}
\label{sec:oc}

The \OC{} is responsible for the organization of the competition, namely it creates the schedule and provides information about the scenarios.
The \LOC{}, on the other hand, is responsible for the set up and organization of the competition venue.


%% %%%%%%%%%%%%%%%%%%%%%%%%%%%%%%%%%%%%%%%%%%%%%%%%%%%%%%%%%%%%%%%%%%%%%%%%%%%
%%
%%    author(s): RoboCupAtHome Technical Committee(s)
%%  description: Introduction - Infrastructure
%%
%% %%%%%%%%%%%%%%%%%%%%%%%%%%%%%%%%%%%%%%%%%%%%%%%%%%%%%%%%%%%%%%%%%%%%%%%%%%%
\section{Infrastructure}\label{sec:introduction:infrastructure}

\paragraph{RoboCup@Home Mailing List}\label{sec:introduction:mailinglist}
The official \AtHome{} mailing list can be found at\\
\href{mailto:robocup-athome@lists.robocup.org}{\small\texttt{robocup-athome@lists.robocup.org}}.
You can subscribe to the mailing list at: {\small\url{https://lists.robocup.org/mm/lists/robocup-athome.lists.robocup.org/}}

\paragraph{RoboCup@Home Web Page}\label{sec:introduction:webpage}
The official \AtHome{} website that also hosts this rulebook can be found at {\small\url{https://athome.robocup.org/}}

\paragraph{RoboCup@Home Rulebook Repository}\label{sec:introduction:repo}
The official \AtHome{} \RR{} is where rules are publicly discussed before applying changes.
The entire \AtHome{} community is welcome and encouraged to actively participate in creating and discussing the rules.
The \RR{} is hosted at {\small\url{https://github.com/RoboCupAtHome/RuleBook/}}

\paragraph{RoboCup@Home Telegram Group}\label{sec:introduction:telegramgroup}
The official \AtHome{} \TG{} is a communication channel for the \AtHome{} community where rules are discussed, announcements are made, and questions are answered.
Beyond supporting the technical aspects of the competition, the group is a meeting point to stay in contact with the community, foster knowledge exchange, and strengthen relationships.
The \TG{} can be reached at {\small\url{https://t.me/RoboCupAtHome}}

\paragraph{RoboCup@Home Wiki}\label{sec:introduction:wiki}
The official \AtHome{} \WIKI{} is meant to be a central place to collect information on all topics related to the \AtHome{} league.
The wiki was set up to simplify and unify the exchange of relevant information; this includes, but is not limited to, hardware, software, media, data, and more.
The \WIKI{} can be reached at {\small\url{https://github.com/RoboCupAtHome/AtHomeCommunityWiki/wiki}}


%% %%%%%%%%%%%%%%%%%%%%%%%%%%%%%%%%%%%%%%%%%%%%%%%%%%%%%%%%%%%%%%%%%%%%%%%%%%%
%%
%%    author(s): RoboCupAtHome Technical Committee(s)
%%  description: Introduction - Leagues
%%
%% %%%%%%%%%%%%%%%%%%%%%%%%%%%%%%%%%%%%%%%%%%%%%%%%%%%%%%%%%%%%%%%%%%%%%%%%%%%
\section{Leagues}
\label{sec:leagues}

\AtHome{} is divided into two Leagues. One of these grants complete freedom to all competitors with respect to the robot used, while in the other all competitors use the same robot. The official leagues and their names are:
\begin{itemize}
  \item \OPL
  \item \DSPL
\end{itemize}

\begin{wrapfigure}[21]{r}{0.30\textwidth}
	\vspace{-30pt}
	\begin{center}
		\includegraphics[width=0.25\textwidth]{images/toyota_hsr.png}
		\vspace{-10pt}
		\caption{Toyota HSR}\label{fig:toyota_hsr}
	\end{center}
\end{wrapfigure}
Each league focuses on a different aspect of service robotics by targeting specific abilities.

\subsection{Domestic Standard Platform League (DSPL)}

The main goal of the DSPL is to assist humans in a domestic environment, paying special attention to elderly people and people suffering from illness or disability.
As a consequence, the DSPL focuses on \AmI, \CV, \OM, safe indoor \NAV{} and \MAP, and \TP.
The robot used in the DSPL is the \HSR, shown in Figure~\ref{fig:toyota_hsr}.

\subsection{Open Platform League (OPL)}

The OPL has had the same modus operandi since the foundation of \AtHome.
With no hardware constrains, OPL is the league for teams who want to test their own robot designs and configurations, as well as for old at-homers.
In this league, robots are tested to their limits without having in mind any concrete design restriction, although the scope is similar to the DSPL.


%% %%%%%%%%%%%%%%%%%%%%%%%%%%%%%%%%%%%%%%%%%%%%%%%%%%%%%%%%%%%%%%%%%%%%%%%%%%%
%%
%%    author(s): RoboCupAtHome Technical Committee(s)
%%  description: Introduction - Competition
%%
%% %%%%%%%%%%%%%%%%%%%%%%%%%%%%%%%%%%%%%%%%%%%%%%%%%%%%%%%%%%%%%%%%%%%%%%%%%%%
\section{Competition}
The competition consists of two \emph{Stages} and a \FINAL.
Each stage consists of a series of \iterm{Tests} that are being held in a daily life environment.
The best teams from \SONE{} advance to \STWO, which consists of more difficult tests.
The competition ends with the \FINAL, where only the two highest-ranked teams of each league compete to select the winner.


%% %%%%%%%%%%%%%%%%%%%%%%%%%%%%%%%%%%%%%%%%%%%%%%%%%%%%%%%%%%%%%%%%%%%%%%%%%%%
%%
%%    author(s): RoboCupAtHome Technical Committee(s)
%%  description: Introduction - Awards
%%
%% %%%%%%%%%%%%%%%%%%%%%%%%%%%%%%%%%%%%%%%%%%%%%%%%%%%%%%%%%%%%%%%%%%%%%%%%%%%
\section{Awards}
\label{sec:awards}

The \AtHome{} league features the \iterm{awards} described below.
Note that all awards need to be approved by the \RCF; based on a decision by the RCF, some of them may not be given.

\subsection{Winner of the Competition}
\label{award:winner}

For each league, there will be 1st, 2nd, and 3rd place award trophies (or first and second place only if the number of teams in a league is eight or less).

%
% As of 2017, the Execs have decided to remove the Innovation Award since
% is rarely given and its discussion is time consuming.
%
% \subsection{Innovation award}
% \label{award:innovation}
% To honour outstanding technical and scientific achievements as well as applicable solutions in the @Home league, a special \iterm{innovation award} may be given to one of the participating teams. Special attention is being paid to making usable robot components and technology available to the @Home community.
%
% The \iaterm{Executive Committee}{EC} members from the RoboCup@Home league nominate a set of candidates for the award. The \iaterm{Technical Committee}{TC} elects the winner. A TC member whose team is among the nominees is not allowed to vote.
%
% There is no innovation award in case no outstanding innovation and no nominees, respectively.


%
% As of 2017, the TC have decided to add an award for the best alternate HRI method
% to bypass speech recognition.
%
\subsection{Best Human-Robot Interface Award}
\label{award:hri}

To honor outstanding human-robot interfaces developed for interacting with robots in \AtHome{}, a special \HRIAward{} may be given to one of the participating teams.
Special attention is paid to making the interface open and available to the \AtHome{} community.

The \AtHome{} EC members nominate a set of candidates for the award and the TC elects the winner.
A TC member whose team is among the nominees is not allowed to vote.
There is no \HRIAward{} in case the EC decides that there is no outstanding interface, and thus no nominees.

\subsection{Best Poster Award}
\label{award:poster}

To foster scientific knowledge exchange and reward the teams' efforts to present their research contributions, all scientific posters of each league are evaluated and have the chance of receiving the \DSPLPosterAward or the \OPLPosterAward respectively.

Candidate posters must present innovative and state-of-the-art research within a field with a direct application to \AtHome, and demonstrate successful and clear results in an easy-to-understand way.
In addition to being attractive and well-rated in the \PS{} (see~\refsec{sec:poster_teaser_session}), the described research must have impact in the team's performance during the competition.

The \AtHome{} EC members nominate a set of candidates for the award and the TC elects the winner.
A TC member whose team is among the nominees is not allowed to vote.

%
% As of 2013, the Execs have decided to remove the Innovation Award due to
% the lack of interest of the participants
%
% \subsection{Winner of the Technical Challenge}
% In parallel to the regular competition, the RoboCup@Home league features a \iterm{Technical Challenge}. The winner of the Technical Challenge is given a special \iterm{award for winning the Technical Challenge}.
%
% As with the innovation award, the award for winning the Technical Challenge is not given in case no team shows a \emph{sufficient performance}. The decision which team wins the Technical Challenge, and if the award is given at all, is conducted by the \iaterm{Technical Committee}{TC}.

\subsection{Open Challenge Award}
\label{award:oc}

To encourage teams to present their research to the other teams in the competition off-hours, \AtHome{} grants the \OCAward{} to the best open demonstration presented during the competition.
This award is granted only if a team has demonstrated innovative research that is related to the global objectives of \AtHome; thus, the award is not necessarily granted.

The \AtHome{} TC members, with a recommendation from the team leaders, nominate a set of candidates for the award; the EC decides if the award should be granted and elects the winner.
A TC member is not allowed to nominate their own team without a recommendation from the team leaders.

\subsection{Skill Certificates}
\label{award:skill}

The \AtHome{} league features certificates for best demonstrated skills in \NAV, \MAN, \PerRec, and \NLP.
A team is given the certificate if it scores at least 75\% of the attainable points for that skill.
This is counted over all tests and challenges, so, for example, if a robot scores manipulation points during the \emph{Help-me-Carry} test to open the door, that will count for the \MAN{} certificate.
Note that the certificate will only be handed out if the team is \emph{not} the overall winner of the competition.

\subsection{Open-source software award}
\label{award:oss}

Since Nagoya 2017, RoboCup@Home awards the best contribution to the community by means of an open source software solution.
To be eligible for the award, the software must be easy to read, have proper documentation, follow standard design patterns, be actively maintained, and meet the IEEE software engineering metrics of scalability, portability, maintainability, fault tolerance, and robustness.
In addition, the open sourced software must be made available as a framework-independent standalone library so it can be reused with any software architecture.

Candidates must send their application to the TC at least one month before the competition by means of a short paper (maximum 4 pages), following the same format used for the \TDP{} (see~\refsec{rule:website_tdp}), including a brief explanation of the approach, comparison with state-of-the-art techniques, statement of the used metrics and software design patterns, and the name of the teams and other collaborators that are also using the software being described.

The \AtHome{} TC members nominate a set of candidates for the award and the EC elects the winner.
An EC/TC member whose team is among the nominees is not allowed to vote.


% \subsection{Procter \& Gamble Dishwasher Challenge Award}
% \label{award:pandg}
% \textit{Procter \& Gamble} gives an special award to the winner of the \textit{Procter \& Gamble: Clean the Table} task (described in \refsec{test:clean-the-table}), typically to the team scoring higher in the task.
% All teams can participate and compete for this award, regardless of whether they advanced to the Stage II or not, and get the award.

% The award for winning the \textit{Procter \& Gamble Dishwasher Challenge Award} is not given in case no team shows a \emph{sufficient performance} in the aformentioned task. The decision on which team wins the \textit{Procter \& Gamble Dishwasher Challenge Award} task, and if the award is given at all, is conducted by \textit{Procter \& Gamble}.


% Local Variables:
% TeX-master: "Rulebook"
% End:
