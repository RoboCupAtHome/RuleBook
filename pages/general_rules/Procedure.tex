\section{Procedure during Tests}

\subsection{Safety First!}\label{rule:safetyfirst}

\begin{enumerate}
	\item \textbf{Emergency Stop:} At any time, if a robot exhibits dangerous behavior towards people or objects, the owners must immediately stop it.
	\item \textbf{Stopping on request:} If a referee, member of the Technical or Organizational committee, an Executive or Trustee of the federation stops the robot (by pressing the emergency button) there will be no discussion. Similarly, if they tell the team to stop the robot, the robot must be stopped \emph{immediately}.
	\item \textbf{Penalties:} Failure to comply results in immediate disqualification from the ongoing competition by the \AtHome{} \TC{}. Additionally, the team and its members may be banned from future competitions for at least one year by the RoboCup Federation Trustee Board.
\end{enumerate}

\subsection{Maximum number of team members}\label{rule:number_of_people}

\begin{enumerate}
	\item \textbf{Regular Tests:} During regular tests, only one team member is allowed inside the \Arena{}.
	Exceptions are tests that explicitly require volunteer assistance.
	\item \textbf{Setup:} During the setup, the number of team members inside the \Arena{} is not limited.
	% \item \textbf{Open Demonstrations:} During the \iterm{Open Challenge} \iterm{Demo Challenge}, and the \iaterm{final demonstration}{Finals}, the number of team members inside the arena is not limited.
	%\item \textbf{Open Demonstrations:} During the \iterm{Open Challenge}, and the \iaterm{final demonstration}{Finals}, the number of team members inside the arena is not limited.
\end{enumerate}

\subsection{Fair play}\label{rule:fairplay}
\iterm{Fair Play} and cooperative behavior are expected from all teams throughout the competition, especially:
\begin{itemize}
	\item while evaluating other teams,
	\item while refereeing, and
	\item when interacting with other teams' robots.
\end{itemize}
This includes the following:
\begin{itemize}
	\item not attempting to cheat (e.g., pretending autonomous behavior where there is none),
	\item not exploiting the rules (e.g., not attempting to solve the task but trying to score), and
	\item not intentionally causing other robots to fail.
	\item not modifying robots in standard platforms.
\end{itemize}
Violating this rule may result in penalties such as negative scores, disqualification for a test, or even the entire competition.

\subsection{Expected Robot's Behavior}
Unless stated otherwise, it is expected that the robot behaves and react s as a polite and friendly human would.
This applies to both how the robot performs tasks and how it interacts with humans.
As rule of thumb, one may ask any non-scientist how they would solve the task.

Please note that average users may not be familiar with the specific procedures for operating a robot. Therefore, interactions should be as natural as those with a human.


\subsection{Robot Autonomy and Remote Control}
\begin{enumerate}
	\item \textbf{No touching:} During a test, the participants are not allowed to physically touch the robot(s) unless it is a \enquote{natural} way required by the task.

	\item \textbf{Natural interaction:} The only allowed means to interact with the robot(s) are gestures and speech.

	\item \textbf{Natural commands:} Anything that resembles direct control is forbidden.

	\item \textbf{Remote Control:} Remotely controlling the robot(s) is strictly prohibited.
	This includes pressing buttons, or intentionally influencing sensors.

	\item \textbf{Penalties:} Disregard of these rules will lead to disqualification for a test or for the entire competition.
\end{enumerate}



\subsection{Collisions}
\begin{enumerate}
	\item \textbf{\iterm{Touching}:} Gentle \emph{touching} of objects is acceptable but not advised. 
	Robots must not crash into anything.
	The \enquote{safety first} rule (\refsec{rule:safetyfirst}) takes precedence over any other rule.

	\item \textbf{\iterm{Major collisions}:} If a robot crushes into something during a test, it is immediately stopped. Additional penalties may apply.

	\item \textbf{\iterm{Functional touching}:} Robots are allowed to apply pressure on objects, push furniture and, and interact with the environment using structural parts other than their manipulators.
	This is known as \iterm{functional touching}.
	However, the robot must clearly announce the interaction and kindly request not being stopped.\\
	\textbf{Remark: } Referees may immediately stop a robot if there is suspicion of \emph{dangerous} behavior.
\end{enumerate}

\subsection{Removal of robots}\label{rule:robot_removal}

Robots that do not comply with the rules are stopped and removed from the \Arena{}.

\begin{enumerate}
	\item It is the decision of the referees and the \TC{} member monitoring the test if and when to remove a robot.

	\item When told to do so by the referees or the \TC{} member, the team must immediately stop the robot, and remove it from the \Arena{} without disturbing the ongoing test.
	
	\item More than one team member is allowed to enter the \Arena{} after the robot has been stopped to quickly remove it.

\end{enumerate}


\subsection{Start Signal}\label{rule:start_signal}
The default \iterm{start signal} (unless stated otherwise) is \iterm{door opening}.
Other start signals are allowed but must be authorized by the \TC{} during the Robot Inspection (see~\refsec{sec:robot_inspection}).

\begin{enumerate}
    \item \textbf{Door opening:} The robot starts outside the \Arena{} and is accompanied by a team member. The test begins when a referee (not a team member) opens the door.

    \item \textbf{Start button:} If the robot cannot start automatically after the door is open, the team may use a start button to initiate the test.
    \begin{enumerate}[nosep]
        \item It must be a physical button on the robot (e.g., a dedicated one or releasing the eStop).
        \item It is allowed to use the robot's contact/pressure sensors (e.g., pushing the head or an arm joint).
        \item The use of the start button must be announced to the referees before the test starts.
        \item There may be penalties for using a start button in some tests.
    \end{enumerate}

    \item \textbf{Ad-hoc start signal:} Other start signals are allowed but must be approved by the \TC{} during the Robot Inspection (see~\refsec{sec:robot_inspection}).

    These include:
    \begin{itemize}[nosep]
        \item QR Codes
        \item Verbal instructions
        \item Custom HRI interfaces (apps, software, etc.)
    \end{itemize}
    \textbf{Remark:} There may be penalties for using Ad-hoc start signals in some tests. The use of mouses, keyboards, and devices connected to \ECRA{} computers is strictly forbidden.

    \item \textbf{No penalties for physical constraints (HSR only):} 
    In cases where a physical issue (e.g., a door bump) prevents the league from ensuring a standard start signal with full safety for the HSR robot, no penalties will be applied. The following guidelines apply:
    \begin{itemize}[nosep]
        \item The team must document the physical issue and notify the committee prior to the Robot Inspection.
        \item The issue must be reviewed and approved by the league's rules committee.
        \item An alternative start method will be determined on-site in consultation with the committee.
    \end{itemize}
    This rule ensures fairness and flexibility while addressing unavoidable physical challenges during league operations.
\end{enumerate}

\subsection{Restart Rule}\label{rule:restart}

During any test, teams are allowed to perform a single restart per attempt under the following conditions:
\begin{itemize}
    \item A restart may only be requested until the robot scores 40 points.
    \item The test time continues to run during the restart.
    \item The team moves the robot back to the starting location.
    \item The team has a maximum of \SI{60}{\second} to restart the robot.
    \item Only one restart is permitted per test attempt.
    \item Only points scored after a restart are counted; points scored before the restart are not considered.
\end{itemize}


\subsection{Entering and leaving the \Arena{}}\label{rule:start_position}
\begin{enumerate}

	\item \textbf{Start position:} Unless stated otherwise, the robot starts outside of the \Arena{}.
	\item \textbf{Entering:} The robot must autonomously enter the \Arena{}.
\end{enumerate}



\subsection{Gestures}\label{rule:gestures}
Hand gestures may be used to control the robot in the following way:
\begin{enumerate}
	\item \textbf{Definition:} The teams define the hand gestures by themselves.

	\item \textbf{Approval:} Gestures need to be approved by the referees and \TC{} member monitoring the test. Gestures should not involve more than the movement of both arms. This includes, for example, expressions of sign language or pointing gestures.

	\item \textbf{Instructing operators:} It is the responsibility of the team to instruct operators.
	\begin{enumerate}
		\item The team may only instruct the operator when told to so by a referee.
		\item The team may only instruct the operator in the presence of a referee.
		\item The team may only instruct the robot for as long as allowed by the referee.
		\item When the robot has to instruct the operator, it is the robot that instructs the operator and \emph{not} the team. The team is not allowed to guide the operator further, such as telling them to come closer or repeat a command.
		\item The robot is allows to instruct the operator at any time.
	\end{enumerate}

	\item \textbf{Receiving gestures:} Unless stated otherwise, it is not allowed to use a speech command to set the robot into a special mode for receiving gestures.
\end{enumerate}



\subsection{Referees}\label{rule:referees}
All tests are monitored by a referee, who is a member of the \TC{}. The referee may appoint an assistant to help with timekeeping and score sheets.

The following rules apply:
\begin{enumerate}
	\item \textbf{Selection:} Referees are chosen by \EC{},\TC{},\OC{}.

	\item \textbf{Referee instructions:} Right before each test, the referee selects one or more assistants to aid during the test. The assistants will be instructed by the referee.
\end{enumerate}


\subsection{Operators}\label{rule:operator}
Unless stated otherwise, robots are operated by the referee or by a person selected by the referee.
If the robot fails to understand the default operator, the team may request the use of a custom operator.
A penalty may apply when using a custom operator.


\subsection{Time limits}\label{rule:time_limits}
\begin{enumerate}
	\item Unless stated otherwise, the time limit for each test is \timing{5 minutes}.

	\item \textbf{Inactivity:} Robots must not stand still or get stuck in endless loops.
	If a robot is not progressing in the task (and is clearly not attempting to do so), it is considered inactive.
	The robot must be removed from the \Arena{} after 30 seconds of inactivity.

	\item \textbf{Requesting time:} A robot (not the team) can request the referees to extend the inactivity time limit for a time-consuming process (e.g., 60~seconds). This time cannot exceed 3 minutes and can only be used once per test.

	\item \textbf{Setup time:} Unless stated otherwise, there is no setup time.
	Robots need to be ready to enter the \Arena{} no later than one minute after the door has been closed to the previous team.

	\item \textbf{Time-up:} When the time is up, the team must immediately remove their robot(s) from the  \Arena{}.
	No more additional score will be giving.

	\item \textbf{Show must go on:} In special cases, the referee may allow the robot to continue the test for demonstration purposes, but no additional points will be scored.
\end{enumerate}
