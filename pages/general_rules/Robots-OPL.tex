\subsection{Robot Specifications for the Open Platform League}
Robots competing in the RoboCup@Home Open Platform League must comply with security specifications to ensure they do not pose any risk to people or property while operating.

\subsubsection{Size and Weight}\label{rule:robots_size}

\begin{enumerate}
	\item \textbf{Dimensions:} The robot must fit within the dimensions of an average door, which are typically \SI{200}{\centi\meter} by \SI{70}{\centi\meter} in most countries.
	The \TC{} may allow the qualification and registration of larger robots, but they cannot guarantee that these robots can actually enter the \Arena{} due to local restrictions.
	If in doubt, please contact the \LOC{}.
	\item \textbf{Weight:} While there are no specific weight restrictions, the robot's total weight and the pressure it exerts on the floor must not exceed local regulations for the construction of offices and/or buildings in the country where the competition is held.
	\item \textbf{Transportation:} Team members are responsible for quickly removing the robot from the \Arena{}.
	The robot must be transportable by the team members in a way that is both quick and easy.
\end{enumerate}

\subsubsection{Appearance}\label{rule:robots_appearance}

\OPL{} robots should appear as safe, finished products rather than early-stage prototypes.
This includes ensuring that all internal hardware (electronics and cables) is completely covered for safety.
While covering the robot's internal components with a t-shirt is not prohibited, it is not recommended.

\subsubsection{Emergency Stop Button}\label{rule:robots_emergency_button}

\begin{enumerate}
	\item \textbf{Accessibility and visibility:} Each robot must have an easily accessible and visible \EmergencyStop{} button.
	\item \textbf{Color:} The \EmergencyStop{} button must be colored red and must be the only red button on the robot.
	The \TC{} may ask the team to tape over or remove any other red buttons present on the robot.
	\item \textbf{Robot behavior:} When the \EmergencyStop{} button is pressed, the robot and all its parts must stop moving immediately.
\end{enumerate}

\subsubsection{Start Button}\label{rule:start_button}

\begin{enumerate}
	\item \textbf{Requirements:} As explained in~\refsec{rule:start_signal}, teams that cannot perform the default start signal (opening the door) must provide a \StartButton{} that can be used to start tests.
	Teams must inform the TC in advance of any test that involves a start signal, including the \RobotInspection{}.
	\item \textbf{Definition:} The \StartButton{} can be any \enquote{one-button procedure} that is easy for a referee to execute, such as releasing the \EmergencyStop{}, pressing a green button, or using a software button in a graphical user interface.
\end{enumerate}

\noindent\textbf{Note:} All robot requirements will be tested during the \RobotInspection{} (see~\ref{sec:robot_inspection}).

% \subsubsection{Audio output plug}
% \label{rule:roobt_audio_out}

% \begin{enumerate}
% 	\item \textbf{Mandatory plug:} Either the robot or some external device connected to it \emph{must} have a \iterm{speaker output plug}. It is used to connect the robot to the sound system so that the audience and the referees can hear and follow the robot's speech output.
% 	\item \textbf{Inspection:} The output plug needs to be presented to the TC during the \iterm{Robot Inspection} test (see~\refsec{sec:robot_inspection}).
% 	\item \textbf{Audio during tests:} Audio (and speech) output of the robot during a test have to be understood at least by the referees and the operators.
% 	\begin{compactitem}
% 		\item It is the responsibility of the teams to plug in the transmitter before a test, to check the sound system, and to hand over the transmitter to next team.
% 		\item Do not rely on the sound system! For fail-safe operation and interacting with operators make sure that the sound system is not needed, e.g., by having additional speakers directly on the robot.
% \end{compactitem}
% \end{enumerate}
