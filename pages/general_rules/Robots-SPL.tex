\subsection{Standard Platform Leagues}

Standardized platforms allow teams to compete under equal conditions by eliminating all hardware-dependent variables from the tasks; therefore, \emph{unauthorized} modifications and alterations to the robots are strictly forbidden.
This includes, but is not limited to, attaching, connecting, plugging, gluing, and taping components into and onto the robot, as well as, modifying or altering the robot structure.
Not complying with this rule leads to an immediate disqualification and penalization of the team (see~\refsec{rule:extraordinary_penalties}).
Robots are, however, allowed to \enquote{wear} clothes, have stickers (such as a sticker exhibiting the logo of a sponsor), or be painted (provided that the robot provider has approved that).

All modifications to the robots will be examined during the \RobotInspection{} (see \ref{sec:robot_inspection}).
Note that the EC and TC members may request re-inspection of an SPL robot at any time during the competition.

\subsubsection{Authorized DSPL Modifications}
\label{rule:osl_dspl}

In the \DSPL{}, teams may use an external laptop, which is safely located in the official \MountingBracket{} provided by Toyota and is connected to the \HSR{} via an Ethernet cable.
Any laptop fitting inside the \MountingBracket{} is allowed to be used, regardless of its technical specification.
Furthermore, teams are allowed to attach the following devices to either the \HSR{} or the laptop in the \MountingBracket:
\begin{itemize}
	\item \textbf{Audio}: A USB audio output device, such as a USB speaker or a sound card dongle.
	\item \textbf{Wi-Fi adapter}: A USB-powered IEEE 802.11ac (or newer) compliant device.
	\item \textbf{Ethernet Switch}: A USB-powered IEEE 802.3ab (or newer) compliant device.
\end{itemize}
In all cases, a maximum of three such devices can be attached, such that they may not increase the robot's dimensions.
For this purpose, using short cables and attaching the devices to the laptop in the \MountingBracket{} is advised.
