% %% %%%%%%%%%%%%%%%%%%%%%%%%%%%%%%%%%%%%%%%%%%%%%%%%%%%%%%%%%
%
% External Devices
%
% %% %%%%%%%%%%%%%%%%%%%%%%%%%%%%%%%%%%%%%%%%%%%%%%%%%%%%%%%%%
\section{External devices}
\label{rule:robot_external_devices}

Everything that a team uses during a test and is not part of the robot is considered an \ExternalDevice.
All \ExternalDevices{} must be authorized by the \TC{} during the \RobotInspection{} test (see~\refsec{sec:robot_inspection}).
The TC specifies whether an \ExternalDevice{} can be used freely or under referee supervision, and determines its impact on scoring.

Note that the use of wireless devices, such as hand microphones and headsets, is not allowed, with the exception of \ExternalComputing{} as specified below.
Please also note that the competition organizers do not guarantee or take any responsibility regarding the availability or reliability of the network or the internet connection in the \Arena{}.
Teams can thus use \ExternalComputing{} resources at their own risk.

\subsection{On-site External Computing}

Computing resources that are not physically attached to the robot are considered \ExternalComputing{} resources.
The use of up to five \ExternalComputing{} resources is allowed, but only in the \ArenaNetwork{} (see \Rulebook) and with a prior approval of the TC.
Teams must inform the TC about the use of any \ExternalComputing{} at least one month before the competition.
Note, however, that robots must be able to operate safely even if \ExternalComputing{} is unavailable.

\ExternalComputing{} devices must be placed in the \ECRA{}, which is announced by the TC during the \SetupDays.
A switch connected to the \Arena{} wireless network will be available to teams in the ECRA.
During a \Testblock, at most two laptops and two people from different teams are allowed in the ECRA simultaneously, one member each of the teams up next.
No peripherals, such as screens, mice, keyboards, and so forth, are allowed to be used.

During a \Testslot, everyone must stay at least \SI{1}{\meter} away from the ECRA.
Interacting with anything in the ECRA after the referee has given the start signal for a test will result in the test being stopped with a score of zero.

If a laptop is used as \ExternalComputing, a team can only place it in the ECRA if their \Testslot{} is up next and must remove the device immediately after the test.


\subsection{Microphones for HSR}
\label{rule:microphones_hsr}

To address the inherent challenges in the auditory systems of the Toyota HSR, the use of external microphones is allowed under the following conditions:
\begin{itemize}
    \item \textbf{Connection:} External microphones may be connected to the HSR via USB or AUX.
    \item \textbf{Mounting:} The microphone can only be mounted directly on the robot.
    \item \textbf{Wireless Microphones:} Wireless microphones are NOT allowed.
    \item \textbf{Built-in Microphone Usage:} Teams may choose to use only the built-in HSR microphone.
\end{itemize}

\subsection{On-line external computing}
\label{rule:robot_external_computing_online}

Teams are allowed to use \ExternalComputing{} through the internet connection of the \ArenaNetwork{}; this includes cloud services or online APIs.
These must be announced to and approved by the TC one month prior to the competition.

% Local Variables:
% TeX-master: "../Rulebook"
% End:
