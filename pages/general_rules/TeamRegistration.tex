%%%%%%%%%%%%%%%%%%%%%%%%%%%%%%%%%%%%%%%%%%%%%%%%%%%%%%%%%
\section{Team Registration and Qualification}


\subsection{Registration and Qualification Process}
\label{rule:participation}

Each year, there are three phases in the process towards participation in \AtHome:
\begin{enumerate}
	\item \iterm{Preregistration}
	\item \iterm{Qualification} announcement
	\item Final \iterm{Registration} for qualified teams
\end{enumerate}
The \iterm{Preregistration} process is initiated by a call for participation sent to the \iterm{RoboCup@Home mailing list}.
Preregistration requires a \iterm{Team Description Paper}, a \iterm{Qualification Video}, and a \iterm{Team Website}.
The expected contents of these are described below.

\subsection{Team Description Paper}\label{rule:website_tdp}

The \TDP{} is an 8-pages scientific paper that must include a description of your main research, including the scientific contribution, goals, scope, and results.
The \TDP{} has to be written in English and formatted using the template of the RoboCup International Symposium without any alterations.
The paper should contain the following items:
\begin{itemize}
	\item The focus of research and the contributions in the respective fields
	\item Innovative technology (if any)
	\item Re-usability of the system for other research groups
	\item Applicability of the robot in real-world scenarios
	\item Photo(s) of the robot(s) used
\end{itemize}

As an appendix on the 9th page (after the references), please include:
\begin{itemize}
	\item Team name
	\item Contact information
	\item Website URL
	\item Names of the team members
	\item Photo(s) of the robot(s), unless included before
	\item Description of the hardware used
	\item A brief, compact list of \iterm{external devices} (see~\refsec{rule:robot_external_devices}), if any
	\item A brief, compact list of any used 3rd party software packages (e.g.~ROS' \texttt{object\_recognition} should be listed, but \texttt{OpenCV} doesn't have to be because it is a rather standard library)
	\item \textbf{[Open Platform League only]} A brief description of the hardware used by the robot(s)
\end{itemize}

During the qualification process, the \TDP{} will be scored according to its scientific value, novelty, and contributions.

\subsection{Qualification Video}

As a proof of running hardware, each team has to provide a \iterm{Qualification Video} that demonstrates at least two of the following minimum requirements:
\begin{itemize}
	\item Human-robot interaction
	\item Safe navigation (indoors, with obstacle avoidance)
	\item Object detection and manipulation
	\item People detection
	\item Speech recognition
	\item Speech synthesis (clear and loud)
\end{itemize}

It is also recommended that the video include some of the following abilities:
\begin{itemize}
	\item Activity recognition
	\item Complex speech recognition
	\item Complex action planning
	\item Gesture recognition
	\item Failure recovery
\end{itemize}

The video should not exceed \SI{10}{\minute}, be self-explanatory and be designed for a general audience.
It must show the robot solving complex tasks relevant to \AtHome{}. 
In particular, to qualify for the competition, the video must demonstrate that the robot is able to successfully solve at least one test from the current or previous year's rule book.
For robots moving slowly, we recommend speeding up the video, but please indicate the speed factor being used when doing so (e.g. 2x). The same rule applies for slow motion scenes.

\subsection{Team Website}

The \iterm{Team Website} should be designed for a broader audience and include scientific material (scientific papers, datasets, and documented open source code).
The requirements for the website are as follows:

\begin{enumerate}
	\item \textbf{Language}: The \iterm{Team Website} must be in English. Other languages may be available, but English must be default language.
	\item \textbf{Team}: A comprehensive list of all team members, including brief profiles.
	\item \textbf{RoboCup}: Link to the league website and previous participations of the team at \RoboCup{} (not necessarily only \AtHome{}).
	\item \textbf{Scientific approach}: Include a research statement, a description of the used approach, and information on scientific achievements.
	\item \textbf{Publications}: Relevant \iterm{publications} from at least the last five years should be included. Downloadable publications are scored higher during the qualification process.
	\item \textbf{Open source material}: Blueprints, datasets, repositories, or any other contributions to the league are highly valued during the qualification process.
	\item \textbf{Multimedia}: Photos and videos of the robot(s) used should be included and easily accessible.
\end{enumerate}

%% %%%%%%%%%%%%%%%%%%%%%%%%%%%%%%%%%%%%%%%
\subsection{Qualification}\label{rule:qualification}

During the \iterm{Qualification Process}, a selection will be made by the \OC{}.
The following factors are evaluated in the decision process:
\begin{itemize}
	\item The scientific value, novelty, and contributions of the \TDP{}
	\item The number of abilities and the complexity of the tasks shown in the \iterm{Qualification Video}
	\item The contents of the \iterm{Team Website}, where relevant publications and open source resources are highly valued
\end{itemize}

In addition, the following evaluation criteria are considered:
\begin{itemize}
	\item The performance in previous competitions
	\item Relevant scientific contributions and publications
	\item Any additional contributions to the \AtHome{} league
\end{itemize}

\paragraph{Important note for the Standard Platform Leagues:} Only unmodified robots may compete in Standard Platform Leagues. Any \textit{unathorized} modification made to the robot that are found in the qualification material will automatically disqualify the team in the qualification process. % (see~\refsec{rule:spl-mods}).

\subsection{Participation Confirmation}\label{rule:participation-confirmation}

In order to have as many participating teams as possible, qualified teams \emph{must} contact the \OC{} to confirm (or cancel) the participation several months in advance.
Confirming attendance implies that the team has sufficient resources to register for the competition and commits to attend the event.
Qualified teams that do not confirm their participation may be disqualified.



