\section{Open Challenge}\label{sec:rules:openchallenge}

On the first two competition days after the end of the regular test blocks, teams will have an opportunity to present an \OpenChallenge{} in which they demonstrate their novel research and approaches.

\subsection{Procedure}
\label{sec:rules:ocprocedure}
\begin{enumerate}
	\item \textbf{Participation:} Teams have to announce whether they want to perform an \OpenChallenge{} to the \abb{OC} during the \SetupDays{}.
	\item \textbf{Time:} Each team gets a 10 minute time slot for the \OpenChallenge; of these, 8 minutes are for presenting and 2 minutes are for questions by the audience.
	\item \textbf{Arena Changes:} The team can rearrange the arena when their time slot starts, but all changes need to be reverted as soon as their time slot ends.
	\item \textbf{Focus:} While the demonstrations are intended to share research insights, we still want to see robots performing; in particular, the \OpenChallenge{} should not be turned into a pure academic lecture.
	\item \textbf{Leagues:} Ideally, the open challenges of all \AtHome{} leagues will be scheduled consecutively to allow all participants to view all demonstrations. 
	However, if more than 12 teams across the leagues register for the \OpenChallenge{}, each league will hold its \OpenChallenge{} concurrently to accommodate the increased number of participants.
	
	\item \textbf{Award:} The \OpenChallenge{} does not contribute any points towards the official competition score, but participating teams are eligible to receive the \OCAward{} (see \ref{award:oc}).
\end{enumerate}
