\section[Human Assistance]{Bypassing Features With Human Help \\ \small Because the Show Must Go On}\label{rule:continue}

Robots are expected to autonomously complete all task objectives.
However, when progress is prevented by a malfunction or limitation, robots may request human assistance to allow the test to continue.

Human assistance is intended as a last resort. Robots should first attempt to skip or defer subtasks they cannot complete and continue with other achievable parts of the task. Assistance should only be requested when the robot cannot make further meaningful progress otherwise.

\subsection{Procedure}\label{rule:continue_procedure}

The robot must \textbf{clearly indicate} that it is requesting human assistance.

If a task explicitly specifies specific forms of human assistance, the robot may request such assistance using simplified or task-specific instructions. However, vague or ambiguous requests may be interpreted incorrectly by the human assistant and are not guaranteed to be executed as intended.

Regardless of how the request is phrased, the robot is expected to confirm that the requested action has been carried out as intended (e.g.\ through dialogue, perception, or other appropriate feedback).

For assistance types not explicitly listed in the task description, the robot still follow the following procedure:

\begin{enumerate}
	\item \textbf{Request help:} The request must specify:
	\begin{compactitem}
		\item the type of assistance required, and
		\item the intended outcome of the action.
	\end{compactitem}

	\item \textbf{Supervise:} The robot should actively supervise the assistance and be able to:
	\begin{compactitem}
		\item recognize whether the requested action has been performed, and
		\item determine whether the outcome matches the intended request.
	\end{compactitem}

	\item \textbf{Acknowledge:} The robot must not assume success without confirmation through perception or explicit interaction.
\end{enumerate}

% \subsection*{Example}\label{rule:continue_example}
% In the following example, a robot has to clean the table but is unable to grasp the spoon.
% \begin{itemize}[noitemsep]
% 	\small
% 	\item[\textcolor{gray}{R:}] \texttt{I am sorry, but the spoon is too small for me to take.\\
% 	Could you please help me with it?\\
% 	Please say "robot yes" or "robot no" to confirm.}
% 	\item[\textcolor{gray}{H:}] \textit{Robot, yes!}
% 	\item[\textcolor{gray}{R:}] \texttt{Thank you! Please follow my instructions.\\
% 	Please take the purple spoon from the table. It is on my left. \\(The robot also shows the result of the perception, e.g. by pointing at the spoon or showing a picture with a bounding box on the screen)}
% 	\item[\textcolor{gray}{H:}] (Referee takes purple spoon)
% 	\item[\textcolor{gray}{R:}] \texttt{I saw you took the spoon.\\
% 	Would you be so kind as to follow me to the kitchen?\\
% 	Please keep the spoon visible in front of you so I can track you. Thank you!}
% 	\item[\textcolor{gray}{R:}] \texttt{You can stop following me now.\\
% 	As you can see, the dishwasher is already open.\\
% 	Please place the spoon in the gray basket on the lower tray.}
% 	\item[\textcolor{gray}{R:}] \texttt{Lovely! Thanks for your help.\\
% 	I'll let you know if I need further assistance.}
% \end{itemize}


\subsection{Scoring}\label{rule:continue_scoring}
Any task element completed through human assistance is scored as zero points, unless explicitly stated otherwise in the task description.

\subsection{Referee Intervention}

The referee may stop the task at any time if:

\begin{itemize}
    \item Assistance is requested that was not explicitly allowed in the description for the task.
    \item Assistance is requested that was not announced during the Team Leader Meeting.
    \item The robot is not making reasonable progress on the task despite being assisted.
    \item From the second human assistance request onwards, repeated or excessive requests without demonstrating autonomous action may lead the referee to immediately stop the task.
\end{itemize}

\subsection*{Example}\label{rule:continue_example}
Example (non-exhaustive):
\begin{itemize}[noitemsep]
	\item Requesting a human to hand over an object instead of picking it up, and later requesting the human to place the same object instead of placing it autonomously may result in immediate termination of the task.
\end{itemize}

\subsection{Bypassing Automatic Speech Recognition}\label{rule:asrcontinue}
Giving commands to the robot is essential in many tests.
When the robot is not able to receive spoken commands, teams are allowed to provide means to bypass ASR via an Alternative method for HRI.
Nonetheless, Automatic Speech Recognition is preferred.

The following rules apply in addition to the ones specified in section \refsec{rule:continue_scoring}
\begin{enumerate}
	\item \textbf{ASR with Default Operator:} No score reduction.
	The command is given by the human operator who must speak (not shout) loud and clear.
	The \iterm{default operator} may repeat the command up to three times.

	\item \textbf{ASR with Custom Operator:} A reduction of 10\% of the maximum attainable score is applied when a \iterm{custom operator} is requested.
	The Team Leader chooses a person who gives the command \emph{exactly as instructed by the referee}.

	\item \textbf{Gestures:} A reduction of 20\% of the maximum attainable score is applied when a gesture (or set of gestures) is used to instruct the robot.

	\item \textbf{Alternative Natural Input Method:} A reduction of up to 20\% of the maximum attainable score is applied when a \iterm{alternative HRI interface}, is used to instruct the robot.
	Natural alternative HRI interfaces must be previously approved by the TC during the Robot Inspection (see~\refsec{sec:robot_inspection}).
\end{enumerate}


\subsubsection{Alternative interfaces for HRI}\label{rule:asralternative}
Alternative methods and interfaces for HRI offer a way for a robot to start or complete a task.
Any reasonable method may be used, with the following criteria:
\begin{itemize}
	\item \textbf{Intuitive to use and self-explanatory:} a manual should not be needed. Teams are not allowed to explain how to interface with the robot. %you immediately know how to use it after a quick glance

	\item \textbf{Effortless use:} Must be as easy to use as uttering a command. %is as easy to use as it is uttering a command

	\item \textbf{Is smart and preemptive:} The interface adapts to the user input, displaying only the options that make sense or that the robot can actually perform.

\end{itemize}

Preferably, the alternative HRI must be also adapted to the user.
Consider localization (with English as the default), but also potential users of service robots at their home.
For example: elderly people and people with physical disabilities.

%\textbf{\textsc{Award:}} The best alternative is awarded the Best Human-Robot Interface award (\refsec{award:hri}).

