\section{Organization of the Competition}\label{sec:procedure_during_competition}

\subsection{Competition Structure}\label{rule:structure}

The competition consists of a series of tasks, repeated on each competition day, designed to test specific robot capabilities, followed by a \FINAL{} as the final evaluation of the teams' overall performance.


\begin{enumerate}
	\item \textbf{Robot Inspection:} 
	All teams must pass a \RobotInspection{} during the \SetupDays{} to be eligible for competition (see~\refsec{sec:robot_inspection}).

	\item \textbf{Tasks:}
	Teams compete in a set of tasks that evaluate different robot abilities.
	Each task is repeated on multiple competition days, giving teams more than one opportunity to attempt it.
	For ranking purposes, the \emph{best score} obtained in each task is considered.

	\item \textbf{Restaurant Task:}
	Teams compete in a restaurant task that tests the robot's ability to serve food and drinks to customers.
	The restaurant task is held once during the competition and teams have a single attempt to complete the task on the competition day before the \FINAL{}.
	The restaurant task is held outside the arena in an unknown location until the time of the task. 
	Ideally it is performed in a real restaurant otherwise a simulated restaurant environment will be used.
	Due to scheduling constraints a registration for restaurant may be required, and participation may depend on current classification.

	\item \textbf{\FINAL:} The top \emph{three teams}, namely the ones with the highest overall score, advance to the \FINAL{}.
	The final round features a single integrated task that tests all abilities evaluated throughout the competition.
\end{enumerate}

In cases where there is no significant score difference between teams, the \TC{} may decide to include additional teams in the \FINAL{}.

%%%%%%%%%%%%%%%%%%%%%%%%%%%%%%%%%%%%%%%%%%%%%%%%%%%%%%%%%
\subsection{Schedule}\label{rule:schedule}

\begin{enumerate}
	 \item \textbf{Daily Task Runs:} 
	Each competition day features the same set of four tasks.
	Two tasks are executed in parallel in different arenas.
	Once a team completes its run in one arena, it proceeds to the other arena according to the schedule.

	\item \textbf{Team Groups:}
	Teams are assigned to fixed groups (e.g.\ A, B, C, D).
	Each block has predetermined start times for the arenas and tasks.
	This ensures an even distribution of teams across arenas and reduced waiting time.

	\item \textbf{Test Slots:} Within each arena, teams are allocated a specific Test Slot that defines the maximum time available to complete the task. 
    Teams must be ready at their assigned block start time.

	\item \textbf{Participation is default:} Teams must inform the \OC{} in advance if they are skipping a Test Block. If they do not provide such notification, they may be penalized for not attending (see~\refsec{rule:not_attending}).
	
\end{enumerate}

\begin{table}[h]
	\centering\small
	\newcommand{\wcell}[2]{%
		\parbox[c]{2.5cm}{%
			\vspace{#1}%
			\centering%
			#2%
			\vspace{#1}%
		}%
	}
	\newcommand{\cell}[1]{\wcell{0.2\baselineskip}{#1}}
	% \newcommand{\mr}[1]{\multirow{2}{*}{#1}}


	\begin{tabular}{
		>{\centering\arraybackslash}m{2.5cm}|c|c|c|c
	}
	\multicolumn{1}{ c }{}
		& \multicolumn{1}{c}{Day 1}
		& \multicolumn{1}{c}{Day 2}
		& \multicolumn{1}{c}{Day 3}
		& \multicolumn{1}{c}{Day 4}
		\\\hhline{~---~}

	\cell{Block 1\\\footnotesize(9:00--11:00)}
		& \cellcolor{color1}\cell{\diagbox{Task 1}{Task 2}}
		& \cellcolor{color1}\cell{\diagbox{Task 1}{Task 2}}
		& \cellcolor{color2}\cell{Restaurant}
		& 
		\\\hhline{~----}



	\multicolumn{1}{ c }{}
		& \multicolumn{3}{ c }{\wcell{0.5\baselineskip}{\color{gray}\----Break---\-}}
		& \multicolumn{1}{|c|}{\cellcolor{color3}\cell{\textbf{Finals}}}
		\\\hhline{~----}

	\cell{Block 2\\\footnotesize(13:00--15:00)}
		& \cellcolor{color1}\cell{\diagbox{Task 3}{Task 4}}
		& \cellcolor{color1}\cell{\diagbox{Task 3}{Task 4}}
		& \cellcolor{color1}\cell{\diagbox{Task 1}{Task 2}}
		& 
		\\\hhline{~---}

	\multicolumn{1}{ c }{}
		& \multicolumn{1}{ c }{\wcell{0.5\baselineskip}{}}
		& \multicolumn{1}{ c }{\wcell{0.5\baselineskip}{\color{gray}\----Break---\-}}
		& \multicolumn{1}{ c }{}
		\\\hhline{~---}
		
	\cell{Block 3\\\footnotesize(17:00--19:00)}
		& 
		& 
		& \cellcolor{color1}\cell{\diagbox{Task 3}{Task 4}}
		& 
		\\\hhline{~---}
	\end{tabular}

	\caption{Example schedule.
		Each team has two Test Slots assigned in every Test Block.
	}
	\label{tbl:schedule}
\end{table}

\begin{table}[h]
\centering
\renewcommand{\arraystretch}{1.4}
\setlength{\tabcolsep}{10pt}
\begin{tabular}{c|c|c|}
\multicolumn{1}{c}{}  & \multicolumn{1}{c}{Arena 1} & \multicolumn{1}{c}{Arena 2} \\
\hhline{~--}
12:00 & \cellcolor{color4}Group A & \cellcolor{color2}Group C \\\hhline{~--}
12:30 & \cellcolor{color3}Group B & \cellcolor{color1}Group D \\\hhline{~--} 
13:00 & \cellcolor{color2}Group C & \cellcolor{color4}Group A \\\hhline{~--} 
13:30 & \cellcolor{color1}Group D & \cellcolor{color3}Group B \\\hhline{~--} 
\end{tabular}
\caption{Example schedule for one block showing timeslots and group rotation between arenas.}\label{tab:schedule_example_times}
\end{table}



\noindent Note that the actual allocation of blocks will be announced by the \OC{} during the \SetupDays{} (see Table~\ref{tbl:schedule}).

\subsection{Scoring System}\label{rule:score_system}

The scoring system for the competition is designed to evaluate robot performance based on task completion. Each task has a primary objective and a set of bonus scores. Teams must complete parts of the main objective to receive any points. Bonuses are not awarded otherwise.

The scoring system has the following constrains:
\begin{enumerate}
	\item \textbf{\FINAL:} The final score is normalized.
	\item \textbf{Minimum score:} The minimum total score per test is \scoring{0 points}.
	While teams generally cannot receive negative points, penalties can result in a total score below zero. Specifically, teams may receive penalties for not attending (see~\refsec{rule:not_attending}) and for extraordinary violations (see~\refsec{rule:extraordinary_penalties}).
\end{enumerate}

