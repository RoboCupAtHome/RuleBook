\section{Organization of the Competition}\label{sec:procedure_during_competition}

\subsection{Competition Structure}\label{rule:structure}

The competition consists of a series of tasks, repeated on each competition day, designed to test specific robot capabilities, followed by a \FINAL{} as the final evaluation of the teams' overall performance.


\begin{enumerate}
	\item \textbf{Robot Inspection:} 
	All teams must pass a \RobotInspection{} during the \SetupDays{} to be eligible for competition (see~\refsec{sec:robot_inspection}).

	\item \textbf{Tasks:}
	Teams compete in a set of tasks that evaluate different robot abilities.
	Each task is repeated on multiple competition days, giving teams more than one opportunity to attempt it.
	For ranking purposes, the \emph{best score} obtained in each task is considered.

	\item \textbf{Restaurant Task:}
The Restaurant task is held only once during the competition, prior to the \FINAL{}. 
It takes place outside the main arena in a location that is not disclosed in advance. 
Preferably in a real restaurant, or otherwise in a restaurant-like environment.  
Due to scheduling constraints, participation in the Restaurant task may require prior registration and may be limited based on the current classification (e.g., score or ranking achieved before the task).

	\item \textbf{\FINAL:} The top teams, namely the ones with the highest overall score, advance to the \FINAL{}.
	The final round features a single integrated task that tests all abilities evaluated throughout the competition.
	The number of teams advancing to the \FINAL{} is determined by the total number of participating teams, as follows:
	\begin{itemize}
		\item Up to 9 teams: top 3 teams advance to the \FINAL{}
		\item 9 to 14 teams: top 4 teams advance to the \FINAL{}
		\item 15 to 20 teams: top 5 teams advance to the \FINAL{}
		\item More than 20 teams: top 6 teams advance to the \FINAL{}
	\end{itemize}

\end{enumerate}

In cases where there is no significant score difference between teams, the \TC{} may decide to include additional teams in the \FINAL{}.

%%%%%%%%%%%%%%%%%%%%%%%%%%%%%%%%%%%%%%%%%%%%%%%%%%%%%%%%%
\subsection{Schedule}\label{rule:schedule}

\begin{enumerate}
	 \item \textbf{Daily Task Runs:} 
	Each competition day features the same set of four tasks.
	Two tasks are executed in parallel in different arenas.
	Once a team completes its run in one arena, it proceeds to the other arena according to the schedule.

	\item \textbf{Team Groups:}
	Teams are assigned to fixed groups (e.g.\ A, B, C, D).
	Each block has predetermined start times for the arenas and tasks.
	This ensures an even distribution of teams across arenas and reduced waiting time.

	\item \textbf{Test Slots:} Within each arena, teams are allocated a specific Test Slot that defines the maximum time available to complete the task. 
    Teams must be ready at their assigned block start time.

	\item \textbf{Participation is default:} Teams must inform the \OC{} in advance if they are skipping a Test Block. If they do not provide such notification, they may be penalized for not attending (see~\refsec{rule:not_attending}).
	
\end{enumerate}

\begin{table}[h]
	\centering\small
	\newcommand{\wcell}[2]{%
		\parbox[c]{2.5cm}{%
			\vspace{#1}%
			\centering%
			#2%
			\vspace{#1}%
		}%
	}
	\newcommand{\cell}[1]{\wcell{0.2\baselineskip}{#1}}
	% \newcommand{\mr}[1]{\multirow{2}{*}{#1}}


	\begin{tabular}{
		>{\centering\arraybackslash}m{2.5cm}|c|c|c|c
	}
	\multicolumn{1}{ c }{}
		& \multicolumn{1}{c}{Day 1}
		& \multicolumn{1}{c}{Day 2}
		& \multicolumn{1}{c}{Day 3}
		& \multicolumn{1}{c}{Day 4}
		\\\hhline{~---~}

	\cell{Block 1\\\footnotesize(9:00--11:00)}
		& \cellcolor{color1}\cell{\diagbox{Task 1}{Task 2}}
		& \cellcolor{color1}\cell{\diagbox{Task 1}{Task 2}}
		& \cellcolor{color2}\cell{Restaurant}
		& 
		\\\hhline{~----}



	\multicolumn{1}{ c }{}
		& \multicolumn{3}{ c }{\wcell{0.5\baselineskip}{\color{gray}\----Break---\-}}
		& \multicolumn{1}{|c|}{\cellcolor{color3}\cell{\textbf{Finals}}}
		\\\hhline{~----}

	\cell{Block 2\\\footnotesize(13:00--15:00)}
		& \cellcolor{color1}\cell{\diagbox{Task 3}{Task 4}}
		& \cellcolor{color1}\cell{\diagbox{Task 3}{Task 4}}
		& \cellcolor{color1}\cell{\diagbox{Task 1}{Task 2}}
		& 
		\\\hhline{~---}

	\multicolumn{1}{ c }{}
		& \multicolumn{1}{ c }{\wcell{0.5\baselineskip}{}}
		& \multicolumn{1}{ c }{\wcell{0.5\baselineskip}{\color{gray}\----Break---\-}}
		& \multicolumn{1}{ c }{}
		\\\hhline{~---}
		
	\cell{Block 3\\\footnotesize(17:00--19:00)}
		& 
		& 
		& \cellcolor{color1}\cell{\diagbox{Task 3}{Task 4}}
		& 
		\\\hhline{~---}
	\end{tabular}

	\caption{Example schedule.
		Each team has two Test Slots assigned in every Test Block.
	}
	\label{tbl:schedule}
\end{table}

\begin{table}[h]
\centering
\renewcommand{\arraystretch}{1.4}
\setlength{\tabcolsep}{10pt}
\begin{tabular}{c|c|c|}
\multicolumn{1}{c}{}  & \multicolumn{1}{c}{Arena 1} & \multicolumn{1}{c}{Arena 2} \\
\hhline{~--}
12:00 & \cellcolor{color4}Group A & \cellcolor{color2}Group C \\\hhline{~--}
12:30 & \cellcolor{color3}Group B & \cellcolor{color1}Group D \\\hhline{~--} 
13:00 & \cellcolor{color2}Group C & \cellcolor{color4}Group A \\\hhline{~--} 
13:30 & \cellcolor{color1}Group D & \cellcolor{color3}Group B \\\hhline{~--} 
\end{tabular}
\caption{Example schedule for one block showing timeslots and group rotation between arenas.}\label{tab:schedule_example_times}
\end{table}



\noindent Note that the actual allocation of blocks will be announced by the \OC{} during the \SetupDays{} (see Table~\ref{tbl:schedule}).

\subsection{Scoring System}\label{rule:score_system}

The scoring system evaluates robot performance based on task completion. Some objectives include optional modifiers and penalties. Modifiers apply only to the specific objective and cannot reduce its score below zero.

The scoring system has the following constraints:
\begin{enumerate}
    \item \textbf{\FINAL:} The final score is normalized.
    \item \textbf{Minimum score:} The minimum total score per test is \scoring{0 points}.
    While teams generally cannot receive negative points, special penalties may result in a total score below zero. Specifically, penalties can be applied for not attending a test (see~\refsec{rule:not_attending}) or for extraordinary violations (see~\refsec{rule:extraordinary_penalties}).
\end{enumerate}

