\chapter{Concepts Behind the Competition}
\label{chap:concepts}

A set of conceptual key criteria builds the basis for the \textsc{RoboCup@Home} competition.
These criteria are to be understood as a common agreement on the general concept of the competition.
The concrete rules are listed in the \AtHome{} Rules \& Regulations document.

\section{Lean Set of Rules}
\label{concept:lean_set_of_rules}

To allow for different, general, and transmissible approaches in the \AtHome{} competition, the rule set should be as lean as possible.
Nonetheless, to avoid rule discussions during the competition itself, it should also be concrete enough to leave no room for diverse interpretations.
If, during a competition, there are any discrepancies or multiple interpretations, a decision will be made by the TC and the referees on site.

\paragraph*{Note:} Once the test scoresheet has been signed or the scores has been published, the TC decision is irrevocable.

\section{Autonomy \& Mobility}
\label{concept:autonomy_and_mobility}

The aim of \AtHome{} is to foster mobile autonomous service robotics and natural human-robot interaction.
Thus, all robots participating in the RoboCup@Home competition must be \emph{mobile} and \emph{autonomous}, which means that humans are not allowed to directly (remotely) control the robot (this also includes verbally remotely controlling the robot).

\section{Aiming for Applications}
\label{concept:aiming_for_applications}

To foster the advance in technology and to keep the competition interesting, the scenario and the tests will steadily increase in complexity.
While necessary individual abilities are still being tested in the competition, tests will focus more and more on real applications with a rising level of complexity and uncertainty.
Useful, robust, general, cost effective, and applicable solutions are rewarded in \AtHome.

\section{Social Relevance}
\label{concept:social_relevance}

The competition and the included tests should produce socially relevant results, as the aim is to convince the public about the usefulness of autonomous robotic applications.
This should be done by showing applications where robots directly help or assist humans in everyday life situations.
Examples of such applications are: a personal robot assistant, a guide robot for the blind, robot care for elderly people, and so forth.
Such socially relevant results are rewarded in \AtHome.

\section{Scientific Value}
\label{concept:scientific_value}

\AtHome{} should not only show what can be put into practice today, but should also present new approaches, even if they are not yet fully applicable or if they demand a very special configuration or setup.
Therefore, a high scientific value of an approach is rewarded.

\section{Time Constraints}
\label{concept:time_constraints}

To allow for many participating teams and tests as well as to foster a simple setup procedures, the setup time and the time for performing the tests is very limited.

\section{No Standardized Scenario}
\label{concept:no_standardized_scenario}

The scenario for the competition should be simple but effective, available world-wide, and at low cost.
As uncertainty is part of the concept, no standard scenario will be provided in \AtHome.
One can expect that the scenario will look typical for the country where the competition is hosted.

The scenario is something that people encounter in daily life; this can be a domestic environment, such as a living room and a kitchen, but also an office space, a supermarket, a restaurant, etc.
The scenario should change from year to year, as long as the desired tests can still be executed.
Furthermore, tests may take place outside of the scenario, that is, in a previously unknown environment, such as a public space nearby.

\section{Attractiveness}
\label{concept:attractiveness}

The competition should be attractive for the audience and the public; thus, high attractiveness and originality of an approach will be rewarded.

\section{Community}
\label{concept:community}

While they have to compete against each other during the competition, the members of the \AtHome{} league are expected to cooperate and exchange knowledge to advance technology together.
The \iterm{RoboCup@Home mailing list} as well as the \RR{} can be used to get in touch with other teams and to discuss league-specific issues such as rule changes, proposals for new tests, etc.
% Since 2007 there is also the \iterm{RoboCup@Home Wiki} (see~\refsec{sec:at_home_wiki}) which serves as a central place to collect information relevant for the @Home league.
In addition, every team is expected to share relevant technical, scientific (and team-related) information in a scientific paper and on the team's website.

Finally, all teams are invited to submit papers on related research to the \Symp{}, which accompanies the annual RoboCup World Championship.

\section{Desired Abilities}
\label{concept:desired_abilities}

The following is a list of desired technical abilities that the tests in \AtHome{} are focusing on:

\begin{itemize}
    \item Navigation in dynamic environments
    \item Fast and easy calibration and setup (the ultimate goal is to have a robot up and running out of the box)
    \item Object recognition
    \item Object manipulation
    \item Detection and recognition of humans
    \item Natural human-robot interaction
    \item Speech recognition
    \item Gesture recognition
    \item Robot applications (\AtHome{} is aiming for applications of robots in daily life)
    \item Ambient intelligence, such as communicating with surrounding devices, retrieving information from the internet, etc.
\end{itemize}


% Local Variables:
% TeX-master: "Rulebook"
% End:
