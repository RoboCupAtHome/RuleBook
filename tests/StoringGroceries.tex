\section{Storing Groceries [DSPL \& OPL]}
The robot helps by storing newly bought groceries in the cupboard next to the objects of the same kind that are already there; for instance by placing fresh apples near other apples.

\subsection{Goal}
The robot has to correctly identify and manipulate objects at different heights, grouping them by category and likelihood.

\subsection{Focus}
This test focuses on the detection and recognition of objects and their features, as well as object manipulation.

% %% %%%%%%%%%%%%%%%%%%%%%%%%%%%%%%%%%%%%%%%%%%%%%%%%%%%%%%
%
% Setup
%
% %% %%%%%%%%%%%%%%%%%%%%%%%%%%%%%%%%%%%%%%%%%%%%%%%%%%%%%%
\begin{minipage}{0.70\textwidth}
	\subsection{Setup}
	\begin{enumerate}
		\item \textbf{Location:} This test can take place either inside or outside the arena. The testing area must have a bookcase or cupboard, and a nearby table. The maximum distance between the Table and the Cupboard is 2 meters.
		\item \textbf{Start position:} The robot starts between the cupboard and the table in a random orientation, but facing towards the Cupboard.
		\item \textbf{Cupboard:} The cupboard has 5 shelves between 0.0m and 1.80m from the ground and contains several objects grouped by category or likeliness (See~\ref{rule:scenario_objects}). The cupboard has at least one free space for starting a new set.
		\begin{itemize}
		 	\item \textbf{Door:} The cupboard has a single door, which is closed initially.
		 	This door encloses some of the objects, covering up to one half of the cupboard (e.g.~the left or bottom half), as indicated by the hatched area in Figure~\ref{fig:storing_groceries_shelf}.
		\end{itemize}
		\item \textbf{Table:} The table has at least 5 objects (but no more than 10). If not all objects fit on the table, they will be added as the robot frees up space.
	\end{enumerate}
\end{minipage}\hfill
\begin{minipage}{0.25\textwidth}
	\begin{figure}[H]
		\centering
		\includegraphics[width=\textwidth]{images/storing_groceries.png}%
		\vspace{-10pt}
		%\caption{Example shelf where objects will be placed.}
		\caption{Shelf}
		\label{fig:storing_groceries_shelf}
	\end{figure}
\end{minipage}

% %% %%%%%%%%%%%%%%%%%%%%%%%%%%%%%%%%%%%%%%%%%%%%%%%%%%%%%%
%
% Task
%
% %% %%%%%%%%%%%%%%%%%%%%%%%%%%%%%%%%%%%%%%%%%%%%%%%%%%%%%%
\subsection{Task}
\begin{enumerate}
	\item \textbf{Evaluating the situation:} The robot inspects its surrounding and analyzing the best course of action. In any order, the robot has to:
	\begin{itemize}
		\item \textit{Inspect the cupboard} (locating and categorizing existing groceries).
		\item \textit{Open the cupboard's door.} If the robot can't open the door, it may ask the Referee to do it.
		\item \textit{Inspect the table} (analyze the newly bought groceries, i.e.~objects).
	\end{itemize}

	\item \textbf{Moving objects:} The robot moves as many objects as possible in the given time
	(only the first five score)
	from the Table to the Cupboard, allocating similar objects all together.
	Stacking is allowed.
	\begin{itemize}
		\item Objects of the same type (i.e.~identical known objects or akin alike objects) must be placed one next to the other.
		\item If the Cupboard has no object of the same type, then objects must be grouped by category (e.g.~drinks with drinks, snacks with snacks, etc)
		\item If the Cupboard has no similar object, the robot must clearly state its decision on how to solve the problem. For instance, the robot can start a new set in a free space for either all unknown objects or all objects sharing a particular feature (color, shape, function, etc.).
		\item Moving two objects at a time (2-handed manipulation) is allowed.
	\end{itemize}

	\textbf{Note:} Either before or after grasping an object the robot may announce the name of the object found.
	\item \textbf{Repeat:} This repeats until the time is up or all groceries are stored.
\end{enumerate}



% %% %%%%%%%%%%%%%%%%%%%%%%%%%%%%%%%%%%%%%%%%%%%%%%%%%%%%%%
%
% Additional Rules
%
% %% %%%%%%%%%%%%%%%%%%%%%%%%%%%%%%%%%%%%%%%%%%%%%%%%%%%%%%
\subsection{Additional rules and remarks}
\begin{enumerate}
	\item \textbf{Bypassing Manipulation:} Bypassing object manipulation via the CONTINUE rule (see~\refsec{rule:asrcontinue}) is not allowed during this test.
	\item \textbf{No setup:} There is no setup time.
	\item \textbf{Startup:} The robot can be started with a simple voice command or via a start button (see~\refsec{rule:start_signal}).
	\item \textbf{Single try:} The robot must be able to start from the first attempt. There is no restart for this test. If the robot is unable to start it must be removed immediately.
	\item \textbf{Collisions:} Slightly touching the cupboard is tolerated (but not advised). Crushing objects or any other form of a major collision terminates the test immediately (see~\refsec{rule:safetyfirst}).
	\item \textbf{Clear area:} The robot may assume that the direct vicinity of the cupboard and table are clear, and that the robot can move slightly backwards for its task.
	\item \textbf{Objects:} The 10 objects are evenly distributed in random fashion including
	3 known objects,
	3 alike objects,
	2 unknown objects, and
	2 special objects (bowl, cloth, dish, etc.).
	\item \textbf{Table} The table's rough location will be announced beforehand, having it's position either left, right, or behind the robot.
	% \item \textbf{Timing:} The robot has to successfully place the first object within the first two minutes, otherwise the test is ended. If the robot opens the cupboard door by on its own, one additional minute is added to the 2-minutes limit. The maximum time for this test is 5 minutes.
\end{enumerate}

% \subsection{Data recording}
% Please record the following data (See~\refsec{rule:datarecording}):
% \begin{itemize}
% 	\item Images
% 	\item Plans
% \end{itemize}

\subsection{OC instructions}

\textbf{2 hours before the test}
\begin{itemize}
    \item Announce the startup location for robots.
    % This test is about manipulation and object recognition, NOT about finding furniture
    % Finding a table is done in the Restaurant task
    \item Announce which table will be used in the test.
    \item Announce a rough location for the table.
\end{itemize}

\subsection{Referee instructions}
The referee needs to
\begin{itemize}
	\item Place the objects in the cupboard and a few of the same class on the table. New items can be placed when there is room or the robot asks for more objects.
	\item Close the door of the cupboard.
	\item Put objects on the table and the corresponding objects in the cupboard: 3 known objects, 2 alike and 5 unknown objects.
\end{itemize}


\newpage
\subsection{Score sheet}

The maximum time for this test is 3 minutes.
The robot is given 1 extra minute to open the cupboard door. 
If the robot is not able to open the door within that minute, it will be opened by the referee. 
In case the robot opens the door within the minute, the robot has a small time advantage. 

\begin{scorelist}
% There are 5 filled shelves, originally with 2 objects, 1 in each corner.
% The table also has 10 objects, that the robot should move to the shelf.
% So 20 objects in total
% This can be a tight fit, as there will be potentially 4 objects per shelf, as the robots moves them from the table one by one

% The robots are not fast enough though to do more than 5 objects in the given time.

% Grasp (any object): 10
% Place (anywhere in the cupboard): 10
% Place in correct place: 15
% Recognize known object correctly (without grasping/placing something of that class): 10
% Label two unknown objects of the same class with the same label (e.g. ``class0''): 15

% Place known object near known object of same class: 40
% Place unknown object near unknown object of the same class: 50


	\scoreheading{Grasping objects}
	\scoreitem[5]{10}{For each successful grasp of any object (lifting it up to at least 5 cm for more than 10 seconds)}

	\scoreheading{Placing objects}
	\scoreitem[5]{10}{For each successful placement of an object anywhere in the cupboard (safely stands still for more than 10 seconds)}
	\scoreitem[5]{5}{For each successful placement of an object at correct place (near an object of the same class)}

	\scoreheading{Recognizing objects}
	\scoreitem[10]{5}{Every correctly recognized known or alike object in the report file}
	% TODO: Split up scores over these 3 variants of a correct label:
	% 1: As unknown, instead of wrongly applying a label from the known or alike objects (e.g. use an Open World assumption in your classifier)
	% 2: Label all instances of the same unknwon class with the same generated label, e.g. label0, so distinguish between different onknown objects
	% 3: Label the unknown objects with a meaningful label, eg. cookies in case its a sort of cookies. I.e. use a classifier that knows many classes. 
	\scoreitem[5]{15}{Correctly label unknown objects}
	\scoreitem[10]{-5}{False positive label}
	
% 	\scoreheading{Total task}
% 	\scoreitem[5]{40}{Place known object near known object of same class}
% 	\scoreitem[5]{50}{Place unknown object near unknown object of same class}

	\scoreheading{Bonus}
	\scoreitem[1]{20}{Open the door without human help}
	
	\setTotalScore{250}
\end{scorelist}


% Local Variables:
% TeX-master: "Rulebook"
% End:


% Local Variables:
% TeX-master: "Rulebook"
% End:
