\section{Navigation}
\label{test:navigation}
The robot must visit a set of waypoints while avoiding obstacles on its path, following a person outside the arena and, finally, guide that person back to the arena.

\subsection{Focus}
This test focuses on tracking and recognizing a previously unknown person, obstacle avoidance, obstacle interaction, and safe navigation in dynamic environments in general.

\subsection{Setup}

\begin{enumerate}
	\item \textbf{Doors:} All doors in the apartment are open, except for the entry door. 
	\item \textbf{Location:} One of the arenas (apartment) and its surroundings. The apartment is in its normal state. Part of the test is performed outside the arena in a public space.
	The arena will likely contain another door that may be used for this test.
	\item \textbf{Operator:} A \quotes{professional} operator is selected by the TC to test the robot during the guiding phases.
	\item \textbf{Other people:} There are no restrictions on other people walking by or standing around throughout the complete task.
	\item \textbf{Path:} A path is setup beforehand and announced, except for Waypoint 4 (later explained).
\end{enumerate}

\subsection{Task}
%%%%%%%%%%%%%%%%%%%%%%%%%%%%%%%%%%%%%%%%%%%%%%%%%%%%%%%%%%%%%%%%%%%%%%
%
% Redundant text. Lets keep it simple. [Mauricio Matamoros]
%
%%%%%%%%%%%%%%%%%%%%%%%%%%%%%%%%%%%%%%%%%%%%%%%%%%%%%%%%%%%%%%%%%%%%%%
% The robot must visit a set of waypoints (rooms, placement locations, furniture, beacons, landmarks, etc.) while avoiding the obstacles on its path. Unless stated otherwise, waypoints may be on the floor. The robot must state when it reached a new waypoint or is not able to reach a waypoint. At the last waypoint inside the arena, the robot must follow a human which will lead the robot outside. After the robot is commanded to stop following, the robot must guide a (different) human back to the apartment.

\begin{enumerate}
	\item \textbf{Entering:} The robot enters the arena.

	\item \textbf{Waypoint 1 (path planning):} After entering the arena, the robot must navigate to \textit{Waypoint 1} that is reachable via, at least, two paths, each one requiring the robot to go through a door which will be shut as the robot approaches. The robot may:
	\begin{itemize}
		\item Take a different path.
		\item Open the closed door.
	\end{itemize}

	\item \textbf{Waypoint 2 (obstacle interaction):} Immediately after reaching \textit{Waypoint 1}, the robot must got to and reach at grasp (or place) distance \textit{Waypoint 2}, a placement location (e.g. a shelf). A large obstacle will prevent the robot from getting close to its destination, having the robot to identify it and interact with it.
	Possible actions include:
	\begin{itemize}
		\item Gently move the obstacle (e.g.~if the obstacle is an object).
		\item Gently ask the obstacle to move away (e.g.~if the obstacle is a human).
		\item Wait for the object to move away by itself (e.g.~if it is unable to identify the type of obstacle).
		% \item Take a different approach to the waypoint. E.g. if the obstacle is a table and one end is unreachable, go to the other end of the table). 
	\end{itemize}
	It must be clear to the referee that the robot has correctly identified the type of obstacle to score points for ``state the nature of the obstacle''. 

	\item \textbf{Waypoint 3 (following a human):} After reaching \textit{Waypoint 2}, the robot must navigate to \textit{Waypoint 3}, a landmark or beacon, where a \textit{Professional Walker} will be waiting. The robot must memorize the \textit{Professional Walker} and follow them outside the arena to \textit{Waypoint 4} which location is unknown.
	\begin{itemize}
		\item \textbf{Training phase:} The robot has to memorize the operator. During this phase, the robot may instruct the operator to follow a certain setup procedure and instruct the operator on what to do when the robot needs to stop following.
		
		\item \textbf{Guiding phase:} When the robot signals that it is ready to start following, the operator starts walking --in a natural way-- through a designated path outside the arena. The robot needs to follow the operator until the operator asks the robot to stop doing so (when \textit{Waypoint 4} has been reached).

		\item \textbf{Resuming:} If the robot loses the \textit{Professional Walker}, it can ask him/her to signal it by waving to resume following, but will be penalized for doing so.

		\item \textbf{Stop following:} Upon reaching \textit{Waypoint 4}, the \textit{Professional Walker} will command the robot to stop following him, using the instructions given by the robot in the training phase.
	\end{itemize}
	
	\item \textbf{Take me back home:} After reaching \textit{Waypoint 4} the robot must guide back the \textit{Professional Walker} to \textit{Waypoint 3}. If the robot is unable to guide a human, it may go back to \textit{Waypoint 3} alone.
	\begin{itemize}
		\item \textbf{Instructing phase:} The robot may instruct the operator to follow a certain setup procedure and give some guidelines while following it back.
		
		\item \textbf{Guiding phase:} When the robot signals it is ready to start guiding, the operator starts walking --in a natural way-- following the robot and its instructions until the robot indicates the operator they have reached their destination. The robot may ask to the operator, for instance, to go through a door before it, rise a hand, or wait.

		%%%%%%%%%%%%%%%%%%%%%%%%%%%%%%%%%%%%%%%%%%%%%%%%%%%%%%%%%%%%%%%%%%%%%%
		%
		% Operator lost during "Take me back home". [Mauricio Matamoros]
		%
		% I would like to add this to ensure the robot hasn't lost track of
		% the operator while stopping or walking away.
		%
		%%%%%%%%%%%%%%%%%%%%%%%%%%%%%%%%%%%%%%%%%%%%%%%%%%%%%%%%%%%%%%%%%%%%%%
		Two people will board the operator, blocking her path and preventing her from following the robot. The robot must locate its operator and resume taking her back home. 
		
		The blocking people will clear again after it is clear the robot has lost the operator and is looking for the operator.

		\item \textbf{End phase:} Once back at \textit{Waypoint 3}, the robot announces the operator is back home. 
	\end{itemize}
	
	\item \textbf{Leaving the arena:} The robot must leave the arena through the indicated door.
\end{enumerate}

\begin{figure}[tbp]
	\centering
	\includegraphics[width=0.5\columnwidth]{images/navigation.png}
	\caption{Navigation test: setup and execution example.}
	\label{fig:restaurant}
\end{figure}

\paragraph*{Remarks:}
\begin{enumerate}
	\item Depending on the layout of the arena, waypoint 1 and 2 may be swapped.
	\item Reaching a waypoint also includes the direction in which the robot should be looking when it reaches, which will be announced by the TC during the setup of the path.
	\item The distance between Waypoints 3 and 4 is about 10-20 meters.
\end{enumerate}

\subsection{Obstacles}
While navigating to waypoints 1, 2, and 3 the robot will find one of the following obstacles on its path:
\begin{itemize}
		\item \textbf{Small object:} Box sized object (between 5 and 15 cm per edge).  
		\item \textbf{3D Object:} A bar table, normal table, rolling chair: some object that is wider at its top than on its bottom, 
		  thus requiring more than just a laser scanner mounted near the ground to avoid obstacles.
		\item \textbf{Smart obstacle:} A person to whom the robot may speak to and kindly ask to move away. When interacting with people, the robot must look at the person and make clear is speaking with him/her.
	\end{itemize}

\subsection{Additional rules and remarks}
\begin{enumerate}
	\item \textbf{Waypoints:} Waypoints may be rooms, placement locations, furniture, beacons, landmarks, etc. The robot must clearly state when it has reached a waypoint or if it was not able to reach the waypoint.

	\item \textbf{Show must go on:} If a robot is unable to reach a waypoint, it must say it and proceed to the next one.

	\item \textbf{Closing doors:}  The door that will be shut will be the door on the route the robot has committed to. It will be shut right after the robot starts driving towards the door, but granting enough time to notice that the door is now closed.	

	\item \textbf{Moving objects:} If the robot finds on its way a \textit{static movable obstacle} (chair, cubes, toys, etc.) which is capable to move, it must announce is going to move an obstacle and then proceed to move the object apart with its manipulator, or by \textbf{gently} pushing it with its body.

	\item \textbf{Asking people to move away:} If the robot finds on its way a person blocking its path, it must announce it has found a person, \textit{gently} ask that person to move away and wait for the path to be clear. \textbf{Robots are not allowed to touch people}.

	\item \textbf{Following people:} 
	\begin{enumerate}
		\item \textbf{Instruction:} The robot interacts with the operator, \emph{not} the team. That is, the team is not allowed to instruct the operator.
		\item \textbf{Natural walking:} The operator has to walk \quotes{naturally}, i.e., move forward facing forward. 
		  The operator is not allowed to walk back, stand still, signal the robot or follow any re-calibration procedure.
		\item \textbf{Asking for passage:} The robot is allowed to (gently) ask people to step aside.
	\end{enumerate}
	
	\item \textbf{Guiding people:} 
	\begin{enumerate}
		\item \textbf{Instruction:} The robot interacts with the follower, \emph{not} the team. That is, the team is not allowed to instruct the follower.
		\item \textbf{Guidance:} The robot is allowed to interact with the operator during the guidance, asking her to stand still, make a gesture, step aside, or walk through a door. However is up to the operator to decide when to attend robot's instructions.
		\item \textbf{Asking for passage:} The robot is allowed to (gently) ask people to step aside.
	\end{enumerate}
\end{enumerate}

\subsection{Data recording}
  Please record the following data (See \refsec{rule:datarecording}):
  \begin{itemize}
   \item Mapping data
   \item Plans
  \end{itemize}

\subsection{Referee instructions}

The referee needs to
\begin{itemize}
	\item Instruct the OC and volunteers on when and where locate objects.
	\item Instruct the OC and volunteers on when and which doors must be closed.
	\item Stop the robot immediately when it is about to collide.
\end{itemize}

\subsection{OC instructions}

\textbf{2 hours before the test}
\begin{itemize}
	\item Announce the locations for waypoints 1, 2, and 3.
	\item Establish location for waypoint 4 and the path for the \textit{follow me} phase. 
\end{itemize}

\textbf{During the test}
\begin{itemize}
	\item Open and close the doors when instructed by the referee.
	\item Place the obstacles (or act as an obstacle) when instructed by the referee.
\end{itemize}

\newpage

\subsection{Score sheet}

The maximum time for this test is 5 minutes.

\begin{scorelist}

	\scoreheading{Waypoint 1}
	\scoreitem{50}{Opening the door and continue instead of plan a new trajectory}
	\scoreitem{10}{Reaching waypoint 1}

	\scoreheading{Waypoint 2}
	\scoreitem{10}{Detecting and asking a person to step aside}
	\scoreitem{50}{Moving aside an object to reach the waypoint}
	\scoreitem{10}{Reaching waypoint 2 (grasp distance)}

	\scoreheading{Waypoint 3}
	\scoreitem{5}{Start following the \textit{Professional Walker}}
	\scoreitem{20}{Reaching again waypoint 3 after reentering the arena (i.e.~after reaching waypoint 4)}

	\scoreheading{Waypoint 4}
	\scoreitem{15}{Reaching waypoint 4}

	\scoreheading{Avoiding objects}
	\scoreitem{10}{Avoiding box-sized object}
	\scoreitem{10}{Avoiding 3D object (Difficult-to-see object)}

	\scoreheading{Leaving the arena}
	\scoreitem{10}{Leaving the arena}

	\setTotalScore{200}
\end{scorelist}

% Local Variables:
% TeX-master: "Rulebook"
% End:


% Local Variables:
% TeX-master: "Rulebook"
% End:
