\chapter{GPSR in detail}
\label{chap:gpsr-appendix}

\section{Command Generation}
General Purpose Service Robot commands are generated randomly using the official [EE]GPSR Command Generator and grammars publicly available at https://github.com/kyordhel/GPSRCmdGen. The official [EE]GPSR Command Generator and the official grammars will be made available two months before the competition. However, teams must be aware that the categories, objects and other data is provided for testing purposes only.

For each command to be executed, the Team Leader must choose a Command Category, namely Category I, Category II, or Category III. If the Team Leader knows \textit{a priori} that the robot won't be able to execute the generated command, is advised to inform the operator immediately in order to proceed with the next command, saving this way valuable time for the task execution.

\section{Command retrieval explained}
The robot has to show it has understood the given command by stating all the required information to accomplish the task. For this purpose, the robot may repeat the understood command and ask for confirmation. It is not required to repeat the command word by word; rephrasing the command is allowed. For instance, if the robot is instructed to \enquote{place a coke onto the tray}, the robot may either say: \textit{\enquote{You want me to place a coke on the tray. Is that correct?}} or \textit{\enquote{do you want me to deliver a coke to the tray?}}.

If The robot can't correctly recognize the given command, it is allowed to request the operator to repeat the command up to three times. After three failed attempts, a new command is generated. The team may opt to use a custom operator or bypassing speech recognition (\refsec{rule:asrcontinue}) at any time, but each generated command will be given to the robot no more than three times. Only three different commands are generated for a robot, if the robot fails to recognize all three commands (i.e.~nine attempts), the test ends immediately.

When a robot has partially understood the command, it is allowed to ask the operator for additional information (e.g.~\textit{\enquote{did you say apple juice or pineapple juice?}}).

\subsection{Missing information}
When a given command lacks of information required for accomplishing the task, the robot should request for that missing part. For instance, if the robot is instructed to \textit{\enquote{offer a drink to the person at the door}}, it may ask \textit{\enquote{which drink should I deliver to the person at the door?}} It is also possible that the robot simply confirms the command and takes a random drink from the drinks location, but in those cases, the jury will consider the command as if it were from an inferior/lower category.

\subsection{Wrong information}
Some Category III commands contains erroneous information. In these cases, the robot should
\begin{itemize}
	\item be able to realize such an error while trying to carry out the task, get back to the operator, and clearly state why it wasn't able to accomplish the task; or
	\item be able to solve the problem by means of an alternative, reasonable solution.
\end{itemize}

For example, lets assume the robot is commanded to \textit{\enquote{move the orange juice from the fridge to the dinner table}}, but in the fridge there are only the apple juice and the milk, while the orange juice lies in the stove. The robot may either explain to the operator that there are no orange juices in the fridge, or search the kitchen for the orange juice, grasp it from the stove and deliver it to the dinner table.

\section{Command categories explained}
All possible actions has been classified previously by the TC according to their difficulty. For each of the three given command, the team may choose from the following categories:

\subsection{Category I}
\label{chap:gpsr-appendix-cat1}
This category comprehends easy-to-solve tasks with a low difficulty degree, involving indoor navigation, grasping known objects, answering questions (from the predefined set of questions), etc.

Some examples are:
\begin{itemize}
	\item \textit{Tell me how many beverages are in the shelf.}
	\item \textit{Put the crackers on the kitchen table.}
	\item \textit{Tell the time to Ana at the bedroom.}
	\item \textit{Tell me the name of the person at the door.}
	\item \textit{Bring me the apple juice from the counter.}
\end{itemize}

\subsection{Category II:}
\label{chap:gpsr-appendix-cat2}
Tasks with a moderate difficulty degree. This category involves following a human, indoor navigation in crowded environments, manipulation and recognition of alike objects, find a calling person (waving or shouting), etc.

Some examples are:
\begin{itemize}
	\item \textit{Tell me how many beverages in the shelf are red.}
	\item \textit{Put the banana on the kitchen table.}
	\item \textit{Count the waiving people in the living room.}
	\item \textit{Follow Ana at the entrance.}
	\item \textit{Tell me the name of the woman in the kitchen.}
\end{itemize}


\subsection{Category III:}
\label{chap:gpsr-appendix-cat3}
This category comprehends challenging tasks involving dealing with incomplete information, environmental reasoning, feature detection, natural language processing, outdoors navigation, pouring, opening doors, etc.

The commands generated for this category heavily depends on the League and are detailed as follow.

\subsubsection{Advanced manipulation [DSPL and OPL]}
Some examples are:
\begin{itemize}
	\item \textit{Pour some cereals in the bowl.}
	\item \textit{Go to the bathroom} (Bathroom's door is closed).
	\item \textit{Bring me the milk from the microwave} (The milk is inside the microwave)
\end{itemize}

\subsubsection{Incomplete and erroneous information [All Leagues]}
These commands are almost the same as the ones of categories I and II, but either the information given is incorrect or incomplete. This means that executing the command as it has been given is not possible. The robot must come up with an appropriate solution to execute the operators' command.

Some examples are:
\begin{itemize}
	\item \textit{Follow John} (John's location is not specified).
	\item \textit{Bring me a drink} (The exact drink is not specified).
	\item \textit{Bring some snacks to Mary} (Neither Mary's location nor the snack are specified).
	\item \textit{Find Ana at the bedroom and tell her the time} (Ana is lying on the floor or standing under the door frame).
	\item \textit{Bring me a drink from the fridge} (There are no drinks in the fridge, but in the kitchen table).
\end{itemize}

\subsubsection{Other tasks [All Leagues]}
Some examples are:
\begin{itemize}
	\item \textit{Follow me and then go to the kitchen} (Operator takes the robot to the audience area).
	\item \textit{Give me the left most object from the shelf.}
	\item \textit{Count the drinks on the table.}
	\item \textit{Tell me how many girls there are in the living room.}
\end{itemize}


\section{Bypassing commands and alternate solutions}
The General Purpose Service Robot is a goal-driven test in which the final results has priority over how the command is executed.
This adds several degrees of freedom to make a plan and execute a command accordingly with the robot's capabilities.

For instance, consider the following command:

\begin{center}
\noindent\textit{Bring me a coke}
\end{center}

It is clear that the operator wants a coke and cares little about how the coke is retrieved. Now, let's say that the robot's manipulator is broken, so it won't be able to handle a coke. In this case, several scenarios become evident:

\begin{itemize}
	\item \textbf{Skipping command:} The robot says \enquote{I understood you want me to bring you a coke, but I cannot grasp objects, so I'll skip this command}. Since the robot is not executing the task, no score is given.

	\item \textbf{Continue Rule:} The robot more or less reaches the position, fuzzily points at the object, and then requests to a human assistant to deliver the coke for it. In this case, the referee might grant up to $\frac{1}{3}$ of the points, if any.

	\item \textbf{Requesting human assistance:} Taking advantage of the Continue Rule, the robot requests assistance from a human to grasp the object, requesting later to follow it. During the guiding phase, the robot actively tracks the human to the operator's position and supervises the delivery (e.g.~telling it noticed the operator has received the coke). In this case, and regarding the execution of the tasks, the referee may grant a full score.

	\item \textbf{Social alternative:} The robot looks for another person in the arena, finds them, and convinces them (or socially bribes them) to deliver a coke to the operator using natural language dialogs. In these rare cases, the referee may grant a full score depending on the success of the interaction.
\end{itemize}
