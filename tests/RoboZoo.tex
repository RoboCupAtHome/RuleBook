\newcommand{\roboZooTokens}{$N$}

\section{RoboZoo}
\label{sec:test_robo_zoo}

During the RoboZoo teams are encouraged to demonstrate the usefulness of robots in home, hospital and office environment to an open audience.

\subsection{Setup}
The robots of all teams are presented and arranged in a way that all of them form a zoo-type corridor (or hall) through which the general audience may walk.

The enclosed space is estimated to be around $2 \times 2$ meters. However, teams should expect reasonable deviations in these dimensions, since space in the venue may require smaller enclosed spaces.

\subsection{Task}
The RoboZoo consist of a demonstration of the usefulness of robots to an open audience. It is an open demonstration which means that the teams may demonstrate anything they like, continuously for up to one hour and without leaving its \textit{security cage}.

General audience evaluates the overall performance and appearance of the robots, while a jury consisting of the TC and EC evaluate the skills demonstrated during the test.

Interaction with the audience is desirable but not mandatory.

\subsection{Jury \& evaluation}
General audience evaluates the overall performance and appearance of the robots. Each member of the audience who enters the corridor will receive \roboZooTokens tokens which will be given to their top favorite robots on both: appearance and performance. Earned tokens are normalized regarding the maximum amount of tokens given to a robot per category. Therefore scoring in each category is proportional to the amount of tokens per category.

A technical Jury (TC and EC) will also evaluate the performance of the robots, granting up to 10 points per ability shown with a maximum of 50 points in the following abilities:
\begin{itemize}
\item Gesture recognition.
\item Human-Robot interaction.
\item Manipulation (single handed).
\item Manipulation (two handed).
\item Object detection and recognition.
\item People detection and recognition.
\item Speech recognition.
\item Task planning.
\end{itemize}

The final score is as follows.

$$\mbox{score}_i = 25 \times \frac{\mbox{appearance}_i}{max(\mbox{appearance})} + 
25 \times \frac{\mbox{performance}_i}{max(\mbox{performance})} + min(50, \sum\mbox{abilities-shown}_i)
$$

\subsection{Security Concerns}
Security is first priority in this competition. To this effect, one team member is required to be inside the enclosed space to ensure that the robot is performing securely. People from the audience won't be allowed inside the enclosed space at any moment.

It is not allowed for robots to touch people. Robots \textbf{must not touch people}.

%%%%%%%%%%%%%%%%%%%%%%%%%%%%%%%%%%%%%%%%%%%%%%%%%%%%%%%%%%%%%%%%%%%%%%%%%%%%%
%
% I removed the restriction of robot handling objects to the audience
% since in stage 2 now we have handovers. I don't see it happening.
% Mauricio
%
%%%%%%%%%%%%%%%%%%%%%%%%%%%%%%%%%%%%%%%%%%%%%%%%%%%%%%%%%%%%%%%%%%%%%%%%%%%%%

\paragraph*{Important Note:} Even while people are not allowed to enter the robots' cages, it may happen people (specially small children) get into the cages.  In those cases, robot must be stopped immediately.

\subsection{Additional rules and remarks}

\begin{itemize}
\item \textbf{People in the cage:} At any time, one team member must be inside the cage to take care of the robot. 

\item \textbf{Protagonist robots:} Robots must be able to perform autonomously during the test. Team members are \emph{not} allowed to interact with the audience, teach instructions, take part of the show, etc. More than one team member inside the cage is \emph{not} allowed.

\item \textbf{Restart:} There is no restart limit nor penalty for restarting a robot. However, it is important to note that this test is essentially scored by the general public, and it is reasonable to expect that the audience will not be attracted to a robot being constantly fixed. 

\item \textbf{Charging:} It is allowed to the team to change robot's batteries or to use a charging station during the RobboZoo, however, all equipment must be inside the cage. 

\item \textbf{Gifts:} Robots and team member are \emph{not} allowed to hand out gifts as part of the RoboZoo challenge. In the case that team members are lurking around the cages or interacting with the audience, that team will be disqualified.

\item \textbf{Language:} RoboCup is an international competition. This means all robots have to interact with the audience in English. 
\end{itemize}

\subsection{Data recording}
No data required to be recorded during the RoboZoo challenge.

\subsection{OC instructions}

2h before test:
\begin{itemize}
\item Announce to teams the dimension of the enclosed spaces.
\item Announce where the presentation will take place.
\item Announce which space will be occupied by which robot.
\end{itemize}

During the test
\begin{itemize}
\item Provide tokens to the audience.
\item Stop unattended robots (by pressing the emergency button).
\end{itemize}

\subsection{Score Sheet}
The maximum time for this test is 60 minutes.

Robots are scored on functionality and on design. The audience can awards tokens for what they elect to be the \textbf{Most functional robot} and the \textbf{Best looking robot}. \textit{TC} and \textit{EC} evaluates technical performance.

\renewcommand{\dataRecordingBonus}{false}
\begin{scorelist}
	\scoreheading{Audience rating}
	\scoreitem{25}{Robot's appearance}
	\scoreitem{25}{Robot's performance}

	\scoreheading{Technical performance (max 50)}
	\scoreitem{10}{Gesture recognition.}
	\scoreitem{10}{Human-Robot interaction.}
	\scoreitem{10}{Manipulation (single handed).}
	\scoreitem{10}{Manipulation (two handed).}
	\scoreitem{10}{Object detection and recognition.}
	\scoreitem{10}{People detection and recognition.}
	\scoreitem{10}{Speech recognition.}
	\scoreitem{10}{Task planning.}
	\scoreitem{10}{Other \hfill}

	\setTotalScore{100}
\end{scorelist}

\paragraph*{Normalization}: Teams get score proportional to the best team of the category:

$$\text{score for this team} = 25 \times \frac{t_{this}}{t_{best}}$$
where $t_{this}$, $t_{best}$ is the number of tokens received by this team, and the number of tokens received by the best team.

\renewcommand{\dataRecordingBonus}{true}
% Local Variables:
% TeX-master: "Rulebook"
% End:



% Local Variables:
% TeX-master: "Rulebook"
% End:
