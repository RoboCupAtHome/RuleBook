\section{Enhanced Endurance General Purpose Service Robot}

This test evaluates Human-Robot Interaction and the integration of advanced capabilities of two robots in parallel, for an extended amount of time. 
There is no predefined story and there is neither a predefined order of tasks. 
The actions that are to be carried out by the robot are randomly generated by the referees, 
  based on a known set of actions that the robots must be able to perform.

A robot must show it has recognized as command and may repeat the understood command and ask for confirmation. 
If it can't recognize the command correctly, it can also ask the speaker to repeat the complete command, or ask for further information.

\subsection{Focus}
This test particularly focuses on the following aspects:
\begin{itemize}
	\item No predefined order of actions to carry out (to get away from state machine-like behavior programming).
	\item Increased complexity in speech recognition (possible commands are less restricted in both actions/operators and arguments/objects, 
	  commands can include multiple objects, e.g., \quotes{put the apple on the kitchen table})
	\item More advanced capabilities (e.g. describing objects, pouring)
	\item Responding to events
	\item Command dependencies. The robot may be asked to transport an item from A to B. 
	  In another command it may be asked to transport the same item to loction C. 
	  The robot should be smart enough to know thatthe item can be grabbed at location B.
\end{itemize}

\subsection{Task}

\begin{enumerate}
	\item \textbf{Entering and command retrieval:} The robot enters the arena and drives to a designated position where it has to wait for further commands.
	\item \textbf{Command generation:} A command is generated randomly, depending on the command category chosen by the team (see below).
	\item \textbf{Task assignment:} The robot is given a command by the operator and may directly start to work on the task assignment. 
	The robot must must prove it has understood the given command by repeating it (Please see the remarks about this in section~\ref{sec:eegpsr_remarks}).
	If a robot is unable to perform a command, it must say so. 
	\item \textbf{Finishing a command:} After finishing a task, the must wait for a new command.
	\item \textbf{Exiting the arena:} When commanded to do so, a robot may leave the arena. 
\end{enumerate}

\subsection{Commands and actions}


The robots must be able to perform the following actions:
\begin{itemize}
 \item Waking someone up
 \item Describing objects at a given location
 \item Grab described object
 \item Pour into a bowl
 \item Take an item from a human
 \item Give an item to a human using a natural handover
 \item Answer a question
 \item Find a person in a given room
 \item Find and bring an object of a given type at a given location
 \item Bring an item to a given location
\end{itemize}


\subsection{Events}
At any time during the test, one of the events below may occur. 
Events will only occur in the field of view of the robot. 
This restriction will be loosened in future competitions. 

When an event occurs, the robot must ask the actor in the event what to do. 
\begin{enumerate}
 \item Someone falls on the ground
 \item Walking away
 \item Standing up
 \item Sitting down
 \item Waking up
 \item Waving
\end{enumerate}


\paragraph{Command examples}
\begin{enumerate}
 \item
\end{enumerate}

\subsection{Additional rules and remarks}
\label{sec:eegpsr_remarks}
\begin{enumerate}
  \item Robots are not scored per command but rather by subcommands and abilities shown. 
\end{enumerate}

\subsection{Referee and OC instructions}
\textbf{2h before test:}
\begin{itemize}
\item Specify and announce the entrance door for each robot. 
\item Specify and announce the waiting position for each robot. 
\end{itemize}
\textbf{During the test:}
\begin{itemize}
\item Generate random sentences by an automatic sentence generator
\end{itemize}

\newpage
\subsection{Score sheet}
The maximum time for this test is 10 minutes.
%
% MAURICIO 2017
% Compact Scoresheet
%
\begin{scorelist}
	\scoreheading{Getting instructions}
	\scoreitem[3]{10}{Understanding the command on the $1^{st}$ attempt}
	\scoreitem[3]{ 5}{Understanding the command on the $1^{st}$ attempt (Custom Operator)}

	\scoreheading{Complete Command Successfully Solved}
	\scoreitem{ 30}{Command Category I}
	\scoreitem{ 50}{Command Category II/III}

	\scoreheading{Incomplete Command Successfully Solved}
	\scoreitem{ 50}{Command Category I}
	\scoreitem{ 80}{Command Category II/III}
	\scoreitem{ 20}{Retrieving missing information}

	\scoreheading{Erroneous Command Successfully Solved}
	\scoreitem{ 70}{Command Category I}
	\scoreitem{100}{Command Category II/III}
	\scoreitem{ 20}{Explain nature of error (regardless command execution)}
	
	% \scoreheading{Leave the arena}
	% \scoreitem{10}{Leave the arena after successfully accomplishing a command}

	\setTotalScore{300}
\end{scorelist}

% Local Variables:
% TeX-master: "Rulebook"
% End:


% Local Variables:
% TeX-master: "Rulebook"
% End:
 
