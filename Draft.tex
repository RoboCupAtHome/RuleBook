\documentclass[a4paper]{article}
\usepackage[utf8x]{inputenc}
\usepackage{times}
\usepackage{helvet}
\usepackage{courier}

\usepackage{url}
\usepackage{graphicx}
\usepackage{color}
\usepackage{tabularx}
\usepackage{caption}
\usepackage{subcaption}
\usepackage{epstopdf}
\usepackage{subfig}
\usepackage{float}
\usepackage{amsmath}
\usepackage{wrapfig}
\usepackage{titlesec}
\usepackage{xfrac}

\newcommand{\quotes}[1]{``#1''}
\newcommand{\textbi}[1]{\textbf{\textit{#1}}}
\newcommand{\sectionbreak}{\clearpage}
\newcommand{\subsectionbreak}{\clearpage}

\begin{document}

\section{Introduction}
\section{Concepts behind the competition}
\section{General rules \& Regulations}
\section{Setup and preparation}

\section{Tests in Stage I}

\begin{itshape}
All ability and integration tests in Stage I grants 15 points and are performed 3 times. The total score for each of this tests is the average of the best two performances (still to be discussed, perhaps taking only the top 1 or --unlikely-- global average).

Open/Demo Challenge and RoboZoo are special cases. Those tests grant up to 10 points each, and have no changes from previous years. In case of Open/Demo Challenge (most likely is going to be a focused Demo Challenge), scoring will be redefined.
\end{itshape}

\section{General Purpose Service Robot}
\label{test:gpsr}
The robot can be asked to do anything involving abilities required in \SONE{} of this rulebook.

\noindent \textbf{Focus:} \SysI.

\subsection*{Main Goal}
Execute 3 commands requested by the operator.

\noindent\textbf{Reward:} 750pts (250 points per command).

\subsection*{Bonus Rewards}
\begin{enumerate}[nosep]
	\item Understand a command given by naive operator (50pts, each).
	\item Autonomously leaving the arena through the exit (100pts).
\end{enumerate}

% %% %%%%%%%%%%%%%%%%%%%%%%%%%%%%%%%%%%%%%%%%%%%%%%%%%%%%%%
%
% Setup
%
% %% %%%%%%%%%%%%%%%%%%%%%%%%%%%%%%%%%%%%%%%%%%%%%%%%%%%%%%
\subsection*{Setup and Procedure}
\begin{enumerate}[nosep]
	\item \textbf{Location:} The test mostly takes place inside the arena (some commands might require the robot to go outside).
	
	\item \textbf{Instruction Point:} An instruction point is assigned inside the arena where the robot has to go to receive a command.
	
	\item \textbf{Operators:} A \emph{professional} operator (i.e.~the referee) gives the commands. Optionally, teams can choose to receive commands by a \emph{naive} operator who has no background in robotics, i.e., an audience member. If the robot consistently fails to understand a command, the team can use a \CustomOperator{}.
	
	\item \textbf{Command Generator:} Commands will be generated using the official \CommandGen{} available 2 months prior to the competition in the official repository\footnotemark \footnotetext{\url{https://github.com/kyordhel/GPSRCmdGen}}.
\end{enumerate}


% %% %%%%%%%%%%%%%%%%%%%%%%%%%%%%%%%%%%%%%%%%%%%%%%%%%%%%%%
%
% Additional Rules
%
% %% %%%%%%%%%%%%%%%%%%%%%%%%%%%%%%%%%%%%%%%%%%%%%%%%%%%%%%
\subsection*{Additional Rules and Remarks}
\begin{enumerate}[nosep]
	\item \textbf{Partial scoring:} The main task allows partial (per command) scoring.
		
	\item \textbf{Deus ex Machina:} Score reduction applies per given command as follows:
	\begin{itemize}[nosep]
	\item \textbf{Custom Operator:} Choosing a custom operator causes a score reduction of 50pts.
	\item \textbf{Further Assistance:} Helping a robot accomplish a task causes a score reduction in line with \ref{sec:rules:demscoring} and other \SONE{} tests.
	\end{itemize}
\end{enumerate}


\subsection*{Referee Instructions}
\begin{itemize}[nosep]
	\item Provide the commands to the operators.
	\item Make sure the \Arena{} is in normal condition.
\end{itemize}


\subsection*{OC Instructions}
\textbf{2 hours before the test}
\begin{itemize}[nosep]
 	\item Generate commands.
	\item Announce the location of the instruction point.
	\item Recruit volunteers to assist during the test.
\end{itemize}


\subsection*{Score Sheet}
\begin{scorelist}[timelimit=7]
	\scoreheading{Main Goal}
	\scoreitem[3]{80}{Understand the spoken command}
	\scoreitem[3]{100}{Demonstrate a plan has been generated}
	\scoreitem[3]{250}{Solving the command}
	
	\scoreheading{Bonus Rewards}
	\scoreitem{200}{Interleaved Task Bonus}

	\scoreheading{Penalties}
	\penaltyitem[3]{20}{Using a custom operator}
	\penaltyitem[6]{30}{Request a rephrasing}
	\penaltyitem[3]{50}{Bypassing speech recognition}
	\penaltyitem[3]{250}{Human assistance: will apply a percentage penalty according to similar penalties in other tests.}
\end{scorelist}




% Local Variables:
% TeX-master: "Rulebook"
% End:



The maximum time for this test is 3 minutes.

\begin{scorelist}

	\scoreheading{Grasping objects}
	\scoreitem[5]{10}{Grasping any object (and successfully lifting it up to at least 5 cm for more than 10 second)}

	\scoreheading{Placing objects}
	\scoreitem[5]{10}{Placing any object (safely and the objects stands still for more than 10 second)}

	\scoreheading{Recognizing objects}
	\scoreitem[5]{10}{Every correctly recognized object in the report file}
% 	\scoreitem[5]{-5}{False positives (labeling non-objects like an edge of the shelf) in the report file}

	\scoreheading{Hidden object optional (up to 50 points)}
	\scoreitem{50}{Finding a hidden or occluded object}

	%\setTotalScore{200}
\end{scorelist}


\footnotetext{The minimum and maximum distance of the objects in the shelf is still being discussed. This value may change. }


% Local Variables:
% TeX-master: "Rulebook"
% End:



The maximum time for this test is 5 minutes.

\begin{scorelist}

	\scoreheading{Waypoint 1}
	\scoreitem{50}{Opening the door and continue instead of plan a new trajectory}
	\scoreitem{10}{Reaching waypoint 1}

	\scoreheading{Waypoint 2}
	\scoreitem{10}{Detecting and asking a person to step aside}
	\scoreitem{50}{Moving aside an object to reach the waypoint}
	\scoreitem{10}{Reaching waypoint 2 (grasp distance)}

	\scoreheading{Waypoint 3}
	\scoreitem{5}{Start following the \textit{Professional Walker}}
	\scoreitem{20}{Reaching again waypoint 3 after reentering the arena (i.e.~after reaching waypoint 4)}

	\scoreheading{Waypoint 4}
	\scoreitem{15}{Reaching waypoint 4}

	\scoreheading{Avoiding objects}
	\scoreitem{10}{Avoiding box-sized object}
	\scoreitem{10}{Avoiding 3D object (Difficult-to-see object)}

	\scoreheading{Leaving the arena}
	\scoreitem{10}{Leaving the arena}

	\setTotalScore{200}
\end{scorelist}

% Local Variables:
% TeX-master: "Rulebook"
% End:


\subsection{Open/Demo Challenge}

The maximum time for this test is 5 minutes.

\begin{scorelist}

	\scoreheading{Operator}
	\scoreitem{30}{Approach or point at the operator}
	\scoreitem{30}{Correctly state operator's gender}
	\scoreitem{30}{Correctly state operator's pose}

	\scoreheading{Crowd}
	\scoreitem{20}{Correctly state crowd's size}
	\scoreitem{20}{Correctly state crowd's number of men}
	\scoreitem{20}{Correctly state crowd's number of women}
	
	\scoreheading{Bonus}
	\scoreitem{5}{Learn the operator's name}
\end{scorelist}


% Local Variables:
% TeX-master: "Rulebook"
% End:


\subsection{RoboZoo}

The maximum time for this test is 5 minutes.

\begin{scorelist}

	\scoreheading{Direct speech recognition}
	\scoreitem[8]{15}{Correctly answered a question}

	\scoreheading{Indirect speech recognition}
	\scoreitem[8]{15}{Correctly answered a question}
	\scoreitem[8]{10}{Turned towards person asking the question}
	\scoreitem[8]{-5}{Penalty for asking for repetition}

	\setTotalScore{320}
\end{scorelist}

% Local Variables:
% TeX-master: "Rulebook"
% End:


\section{Tests in Stage II}

\begin{itshape}
All ability and integration tests in Stage II grants 25 points and are performed only once. Some tests --like Wake-me-up Test-- have optional tasks that grant additional points when performed correctly, clean and fast. TC must be informed if a team is planning to perform any of the optional tasks. No additional time is given while performing optional tasks.
\end{itshape}

\section{Restaurant}
\label{test:restaurant}
The robot takes and serves orders to several customers in a real restaurant.

\noindent \textbf{Focus:} \SysI{}, \NAV{}, \MAP{}, \HRI{}, \MAN{}, \PerDet{}, \OR{}.

\subsection*{Main Goal}
Take and serve orders from 2 customer.

\noindent\textbf{Reward:} 1000pts (500pts per order)

\subsection*{Bonus Rewards}
\begin{enumerate}[nosep]
	\item Use an unattached tray to transport an order(250pts each)
	\item Transport two orders at once (250pts)
\end{enumerate}

\subsection*{Setup and Procedure}
\begin{itemize}[nosep]
	\item \textbf{Location:} A real restaurant fully equipped and in business. There may be real customers and waiters around. The location is not announced beforehand.

	\item \textbf{Start Location:} The robot starts next to the bar.

	\item \textbf{Customers:} There are multiple tables with people. At least two people at different tables have orders. People who want to order wave or call the robot.
	
	\item \textbf{Orders:} People can order one or two objects. Orders must be placed on the customer's table.

    \item \textbf{Bar:} A table located near the restaurant's kitchen on which objects that can be ordered are placed. All edible/drinkable \KnownObjects{} can be ordered.

	\item \textbf{Barman:} A member of the \abb{TC} is standing by the bar to assist the robot on request. The barman will handover orders or place them in a basket or tray.
\end{itemize}

\subsection*{Additional Rules and Remarks}
\begin{enumerate}[nosep]
	\item \textbf{Safety First:} This test takes place in a public area. Therefore, referees will be more careful and will not tolerate even slight collisions.
	
	\item \textbf{Fair Play:} Upon arrival to the restaurant, only two team members are allowed next to the robot. Tweaking, coding, debugging, or mapping the area will lead to immediate disqualification.
	
	\item \textbf{Power outlets, WiFi and ECRA:} The availability of wireless, external computing devices, or electrical outlets can't be guaranteed. Assume unavailability.
	
	\item \textbf{Disturbances from Outside:} If a person from the audience (severely) interferes with the robot in a way that makes it impossible to solve the task, the teams may repeat the test immediately.
	
	\item \textbf{Partial Scoring:} The main task allows partial (per order) scoring.
	
	\item \textbf{Deus ex Machina:} Score reduction applies per order as follows.
	\begin{itemize}[nosep]
		\item\textbf{Asking for Directions:} Receiving directions, verbally or by pointing, reduces the score by 25pts.
		\item\textbf{Being Guided:} Being guided to a table causes a score reduction of 200pts.
		\item\textbf{Wrong Orders:} Delivering a wrong object reduces score by 350pts if at least part of the order is correct. Reduction of 500pts otherwise.
		\item\textbf{Handing Orders:} Handing objects to the customer instead of placing them causes a score reduction of 100pts.
	\end{itemize}


\end{enumerate}


\subsection*{Referee Instructions}
The referee needs to
\begin{itemize}
	\item Prepare orders for each customer.
\end{itemize}


\subsection*{OC Instructions}
During Setup days:
\begin{itemize}[nosep]
	\item Check with local (security) management if the possible location, including a sufficient queuing area, can be used for the restaurant test.
\end{itemize}
1 hour before the test:
\begin{itemize}[nosep]
	\item Gather all teams and robots to move to some nearby queuing area and instruct the teams how/when to move to the actual test location.
\end{itemize}


\subsection*{Score Sheet}
The maximum time for this test is 15 minutes.

\small\begin{scorelist}
	\scoreheading{Main Goal}
	\scoreitem[2]{500}{Complete an order}

	\scoreheading{Bonus rewards}
	\scoreitem{75}{Detect calling or waving guest}
	\scoreitem{75}{Arrive at table of calling or waving guest without guidance}
\end{scorelist}

% Local Variables:
% TeX-master: "Rulebook"
% End:


% Local Variables:
% TeX-master: "Rulebook"
% End:



The maximum time for this test is 10 minutes.

\small\begin{scorelist}

	\scoreheading{Attending request}
	\scoreitem{20}{Reach patient after being called}
	\scoreitem{10}{Await command to get pills}

	\scoreheading{Describing pills}
	\scoreitem{70}{Real time description (given upon arrival)}
	\scoreitem{40}{Description given within $t \leq 5$ seconds}
	\scoreitem{20}{Description given within $5 < t \leq 15$ seconds}
	\scoreitem{10}{Description given within $15 < t \leq 30$ seconds}
	\scoreitem{00}{Description given within $ t \geq 30$ seconds}

	\scoreheading{Picking pills}
	\scoreitem{40}{Choose the correct pills}
	\scoreitem{20}{Grasp the correct pills}
	\scoreitem{5}{Grasp wrong pills}

	\scoreheading{Pills handover}
	\scoreitem{20}{Natural delivery (no instructions are given to operator)}
	\scoreitem{10}{Assisted delivery (operator instructs robot for delivery)}

	\scoreheading{Activity recognition}
	\scoreitem{50}{Granny trying to reach drop blanket}
	\scoreitem{50}{Falling Granny}
	\scoreitem{50}{Granny stands up and walk away + sit}

	\scoreheading{Response to activity}
	\scoreitem{40}{Pickup the blanket + give the blanket}
	\scoreitem{40}{Grasp phone + give phone}
	\scoreitem{40}{Take walking stick / cane}

	\scoreheading{Bonuses (up to 180 points)}
	\scoreitem{10}{Using Smart House to call instead of grasping \& giving the phone}
	\scoreitem{40}{Opening the pill bottle (with a screw cap)}
	\scoreitem{100}{Picking a single pill from the bottle}
	
	\setTotalScore{250}
\end{scorelist}



% Local Variables:
% TeX-master: "Rulebook"
% End:


\section{Wake me up test}

The robot's owner has overslept. Knowing the schedule of the owner and noticing it is getting late, the robot helps it's owner to wake up and start the day.

The robot has to help a human in a daily morning task. The task involves interact with a smart house, awake a dormant human, take an order, prepare the breakfast and deliver it to the human.

\subsection{Focus}

This test focuses on advanced object manipulation, human pose detection, object recognition and and manipulation; as well as object recognition.


\subsection{Task}

\begin{enumerate}

	\item \textbf{Awakening the owner:} The robot enters the bedroom, approaches to the bed, and starts to awaken the owner (operator lying on the bed) for one minute by playing an alarm-like sound or using it's own voice. Within one minute starting from the first call, the owner will wake up in a natural way (sit on the bed and rub face; sit on bed, rise arms and yawn; stand up etc.), then the robot must announce it has successfully detected the awakening by greeting it's owner.
	\begin{itemize}
		\item \textbf{Turning-on bedroom's lights [Smart-house option]:} After entering to the bedroom, the robot can send a command to the house to turn on the bedroom lights.
		\item \textbf{No annoying sounds:} Alarm-like sounds must be short and clean (no continuous music is allowed), and voice calls must be short and clear. A silence gap of 10 seconds between calls is advised.
		\item \textbf{Show must go on:} One minute after the first call, the owner is awake, so the robot must proceed to the next point.
	\end{itemize}

	\item \textbf{Delivering the newspaper [Optional]:} After awakening it's owner, the robot approaches to her and delivers a newspaper into the owner's hand (the owner will face the robot after being awakened and extend her hand to it). Robot must release the newspaper only after the human has grasped it.

	\item \textbf{Taking breakfast order:} The robot asks to it's owner for a breakfast of her preference. The order will include: one random fruit/snack, one kind of cereal, and one kind of milk (stating no milk means whole milk), but those can be given in any order. The robot may ask for a confirmation of the order up to three times. If the robot is not able to handle a tray (see below), it must state that breakfast will be delivered to the dining room. Examples of the order are:

	\begin{itemize}
	\item Froot-loops with banana and light milk.
	\item Flakes with lactose-free milk and a peach.
	\item Apple and choco-flakes (i.e.~one apple, and choco-flakes with whole milk).
	\end{itemize}

	\item \textbf{Opening kitchen's door [Optional]:} The kitchen's door is closed. Upon arrival, robot may try to open the door. The robot may also give up and request for the door to be opened by a referee. If the robot succeed on opening the door, the jury may add up to 5 minutes to the time for completing the test.

	\item \textbf{Turning-on kitchen light [Smart-house option]:} After entering to the kitchen, the robot can send a command to the house to turn on the kitchen lights and the coffee brewer. Kitchen lights must be turned on every time the robot enters the kitchen.

	\item \textbf{Serving the breakfast:} Once in the kitchen, the robot must locate the tray and place into it the requested fruit/snack, a box of the requested type of milk, and a bowl; then pour the requested type of cereal into the bowl. If the robot is not capable of handling a tray, it may serve the breakfast directly at the diner table. The placement order is not relevant, nor is the serving order. An example is provided below:
	\begin{itemize}
		\item \textbf{Find tray:} The robot locates the tray in the table and pulls it to make easier placing objects.
		\item \textbf{Place milk:} The robot locates the requested type of milk (whole) among many (whole, light, lactose-free) in the shelf, and places it on the tray at the top-right corner.
		\item \textbf{Place fruit:} The robot locates the requested type of fruit (apple) among many (apple, banana, apricot) in the table, and places it on the tray at the top-left corner.
		\item \textbf{Place bowl:} The robot locates the bowl in the table and places it on the tray at the middle.
		\item \textbf{Pour cereal:} The robot locates the requested type of cereal (flakes) among many (choco-flakes, flakes, froot-loops) in the shelf, grasps the box, pours the cereal it into the bowl and puts the cereal box back into the shelve.
	\end{itemize}

	\item \textbf{Placing the spoon [Optional]:} After placing the cereal bowl on the tray or dining room table, the robot may place a spoon close to it.
	Delivering the tray: After placing objects into the tray, the robot must take the tray and deliver it to the human in the bedroom, leaving it on a table or directly to the owner's hands.

	\item \textbf{Turning-off kitchen light [Smart-house option]:} After leaving to the kitchen, the robot can send a command to the house to turn off the kitchen lights. Kitchen lights must be turned off every time the robot leaves the kitchen.

	\item \textbf{Doing the bed [Optional]:} After the breakfast has been delivered, the robot may proceed to do the owner's bed. Points are awarded based on a \quotes{Professional Mom} criteria.

\end{enumerate}

\subsection{Additional rules and remarks}

\begin{itemize}
	\item \textbf{Bowl and tray:} Both, the bowl and tray are taken from the official containers list (see \refsec{rule:scenario_objects}) and known beforehand. The bowl will be used to pour cereal inside (only cereal, not milk) and the tray to transport the bowl, milk and fruit. Containers will be placed on a flat surface for convenience.

	\item \textbf{Breakfast objects:} The milk and cereal boxes will be taken from the known objects list, and the fruit from the alike objects list (see \refsec{rule:scenario_objects}). There will be more than one milk, cereal and fruit, so the robot will need to pick the proper one. Objects will be placed close in order to minimize the required navigation time for the robot.

	\item \textbf{Collaborative test:} The team leader may request help from a second team to perform the \quotes{serving the breakfast}, \quotes{delivering the tray} tasks, and \quotes{smart-house} optionals. All score achieved by both robots is given to the main team, but also the points scored by the helping-robot are given to the helping team as a bonus. This cooperation must be informed to the TC at least two hours before the competition.

	\item \textbf{Fruit or snack?:} If the robot is not able to properly handle fruits (alike objects), it can be replaced by easier-to-manipulate objects from the known objects list. Team leader must contact a TC member to request using snacks (known objects) instead of fruits (alike objects, see \refsec{rule:scenario_objects}).

	\item \textbf{Newspaper:} The Newspaper is provided to the Team Leader by the OC before the robot enters the arena holding the newspaper. The Team Leader must bring the Newspaper back to the OC before the end of the test.

	\item \textbf{Optional tasks:} The test includes optional tasks (such as deliver the newspaper, placing the spoon, and doing bed) which are not required to be performed as part of the overall test but brings an additional scoring for solving it. Team leader must contact a TC member to request optional tasks to be available.

	\item \textbf{Pouring the cereal:} In case the robot is not able to pour inside the bowl, it may just handle cereal box, either into the tray or to the table.

	\item \textbf{Smart-house:} The arena-house may have enabled official smart-house devices (Section \refsec{rule:smarthomedevices}), there are additional scoring for interacting with the house.

	\item \textbf{Two robots:} This is a very challenging test and serving the breakfast is complex and time-consuming task. If a team has more than one robot, up to two robots may collaborate in the \quotes{serving the breakfast} and \quotes{delivering the tray} tasks. Team leader must contact a TC member to inform there will be two robots in the arena.
\end{itemize}

\subsection{Referee instructions}

The referee needs to
\begin{itemize}
	\item Give a wake-up signal to the operator within a minute, starting when the robot begins to try to awake the operator.
	\item Generate and provide a random breakfast order for the operator
	\item Type the breakfast order in a qualified typing device when required (Continue rule, Section \refsec{rule:asrcontinue}).
	\item Stop the robot immediately when tray is about to fall
\end{itemize}

\subsection{OC instructions}

\textbf{2 hours before the test}
\begin{itemize}
	\item Announce the placement of the objects (Cereal, milk, and fruit).
	\item Announce the placement of the containers (tray and bowl).
	\item Announce the default breakfast objects.
\end{itemize}

\textbf{During the test}
\begin{itemize}
	\item Provide teams with the newspaper
	\item Place tray and breakfast objects into the kitchen
	\item Place spoon when needed
\end{itemize}

\subsection{Score sheet}
The maximum time for this test is 10 minutes.

\begin{tabularx}{\textwidth}{ X r }
	\textbf{Action} & \textbf{Score} \\ \hline
	\textbi{Awakening the human}  \\
	Detect the human awakening & 2.0 \\
	\\
	\textbi{Taking the order} \\
	Understanding whole order & 2.0 \\
	Understanding whole order on console (typed) & 0.5 \\
	Robot's own suggestion for breakfast & 0.0 \\
	\\
	\textbi{Serving breakfast} \\
	Placing the bowl & 2.0 \\
	Placing the milk bottle (½ score on wrong milk type) & 1.0 \\
	Placing the fruit/snack (½ score if using snack instead of fruit, ½ score on wrong object type) & 2.0 \\
	Pouring cereal into the bowl (½ score on wrong cereal type) & 3.0 \\
	Spilling cereal outside the bowl & -1.0 \\
	Spilling much cereal outside the bowl & -2.0 \\
	\\
	\textbi{Delivering breakfast} \\
	Grasping the tray (and successfully lifting it up to at least 5 cm for more than 10 second) & 3.0 \\
	Safely transporting the tray (no object inside flipped or fell during transport) & 1.0 \\
	Placing the tray (safely and the tray stands still for more than 10 second) & 2.0 \\
	Handing-over the tray to the operator's hands & 4.0 \\
	Complete the task with complete and correct order & 2.0 \\
	\\
	\textbi{Smart-House optionals} \\
	Turning on bedroom lights on enter & 1.0 \\
	Turning on kitchen lights and coffee brewer on enter & 1.0 \\
	Turning off kitchen lights on leave & 1.0 \\
	\\
	\textbi{Optional tasks (up to 20 points)} \\
	Handing-over the newspaper & 2.0 \\
	Opening kitchen's door & 5.0 \\% & \textit{+5 minutes} \\
	Placing the spoon & 3.0 \\
	Doing bed & 10.0 \\ \hline
	\textbf{Total score} (excluding optional tasks, penalties, and bonuses) & 25.0 \\
\end{tabularx}


\section{Finals}

\end{document}