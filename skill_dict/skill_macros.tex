%\documentclass{article}
\usepackage{etoolbox}
\usepackage{xparse}

\newcounter{skillcounter}
\newcommand{\skills}{}
\newcommand{\skillcategories}{}


% Usage:
% \DefSkill{SkillId}{Points}{Default Description}{Category}
% Define a skill
\newcommand{\DefSkill}[4]{%
  \stepcounter{skillcounter}%
  \expandafter\edef\csname skillnum#1\endcsname{\arabic{skillcounter}}
  \csdef{skillname#1}{#1}
  \csdef{skillpoints#1}{#2}
  \csdef{skilldesc#1}{#3}
  \csdef{skillcat#1}{#4}
  \listadd{\skills}{#1}
  \listcsadd{skillcatlist#4}{#1}
  % Add category to the list if not already present
  \ifinlist{#4}{\skillcategories}{}{\listgadd{\skillcategories}{#4}}
}

\newtoggle{showcategory}

% \SkillTarget is similar to \Skill, but it is used during the definition and a label is added
% This is a helper and you usually don't need to sue it
\newcommand{\SkillTarget}[1]{%
  \ifcsdef{skillname#1}{%
    \renewcommand{\labelitemi}{} % Hide bullet points
    \item \begin{tabularx}{\linewidth}{@{}lXr@{}}
      \label{skill:#1}
      \csuse{skillnum#1}. & \textbf{\csuse{skillname#1}} & \textbf{\csuse{skillpoints#1}} \\ % Points: \\
      & Description: \csuse{skilldesc#1} & 
      \iftoggle{showcategory}{\\ & Category: \csuse{skillcat#1} &}{} 
    \end{tabularx}
  }{%
    \item \textbf{Skill not found: #1}
  }
}


% Render the dictionary
\newcommand{\RenderDictionary}{%
  \toggletrue{showcategory}
  \section*{Skill Dictionary}
  \begin{itemize}
    \forlistloop{\SkillTarget}{\skills}
  \end{itemize}
  \togglefalse{showcategory}
}

\newcommand{\RenderCategory}[1]{%
  \subsection*{#1}
  \begin{itemize}
    \forlistcsloop{\SkillTarget}{skillcatlist#1}
  \end{itemize}
}

% Render all categories
\newcommand{\RenderCategoryDictionary}{%
  \renewcommand*{\do}[1]{\RenderCategory{##1}}
  \dolistloop{\skillcategories}
}

% \Skill is a reference to a defined skill. It will link to the Skill Dictionary

% Example usage:
% Pick up Dish Item (Selection Group)
%\begin{Group}{PickupDishItem}{Pick up various dish items from the table}{30}{Selection}
%    \Skill{PickTinyKnown}[Pick up a Spoon, Fork or a Knife]
%\end{Group}

% \Skill{SkillID}[Custom Description]
\NewDocumentCommand{\Skill}{m o}{%
  \ifcsdef{skillname#1}{%
    \item \begin{tabularx}{\linewidth}{@{}p{1cm}Xr@{}}
      \hyperref[skill:#1]{\csuse{skillnum#1}} & % ID
      \IfNoValueTF{#2}{%
        \csuse{skilldesc#1} % Description
      }{%
        #2 % Custom Description
      } & 
      \csuse{skillpoints#1} % Points
    \end{tabularx}
  }{%
    \item \textbf{\textcolor{red}{Error: Skill with ID '#1' not found.}} \\
    \textcolor{red}{Please define the skill using \textbackslash DefSkill\{skillId\}\{Points\}\{Default Description\}\{Category\} or check for typos.} \\
    \textcolor{red}{You can use the defined skill with \textbackslash Skill\{SkillId\}[Sheet specific description] within a Group environment}
  }
}
