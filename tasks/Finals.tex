\chapter{Finals}\label{test:final}

The competition ends with the Finals on the last day, where the top teams with the highest total score compete.

\section{Scoring}
The final score and ranking are determined by the final scoring and the previous performance of the team, in the following manner:

\begin{enumerate}
  \item The influence of this tests ranking is \SI{50}{\percent}.
  \item The influence of the total sum of points scored by the team is \SI{50}{\percent}.
\end{enumerate}

There is no maximum score during the Final.

These demonstrations are carried out in a serialized fashion in one \Arena{}.

\section{Task}
The robot is asked to maintain the household by cleaning up the arena and assisting people.

\noindent \textbf{Main Goal:} Solve different problems in the arena.\\

% %% %%%%%%%%%%%%%%%%%%%%%%%%%%%%%%%%%%%%%%%%%%%%%%%%%%%%%%
% Setup
% %% %%%%%%%%%%%%%%%%%%%%%%%%%%%%%%%%%%%%%%%%%%%%%%%%%%%%%%
\subsection*{Setup}
The arena is in its default state apart from problems set up for the robot to solve:
\begin{itemize}
  \item \textbf{Trash:} Objects on the floor are to be thrown in the trash.
  \item \textbf{Objects:} Objects that are not in their default location should be returned to their default location (see~\ref{rule:scenario_objects}). 
  \item \textbf{Persons:} Some persons in the arena will have requests for the robot. They will raise their hand if the robot is in the same room.
  \item \textbf{Closing Furniture:} Doors of the Cabinet as well as the Dishwasher need to be closed.
  \item \textbf{Welcome Guest:} There is an additional person waiting behind the exit door. The person will state their request after being welcomed by the robot. The door must be opened without human assistance. As the position is known, there will be no points awarded for finding this person.
  \item \textbf{Custom Tasks:} Additional reasonable household task.
\end{itemize}


\subsubsection{Custom Tasks}
\begin{enumerate}[nosep]
  \item \textbf{Requesting Additional Tasks:} Additional reasonable household tasks may be requested for scoring during the team leader meetings. These tasks must be approved by the \TC{}.
  \item \textbf{Arena-Specific Tasks:} Depending on the arena setup, certain household chores may be delegated to the robot for completion. Examples of tasks that may be handled by the robot include, but are not limited to, window cleaning and picture alignment on the wall.
  \item \textbf{Team-Supplied Items} In the event that additional items are necessary for the completion of a task (e.g., clothing for folding, watering cans), the requesting team is responsible for supplying these items.
  \begin{enumerate}[nosep]
    \item \textbf{Standardization} These items must be regular household items (no markers, no custom printed handles etc.).
    \item \textbf{Availability} All requested items must be provided and made available to the competitors prior to the day of the final. At least two identical copies of each item must be supplied to ensure adequate access.
  \end{enumerate}
\end{enumerate}

% %% %%%%%%%%%%%%%%%%%%%%%%%%%%%%%%%%%%%%%%%%%%%%%%%%%%%%%%
% Procedure
% %% %%%%%%%%%%%%%%%%%%%%%%%%%%%%%%%%%%%%%%%%%%%%%%%%%%%%%%
\subsection*{Procedure}
\begin{enumerate}[nosep]
	\item \textbf{Test start:} The robot enters when the arena door is open.
	\item \textbf{Finding Problems:} The robot has to find problems to solve on its own.
\end{enumerate}

\subsection*{Additional rules and remarks}
\begin{enumerate}[nosep]
	\item \textbf{Number of Problems:} The number of problems depends on the arena size the minimum count of generated problems is 8.
	\item \textbf{Repeating Problem Category:} Solving the same Category of Problem incurs a penalty. 
	\item \textbf{Solving more:} You can continue solving problems to compensate for penalties.
	\item \textbf{Partial Scoring:} The main task allows partial scoring (per \emph{solved} problem).
	\begin{enumerate}[nosep]
		\item \textbf{Scores:} Score reduction is applied as a percentage depending on the solution.
		\item \textbf{Penalties:} The Repetition penalty is applied before any partial penalties
		\item \textbf{Example:} If picking up trash off the floor is valued as 60\% of the solution, then requesting a handover should be a $ 650 \times 0.6 = 390 $ points penalty for the first pick and  $ \left( 650 - 300 \right) \times 0.6 = 210 $ for the second pick.
	\end{enumerate}
	\item \textbf{Command Generator:} Problems and commands will be generated using the official command generator\footnote{\url{https://github.com/RoboCupAtHome/CommandGenerator}}. Commands are generated as described in~\ref{test:gpsr}.
	\item \textbf{Finding People:} Finding a person and stating they need help counts as finding the problem.
	\item \textbf{Understanding Commands:} The robot must correctly interpret and repeat commands given by people. Correctly repeating the command given by a person counts as partially solving the problem. Commands are issued as described in~\ref{test:gpsr}.
\end{enumerate}


\subsubsection{The Show Must Go On}
To ensure a good experience for the audience the teams are allowed to restart the robot inside the arena.
\begin{enumerate}[nosep]
  \item All previously collected points will be kept.
  \item The tasks may be rearranged by the Referee during a restart.
  \item The restart penalty is only applied if the robot continues scoring afterwards.
\end{enumerate}

\subsection{Commentator}
The team will be asked to provide a commentator to explain the robot's behavior and answer questions to produce a better viewing experience for the audience.


\subsection*{Score sheet}
\begin{scorelist}[timelimit=10,attempts=1,outstanding=False,continue=False, specialpenbonus=False]
	\scoreheading{Main Goal (can be repeated unlimited times)}
	\scoreitem[3]{150}{Find and clearly state an encountered problem}
		\scorepen{100}{Find repeated problem category}
	\scoreitem[10]{650}{Solve a problem}
		\scorepen{300}{Solving repeated problem category for the 2nd time}
		\scorepen{500}{Solving repeated problem category for the 3rd (or more) time}
		\scorepen{650}{%
			Instructing a human to perform parts  \\
			\hspace{3em}of the task will apply a percentage penalty according to \\
			\hspace{3em}similar scores in other tests.
		}
	\scoreitem{0}{%
		\hspace{1em}Solving only parts of a GPSR command will apply \\
		\hspace{1em}a percentage penalty according to similar \\
		\hspace{1em}scores in other tests.
	}
	\scoreitem[1]{600}{Opening the Door of the Apartment}
	\scoreitem[1]{600}{Closing the Dishwasher}
	\scoreitem[1]{0}{Custom Task}
	\scoreitem[1]{0}{Custom Task}
	\scoreitem[1]{0}{Custom Task}

	\scoreheading{Penalties}
	\penaltyitem[1]{100}{Find repeated problem category}
	\penaltyitem[1]{300}{Solving repeated problem category for the 2nd time}
	\penaltyitem[1]{500}{Solving repeated problem category for the 3rd (or more) time}
	\penaltyitem[3]{150}{Asking for location of a problem}
	\penaltyitem[1]{50}{Restart (only applies if the robot continues scoring afterwards)}

\end{scorelist}

\ifShortScoresheet{}{
	\textbf{Referee Information:}
	Note each problem (category, item, location) and mark if they are stated by the robot in remarks
}%




%% %%%%%%%%%%%%%%%%%%%%%%%%
\section{Final Ranking and Winner}

The winner of the competition is the team that gets the highest ranking in the \iterm{Finals}.
The second place will be the team that got the second-highest ranking in the \iterm{Finals}.
The third place will be the team with the third-highest score in the \iterm{Finals}.
And so on.


% Local Variables:
% TeX-master: "Rulebook"
% End:
