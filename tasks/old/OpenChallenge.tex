\newcommand{\bonusRobotCoop}{50~}

\section{Open Challenge}
\label{sec:test_open_challenge}

During the Open Challenge teams are encouraged to demonstrate recent research results and the best of the robots' abilities. It focuses on the demonstration of new approaches/applications, human-robot interaction and scientific value.

\subsection{Task}

The Open Challenge consists of a demonstration and an interview part.
It is an open demonstration which means that the teams may demonstrate anything they like.
The performance of the teams is evaluated by a jury consisting of all team leaders, TC and EC.
\OpenDemonstrationTask{seven}{three}

\subsection{Presentation}
During the demonstration, the team can present the addressed problem and the demonstrated approach.
\begin{itemize}
	\item A video projector or screen, if available, may be used to present a brief (max. 2 minute) presentation relevant to the demonstration.
	\item Teams may omit the video, use a more brief video, or have the robot act over the video in order to make more time for the robot demo.
	\item There may be no human presenter. This is intended to be a demonstration of the robot's capabilities and not a research talk. The robot may present for itself (e.g., describing what it is doing or providing a narrative for the presentation on its own).
	\item Humans may interact with the robot during the interaction, but are not to act as presenters. This judgement is left to the jury.
	\item The team can also visualize robot's internals, e.g., percepts.
\end{itemize}

It is important to note that the jury may decide to end the demonstration if there is nothing happening or nothing \emph{new} is happening.

\OpenDemonstrationChanges

\subsection{Jury evaluation}
\begin{enumerate}
	\item \textbf{Jury of team leaders:} All teams have to provide \emph{one} person
	(preferably the team-leader) to follow and evaluate the entire Open Challenge.
	\item \textbf{Evaluation:} Both the demonstration of the robot(s), and the answers of the team in the interview part are evaluated.\\
	For each of the following \emph{evaluation criteria}, each jury member submits a score from $0-100$:
	\begin{enumerate}
	\item Novelty and (scientific) contribution
	\item Difficulty level of the demonstrated task
	\item Success of the demonstration
	\item Overall (demo was convincing, fluent, interesting, etc.)
	\end{enumerate}
	A jury member is not allowed to evaluate and give points for the own team.
	\item \textbf{Normalization and outliers}:
	\begin{enumerate}
		\item The points given by each jury member are scaled to obtain a score from $0.0-1.0$.
		\item The normalized total score for each team is the mean of the jury member scores.
			To neglect outliers, the $N$ best and worst scores are left out:
			$$\mbox{score}_{norm} = \frac{\sum\mbox{team-leader-score}}{\mbox{number-of-teams} - (2N+1)}\times\frac{1}{100},
			\quad N=\begin{cases}2, & \mbox{number-of-teams} \ge 10\\1, & \mbox{number-of-teams} < 10 \end{cases}$$
		\end{enumerate}
		\item The final Open Challenge score for each team is computed at the end of regular tasks. The Open Challenge \scoring{final score} is the product of the normalized score multipled by the highest score achieved in the competition:
		$$\mbox{score} = \mbox{score}_{norm} \times \frac{min\Big(250, max\big(\{S_2\}\big)\Big)}{250},
		\quad \{S_2\}=\mbox{All scores}
		$$
\end{enumerate}

\subsection{Additional rules and remarks}
\begin{enumerate}
	\item \textbf{Start signal:} There is no standard start-signal for this test.
	\item \textbf{Abort on request:} At any time during the demonstration, the jury may interrupt and abort the demonstration:
	\begin{enumerate}
		\item if nothing is shown: in case of longer delays (more than one minute), e.g., when the robot does not start or when it got stuck;
		\item if nothing new is shown: the demonstrated abilities were already shown in previous tests (to avoid dull demonstrations and push teams to present novel ideas).
	\end{enumerate}
\end{enumerate}

\subsubsection{Team-team-interaction:}
\label{rule:OC-team-team-interaction}
An extra bonus of up to \bonusRobotCoop points can be earned if robots from two teams (4 robots maximum, 2 from each team) successfully collaborate (robot-robot interaction).
\begin{enumerate}
	\item This bonus is earned for both teams.
	\item The robot(s) of the other team must only play a minor role in the total demonstration.
	\item It must be made clear that the demonstrations from the two teams are not similar, otherwise the points cannot be awarded.
	\item In case a team receives two (or more) bonuses, the maximum bonus will be taken.
	\item The collaboration is possible even if one of the two teams has not reached Stage 2.
	\item A team not participating in Stage 2 receives no bonus points for this test.
\end{enumerate}

% \paragraph*{Inter-league collaboration}:
% \label{rule:OC-inter-league-collaboration}
% Inter-league collaboration must be announced to the OC at least one day before the test. Teams participating in multiple @Home Leagues does receive no bonus for cooperation. Standard Platform robots are allowed to take part in the Open Challenge of the Open Platform League, but Open Platform robots can \emph{not} participate in any Standard Platform League's test.

% For sake of clarity, please consider the following example: Let be A, B two teams participating in RoboCup @Home where
% \begin{itemize}
% 	\item Team A participates in SSPL.
% 	\item Team B participates in both SSPL and OPL.
% 	\item Team A and B have qualified into Stage II.
% \end{itemize}
% Then, by applying the \textit{Inter-league collaboration Rule} (See~\refsec{rule:OC-inter-league-collaboration}) the following statements can be concluded:
% \begin{itemize}
% 	\item B OPL can not participate in A SSPL's open challenge.
% 	\item B OPL can not participate in B SSPL's open challenge.
% 	\item A SSPL can participate in B OPL's open challenge. Team A and B get a bonus because A <> B.
% 	\item B SSPL can participate in B OPL's open challenge. There is no bonus because B = B.
% \end{itemize}




% Local Variables:
% TeX-master: "Rulebook"
% End:
