\section{Storing Groceries [House Holder]}
The robot stores groceries into the shelf next to the objects of the same kind that are already there (e.g. for instance by placing apples near pears and bananas).

% \subsection{Focus}
% This test focuses on the detection and recognition of objects and their features, as well as object manipulation.

\subsection{Main Goal}
The robot has to move 5 out of 10 objects from a nearby table into the shelf. Placed objects must be placed next to objects of the same Category or grouped by similarity.

\noindent\textbf{Reward:} 500pts\\

\noindent\textbf{HINT:} The robot can ask the referee where to place the carried object (relative positions or pointing are both allowed).

\subsection{Bonus rewards}
\begin{enumerate}[nosep]
	\item Opening the shelf door (300pts)
	\item Moving a \emph{tiny} object (100pts)
	\item Moving a \emph{heavy} object (100pts)
\end{enumerate}

% %% %%%%%%%%%%%%%%%%%%%%%%%%%%%%%%%%%%%%%%%%%%%%%%%%%%%%%%
%
% Setup
%
% %% %%%%%%%%%%%%%%%%%%%%%%%%%%%%%%%%%%%%%%%%%%%%%%%%%%%%%%
\subsection{Setup}
\begin{enumerate}
	\item \textbf{Location:} The testing area has a shelf and a table.
	The distance between the Table and the Shelf cannot exceed 2 meters.

	\item \textbf{Shelf:} The shelf contains at least 10 objects arranged in groups of 2 or more, either by category or likeliness.
	The shelf has at least one free space for starting a new set.

	\item \textbf{Shelf door:} The shelf door is open by default.
	The team leader can, request the door to be closed and score additional points for opening it. If the robot fails to open the door, it must clearly state it and request the referee to open it.

	\item \textbf{Objects:} Some of the objects are placed behind the door and cannot be accessed unless the door is open.

	\item \textbf{Table:} The table can have up to 10 objects, but never less than 5.
	In small tables, objects will be added as the robot frees up space.
\end{enumerate}


% %% %%%%%%%%%%%%%%%%%%%%%%%%%%%%%%%%%%%%%%%%%%%%%%%%%%%%%%
%
% Additional Rules
%
% %% %%%%%%%%%%%%%%%%%%%%%%%%%%%%%%%%%%%%%%%%%%%%%%%%%%%%%%
\subsection{Additional rules and remarks}
\begin{enumerate}
	\item \textbf{Clear area:} The robot may assume that the working area is clear, (i.e. can move slightly backwards for its task).

	\item \textbf{Object distribution:} The 10 objects to be moved are evenly distributed in random fashion including
	4 known objects,
	4 alike objects, and
	2 unknown objects.
	Among these, the robot will always find
	a heavy object,
	a tiny object, and 
	an amorphous object.

	\item \textbf{Table} The table's rough location will be announced beforehand, having its position either left, right, or behind the robot.
\end{enumerate}

\newpage
\subsection{OC instructions}

\textbf{2 hours before the test}
\begin{itemize}
	\item Announce which table will be used in the test.
	\item Announce a rough location for the table.
\end{itemize}

\subsection{Referee instructions}
The referee needs to
\begin{itemize}
	\item Place the objects in the shelf, grouping them by likeliness.
	\item Open the door of the shelf.
	\item Place objects on the table.
\end{itemize}


% \newpage
% \subsection{Score sheet}
% 
The maximum time for this test is 3 minutes.
The robot is given 1 extra minute to open the cupboard door. 
If the robot is not able to open the door within that minute, it will be opened by the referee. 
In case the robot opens the door within the minute, the robot has a small time advantage. 

\begin{scorelist}
% There are 5 filled shelves, originally with 2 objects, 1 in each corner.
% The table also has 10 objects, that the robot should move to the shelf.
% So 20 objects in total
% This can be a tight fit, as there will be potentially 4 objects per shelf, as the robots moves them from the table one by one

% The robots are not fast enough though to do more than 5 objects in the given time.

% Grasp (any object): 10
% Place (anywhere in the cupboard): 10
% Place in correct place: 15
% Recognize known object correctly (without grasping/placing something of that class): 10
% Label two unknown objects of the same class with the same label (e.g. ``class0''): 15

% Place known object near known object of same class: 40
% Place unknown object near unknown object of the same class: 50


	\scoreheading{Grasping objects}
	\scoreitem[5]{10}{For each successful grasp of any object (lifting it up to at least 5 cm for more than 10 seconds)}

	\scoreheading{Placing objects}
	\scoreitem[5]{10}{For each successful placement of an object anywhere in the cupboard (safely stands still for more than 10 seconds)}
	\scoreitem[5]{5}{For each successful placement of an object at correct place (near an object of the same class)}

	\scoreheading{Recognizing objects}
	\scoreitem[10]{5}{Every correctly recognized known or alike object in the report file}
	% TODO: Split up scores over these 3 variants of a correct label:
	% 1: As unknown, instead of wrongly applying a label from the known or alike objects (e.g. use an Open World assumption in your classifier)
	% 2: Label all instances of the same unknwon class with the same generated label, e.g. label0, so distinguish between different onknown objects
	% 3: Label the unknown objects with a meaningful label, eg. cookies in case its a sort of cookies. I.e. use a classifier that knows many classes. 
	\scoreitem[5]{15}{Correctly label unknown objects}
	\scoreitem[10]{-5}{False positive label}
	
% 	\scoreheading{Total task}
% 	\scoreitem[5]{40}{Place known object near known object of same class}
% 	\scoreitem[5]{50}{Place unknown object near unknown object of same class}

	\scoreheading{Bonus}
	\scoreitem[1]{20}{Open the door without human help}
	
	\setTotalScore{250}
\end{scorelist}


% Local Variables:
% TeX-master: "Rulebook"
% End:


% Local Variables:
% TeX-master: "Rulebook"
% End:
