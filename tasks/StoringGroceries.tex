\section{Storing Groceries}\label{test:storing-groceries}
The robot stores groceries into a cabinet with shelves. Objects are sorted on the shelves based on similarity, for instance an apple is stored next to other fruits.\\

\noindent \textbf{Main goal:} Move objects from a table to the cabinet, grouping them by category or similarity. Refilling the cereal container.\\

\noindent \textbf{Optional goals:}
\begin{enumerate}[nosep]
	\item Opening the cabinet doors
	\item Moving a \emph{tiny} object
	\item Moving a \emph{heavy} object
	\item Picking Objects from the shopping bag
\end{enumerate}

\subsection*{Focus}
\emph{Object detection and recognition}, \emph{object feature recognition}, \emph{object manipulation}.

% %% %%%%%%%%%%%%%%%%%%%%%%%%%%%%%%%%%%%%%%%%%%%%%%%%%%%%%%
% Setup
% %% %%%%%%%%%%%%%%%%%%%%%%%%%%%%%%%%%%%%%%%%%%%%%%%%%%%%%%
\subsection*{Setup}
\begin{itemize}
	\item \textbf{Locations:}
		\begin{itemize}
			\item \textbf{Start location:} Before the test, the robot waits outside the \Arena{} and navigates to the testing area when the door is open.
			\item \textbf{Test location:} The testing area has a cabinet and a table nearby.
		\end{itemize}
	\item \textbf{People:}
		\begin{itemize}
			\item No people are involved in the test, unless the robot requires human assistance.
		\end{itemize}
	\item \textbf{Furniture:}
		\begin{itemize}
			\item \textbf{Table:} The table has 5--10 objects placed on it and the robot can choose which ones to grasp and in what order. On small tables, objects will be added as the robot frees up space.
			\item \textbf{Shopping Bag:} There is a shopping bag with 5--10 additional normal items on the ground next to the table.
			\item \textbf{Cabinet:} The cabinet contains objects arranged in groups --- either by category or likeliness --- on different shelves.
			\item \textbf{Cabinet door:} The cabinet is closed. If the robot fails to open the door, it must clearly state this and request the referee to open it.
		\end{itemize}
	\item \textbf{Objects}:
		\begin{itemize}
			\item \textbf{Table objects:} The object on the table are arranged arbitrarily.
			\item \textbf{Cabinet objects:} The objects are placed behind the cabinet doors and cannot be accessed unless the doors are open.
			\item \textbf{Containers:} The container for the cornflakes are placed on the table.
		\end{itemize}
\end{itemize}


% %% %%%%%%%%%%%%%%%%%%%%%%%%%%%%%%%%%%%%%%%%%%%%%%%%%%%%%%
% Procedure
% %% %%%%%%%%%%%%%%%%%%%%%%%%%%%%%%%%%%%%%%%%%%%%%%%%%%%%%%
\subsection*{Procedure}
\begin{enumerate}[nosep]
	\item \textbf{Table location:} At least two hours before the test, the referees announce the table and cabinet that will be used in the test, as well as a rough location of the table.
	\item \textbf{Test start:} The robot moves to the testing area when the arena door is open.
	\item \textbf{Storing groceries:} After identifying the table, the robot moves the objects from the table to the cabinet.
	\item \textbf{Pouring cereal:} The robot should pour cereal into the designated open container.
\end{enumerate}


% %% %%%%%%%%%%%%%%%%%%%%%%%%%%%%%%%%%%%%%%%%%%%%%%%%%%%%%%
% Additional Rules
% %% %%%%%%%%%%%%%%%%%%%%%%%%%%%%%%%%%%%%%%%%%%%%%%%%%%%%%%
\subsection*{Additional rules and remarks}
\begin{enumerate}
	\item \textbf{Table:} The table's rough location will be announced beforehand, having its position to the left, right, or behind the robot.
	\item \textbf{Incorrect categorization:} The score is reduced if an object is stored on the cabinet, but not on a shelf with similar objects; this reduction is applied per incorrectly stored object.
	\item \textbf{New category:} Objects that do not semantically belong to any of the categories represented on the shelves should be grouped together on a new shelf.
	\item \textbf{Deus Ex Machina:} The scores are reduced if human assistance is received, in particular for:
	\begin{itemize}
		\item telling or pointing out to the robot where to place an object
		\item moving an object instead of the robot
		\item opening the cabinet doors
	\end{itemize}
	\item \textbf{Communicating Perception}: The robot must clearly communicate its perception to the referee.
	Pointing at the object or attempting to pick it up is sufficient.
	When using visualization, make sure the robot tells the referee where to look and make the visualization easily accessible. 
	If the team wants to utilize bounding boxes make sure \textbf{only} one object with a bounding box is shown at a time, so the referee is easily able to check and verify.
	Also make sure the surrounding scene is visible, i.e. just showing a cropped bounding box is not enough.
\end{enumerate}

\subsection*{OC Instructions}

At least two hours before the test:
\begin{itemize}
	\item Announce which table and cabinet will be used in the test.
	\item Announce a rough location for the table.
	\item Select which bags will be used in the test.
	\item Select which container will be used in the test.
\end{itemize}

\subsection*{Referee Instructions}

The referee needs to:
\begin{itemize}
	\item Place 5--10 objects on the table.
	\item Place container and cornflakes on the table.
	\item Place 5--10 objects in a bag near the table.
	\item Place objects in the cabinet, grouping them by category or likeliness.
	\item Close the door of the cabinet.
\end{itemize}


% \newpage
\subsection*{Score sheet}

The maximum time for this test is 3 minutes.
The robot is given 1 extra minute to open the cupboard door. 
If the robot is not able to open the door within that minute, it will be opened by the referee. 
In case the robot opens the door within the minute, the robot has a small time advantage. 

\begin{scorelist}
% There are 5 filled shelves, originally with 2 objects, 1 in each corner.
% The table also has 10 objects, that the robot should move to the shelf.
% So 20 objects in total
% This can be a tight fit, as there will be potentially 4 objects per shelf, as the robots moves them from the table one by one

% The robots are not fast enough though to do more than 5 objects in the given time.

% Grasp (any object): 10
% Place (anywhere in the cupboard): 10
% Place in correct place: 15
% Recognize known object correctly (without grasping/placing something of that class): 10
% Label two unknown objects of the same class with the same label (e.g. ``class0''): 15

% Place known object near known object of same class: 40
% Place unknown object near unknown object of the same class: 50


	\scoreheading{Grasping objects}
	\scoreitem[5]{10}{For each successful grasp of any object (lifting it up to at least 5 cm for more than 10 seconds)}

	\scoreheading{Placing objects}
	\scoreitem[5]{10}{For each successful placement of an object anywhere in the cupboard (safely stands still for more than 10 seconds)}
	\scoreitem[5]{5}{For each successful placement of an object at correct place (near an object of the same class)}

	\scoreheading{Recognizing objects}
	\scoreitem[10]{5}{Every correctly recognized known or alike object in the report file}
	% TODO: Split up scores over these 3 variants of a correct label:
	% 1: As unknown, instead of wrongly applying a label from the known or alike objects (e.g. use an Open World assumption in your classifier)
	% 2: Label all instances of the same unknwon class with the same generated label, e.g. label0, so distinguish between different onknown objects
	% 3: Label the unknown objects with a meaningful label, eg. cookies in case its a sort of cookies. I.e. use a classifier that knows many classes. 
	\scoreitem[5]{15}{Correctly label unknown objects}
	\scoreitem[10]{-5}{False positive label}
	
% 	\scoreheading{Total task}
% 	\scoreitem[5]{40}{Place known object near known object of same class}
% 	\scoreitem[5]{50}{Place unknown object near unknown object of same class}

	\scoreheading{Bonus}
	\scoreitem[1]{20}{Open the door without human help}
	
	\setTotalScore{250}
\end{scorelist}


% Local Variables:
% TeX-master: "Rulebook"
% End:


% Local Variables:
% TeX-master: "Rulebook"
% End:
