\section{Give Me a Hand}
\label{test:hri-task}

\subsection*{Description}
The robot needs to place new items in the locations given by a human user through natural human-robot interaction.

\textbf{Main goal:}
New items have arrived at the house. The user takes each item from the box or bag, one by one, hands it to the robot, and instructs the robot to place it in a specified location.

\textbf{Optional goal:}
The robot must retrieve and deliver an item based on a remote command. 

\subsection*{Focus}
This task focuses on
\textit{Object perception},
\textit{Human perception},
\textit{Verbal interaction}, and
\textit{Human-Robot Interaction}.

\subsection*{Setup}
\begin{itemize}[nosep]	
	\item \textbf{Locations:} 
	\begin{itemize}
		\item This task takes place inside the Arena.
	\end{itemize}	 
	\item \textbf{People:} 
	\begin{itemize}
		\item There might be several persons inside the room (e.g. referee, assistant, and the user).
	\end{itemize}
	\item \textbf{Objects:}
		\begin{itemize}
			\item There is a box or bag with five random objects from the dataset.
		\end{itemize}
\end{itemize}

\subsection*{Procedure}
\begin{enumerate}[nosep]
    \item \textbf{Main goal:}
    \begin{enumerate}[nosep]
        \item The robot starts inside the arena, in the task room indicated by the referee.
        \item The robot approaches a calling user.
        \item The user takes an item out of the box and hands it to the robot. 
        \item The user instructs the robot to place the object in a specified location in the same room.
        \item The robot delivers the item and returns for a new one.
    \end{enumerate}
    \item \textbf{Optional goal:} 
    \begin{enumerate}[nosep]
        \item The user moves to a different room from the robot and then, requests an item using a remote command.
    \end{enumerate}
\end{enumerate}


\subsection*{Additional rules and remarks}
\begin{enumerate}[nosep]
	\item \textbf{HRI:}
	\begin{itemize}[nosep]
            \item  The user instructs the robot using human-robot interaction (e.g. \textit{Place this item there (pointing), next to the flower base}).
		\item At any moment, the robot may ask questions to confirm the information, request more details, or ask the user to repeat the instructions.
		\item In this task, natural human-robot interaction is the primary focus. When the robot needs to engage with the human, it must first capture the human’s attention using multimodal communication methods (e.g., visual signals, gestures, or sound cues) rather than relying solely on voice interaction. This is particularly important when the human and the robot are not in close proximity.
	\end{itemize}
	\item \textbf{Object manipulation:} 
        \begin{itemize}[nosep]
            \item For a full scoring, no \textit{Deus ex Machina} is allowed in while handing over the objects (e.g. \textit{Please, place the item in my gripper and wait until I close it in three, two, one}). However, the robot may warn the user about closing their manipulator, for example, \textit{I have reached the object, I will close my gripper in three seconds -- please, be careful}.
            \item The goal locations of objects in the environment may introduce some ambiguity. For example, there could be three bins or two plates positioned close to one another. Delivery instructions may specify actions like: \textit{"Place this apple on that plate"} or \textit{"Put this can in the trash bin."}
            \item Full points for a correct delivery will only be awarded if the object is placed inside (bins) or within a few centimeters (on surfaces) of the designated target location. The robot may seek confirmation or request further clarification if needed (e.g., Robot: \textit{"Is this the correct location?"} User: \textit{"No, move the object to the bin on your left."}).
        \end{itemize}
	\item \textbf{Remote communication:} 
        \begin{itemize}[nosep]
            \item Given the advances in communication technologies, this optional task encourages the use of modern platforms to facilitate human-robot interaction. In this task, the robot and the user are in different rooms, and the robot must retrieve and deliver an item to the user based on a remote command sent through smart assistants (e.g., Alexa or Siri), messaging applications (e.g., Telegram or Slack), custom-made graphical user interfaces, or other communication technologies. Teams are free to choose the technology they implement, fostering innovation and flexibility in human-robot communication.
        \end{itemize}
\end{enumerate}

\subsection*{Instructions:}
\subsubsection*{Referee instructions}

The referee needs to:
\begin{itemize}[nosep]
	\item Select random objects from the dataset and goal locations.
\end{itemize}

\subsubsection*{OC Instructions}
During the \SetupDays:
\begin{itemize}[nosep]
	\item Provide official objects with different shapes and sizes for training.
\end{itemize}
At least two hours before the test:
\begin{itemize}[nosep]
	\item Provide the location of the deposits.
\end{itemize}

\subsubsection*{Team instructions}

The team needs to:
\begin{itemize}[nosep]
	\item Instruct the referees on the communication technology to be used for the optional goal.
\end{itemize}

\subsection*{Score sheet}
\input{scoresheets/HRITask.tex}
