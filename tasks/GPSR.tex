\section{General Purpose Service Robot}\label{test:gpsr}

The robot is asked to understand and execute commands requiring a wide range of different abilities. \\

\noindent \textbf{Main Goal:} Execute three commands issued by the operator.\\

\noindent \textbf{Optional Goals:}
\begin{enumerate}[nosep]
	\item Understand a command given by a non-expert operator.
\end{enumerate}

\subsection*{Focus}
\emph{Task planning}, \emph{object/people detection and recognition}, \emph{object feature recognition}, \emph{object manipulation}

% %% %%%%%%%%%%%%%%%%%%%%%%%%%%%%%%%%%%%%%%%%%%%%%%%%%%%%%%
% Setup
% %% %%%%%%%%%%%%%%%%%%%%%%%%%%%%%%%%%%%%%%%%%%%%%%%%%%%%%%
\subsection*{Setup}
\begin{itemize}[nosep]
    \item \textbf{Locations:}
        \begin{itemize}
            \item \textbf{Task location:} The task takes place inside the \Arena{}. The \Arena{} is in its nominal configuration for this task. Commands may require the robot to leave the room.
            \item \textbf{Start location:} The robot starts outside the \Arena{}. When the door opens, it must navigate towards the \textit{Instruction Point}.
            \item \textbf{Instruction point:} The robot returns to this point after completing all of the commands.
        \end{itemize}
    \item \textbf{People:}
        \begin{itemize}
			\item \textbf{Professional Operator:} Typically the referee, issues standard commands to the robot.
			\item \textbf{Non-Expert Operator (optional):} A person without robotics expertise (e.g., from the audience). The referee provides the intended command goal to this operator, who then issues the instruction using their own wording.

            For example, the generated command might be \textit{"Bring me a coke from the kitchen."} then the non-expert operator will be told \textit{"The robot should bring you a coke, which is found in the kitchen."}, who then tells the robot \textit{I want a coke. Go to the kitchen and get me one."}

            If the robot consistently fails to understand the non-expert operator (e.g. after two retries), teams can default to a custom operator.
        \end{itemize}
\end{itemize}


% %% %%%%%%%%%%%%%%%%%%%%%%%%%%%%%%%%%%%%%%%%%%%%%%%%%%%%%%
% Procedure
% %% %%%%%%%%%%%%%%%%%%%%%%%%%%%%%%%%%%%%%%%%%%%%%%%%%%%%%%
\subsection*{Procedure}
\begin{enumerate}[nosep]
	\item \textbf{Instruction point:} At least two hours before the test, the referees announce the location of the \textit{Instruction Point}.
	\item \textbf{Command execution:} The operator provides up to three commands issued consecutively to instruct the robot to perform the tasks. If the operator provides the commands one-by-one, the robot must return to the operator after completing each task. The operator will say "please start now" to initiate the task.
	\item \textbf{Back to the instruction point:} The robot goes back to the \textit{Instruction Point} after completing all of the commands given by the operator.
	\item \textbf{Pausing the Timer:} If commands are issued one-by-one, the referee pauses the timer as soon as the robot reaches the instruction point to reset the arena for the next command. The timer resumes once the referee signals the start of the next command.
\end{enumerate}


% %% %%%%%%%%%%%%%%%%%%%%%%%%%%%%%%%%%%%%%%%%%%%%%%%%%%%%%%
% Additional Rules
% %% %%%%%%%%%%%%%%%%%%%%%%%%%%%%%%%%%%%%%%%%%%%%%%%%%%%%%%
\subsection*{Additional Rules and Remarks}
\begin{enumerate}[nosep]
	\item \textbf{Partial scoring:} The main task allows partial scoring (per \emph{completed} command).

	\item \textbf{Command generator:} Tasks will be generated using the official command generator\footnote{\url{https://github.com/RoboCupAtHome/CommandGenerator}}. Once a command has been generated it will be entered into an LLM to re-generate a similar phrase.
	\item \textbf{Test start:} The robot moves to the \textit{Instruction Point} when the arena door is open.

	\item \textbf{Non-Expert Operators:}
    \begin{itemize}[nosep]
        \item Referees must not assist non-expert operators in phrasing the commands.
        \item Teams are not allowed to coach, instruct, or bias non-expert operators. Doing so results in disqualification from the task.
    \end{itemize}

	\item \textbf{Deus ex Machina:} Score reductions apply in the following cases:
	\begin{itemize}
		\item Use of a custom operator.
		\item Bypassing speech recognition with an alternative HRI.
		\item Receiving human assistance to accomplish a command.
		\item Instructing a human assistant to perform the whole task.
		\item QR codes will not be available.
	\end{itemize}
\end{enumerate}

\subsection*{Referee Instructions}
\begin{itemize}[nosep]
	\item Provide the commands to the operators.
\end{itemize}

\subsection*{OC Instructions}

At least two hours before the test:
\begin{itemize}[nosep]
	\item Generate the robot commands and pass through LLM to get a similar command (do not reveal them to the teams).
	\item Announce the location of the instruction point.
	\item Recruit volunteers to assist during the test.
	\newline
\end{itemize}

\noindent During the test:
\begin{itemize}[nosep]
	\item If commands are given one-by-one, restore the arena to its nominal state before each command execution.
\end{itemize}

\subsection*{Score Sheet}
\begin{scorelist}[timelimit=7]
	\scoreheading{Main Goal}
	\scoreitem[3]{80}{Understand the spoken command}
	\scoreitem[3]{100}{Demonstrate a plan has been generated}
	\scoreitem[3]{250}{Solving the command}
	
	\scoreheading{Bonus Rewards}
	\scoreitem{200}{Interleaved Task Bonus}

	\scoreheading{Penalties}
	\penaltyitem[3]{20}{Using a custom operator}
	\penaltyitem[6]{30}{Request a rephrasing}
	\penaltyitem[3]{50}{Bypassing speech recognition}
	\penaltyitem[3]{250}{Human assistance: will apply a percentage penalty according to similar penalties in other tests.}
\end{scorelist}




% Local Variables:
% TeX-master: "Rulebook"
% End:
