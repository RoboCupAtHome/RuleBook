\section{General Purpose Service Robot}\label{test:gpsr}

The robot is asked to understand and execute commands requiring a wide range of different abilities. \\

\noindent \textbf{Main Goal:} Execute three commands issued by the operator.\\

\subsection*{Focus}
\emph{Task planning}, \emph{object/people detection and recognition}, \emph{object feature recognition}, \emph{object manipulation}

% %% %%%%%%%%%%%%%%%%%%%%%%%%%%%%%%%%%%%%%%%%%%%%%%%%%%%%%%
% Setup
% %% %%%%%%%%%%%%%%%%%%%%%%%%%%%%%%%%%%%%%%%%%%%%%%%%%%%%%%
\subsection*{Setup}
\begin{itemize}[nosep]
    \item \textbf{Locations:}
        \begin{itemize}
            \item \textbf{Task location:} The task takes place inside the \Arena{}. Commands may require the robot to leave the room. The \Arena{} is in its nominal configuration for this task.
            \item \textbf{Start location:} The robot starts outside the \Arena{}. When the door opens, it must navigate towards the \textit{Instruction Point}.
            \item \textbf{Instruction point:} The robot returns to this point after completing all the commands.
        \end{itemize}
    \item \textbf{People:}
        \begin{itemize}
			\item \textbf{Professional Operator:} The referee issues standard commands to the robot.
            If the robot consistently fails to understand the command (e.g. after three tries), teams can use a custom operator.
        \end{itemize}
\end{itemize}


% %% %%%%%%%%%%%%%%%%%%%%%%%%%%%%%%%%%%%%%%%%%%%%%%%%%%%%%%
% Procedure
% %% %%%%%%%%%%%%%%%%%%%%%%%%%%%%%%%%%%%%%%%%%%%%%%%%%%%%%%
\subsection*{Procedure}
\begin{enumerate}[nosep]
	\item \textbf{Instruction point:} At least two hours before the test, the referees announce the location of the \textit{Instruction Point}.
	\item \textbf{Command execution:} The robot will decide how the commands will be issued and advise the operator, i.e., either consecutively or one-by-one. If the commands are issued one-by-one, the robot must return to the operator after completing each task.
	\item \textbf{Back to the instruction point:} The robot goes back to the \textit{Instruction Point} after completing all the commands given by the operator.
	\item \textbf{Pausing the Timer:} The referee pauses the timer as soon as the robot reaches the instruction point to reset the arena for the next command. The timer resumes once the referee signals the start of the next command.
\end{enumerate}


% %% %%%%%%%%%%%%%%%%%%%%%%%%%%%%%%%%%%%%%%%%%%%%%%%%%%%%%%
% Additional Rules
% %% %%%%%%%%%%%%%%%%%%%%%%%%%%%%%%%%%%%%%%%%%%%%%%%%%%%%%%
\subsection*{Additional Rules and Remarks}
\begin{enumerate}[nosep]
	\item \textbf{Interleaved Task Bonus:} The robot receives an additional bonus if it successfully completes commands in an interleaved order rather than strictly consecutively. This bonus is awarded only when all three commands are received at once. The interleaved execution must be meaningful, for example by saving time or reducing unnecessary movements.
	\textit{Example:} The robot first picks up an object, then performs another task along the way, and only afterward delivers the object to its original destination. 
	\item \textbf{Partial Scoring:} The solution allows partial scoring.
	\item \textbf{Command generator:} Tasks will be generated using the official command generator\footnote{\url{https://github.com/RoboCupAtHome/CommandGenerator}}. Once a command has been generated it will be entered into an LLM to re-generate a similar phrase, e.g. the generated command is "get me a coke from the kitchen" re-phrased command is "Go to the kitchen, find a coke, and bring it to me". Each command may be re-phrased up to 3 times getting simpler with each rephrasing.   
	\item \textbf{Test start:} The robot moves to the \textit{Instruction Point} when the arena door is open.
	\item \textbf{Team Coaching:} Teams are not allowed to coach, or instruct the operators. Doing so results in disqualification from the task.
	\item \textbf{Custom Operators:} If a custom operator is used they can only choose between the three re-phrased commands to give.

	\item \textbf{Autonomy Skip:} Score reductions apply in the following cases:
	\begin{itemize}
		\item Use of a custom operator.
		\item Bypassing speech recognition.
		\item Receiving human assistance to accomplish a command.
		\item Instructing a human assistant to perform the whole task.
		\item QR codes will not be available.
	\end{itemize}
\end{enumerate}

\subsection*{Referee Instructions}
\begin{itemize}[nosep]
	\item Provide the commands to the operators.
\end{itemize}

\subsection*{OC Instructions}

At least two hours before the test:
\begin{itemize}[nosep]
	\item Generate the robot commands and pass through LLM to get a similar command (do not reveal them to the teams).
	\item Announce the location of the instruction point.
	\item Recruit volunteers to assist during the test.
\end{itemize}

\noindent During the test:
\begin{itemize}[nosep]
	\item The arena will be setup for all command executions.
\end{itemize}

\subsection*{Score Sheet}
\begin{scorelist}[timelimit=7]
	\scoreheading{Main Goal}
	\scoreitem[3]{80}{Understand the spoken command}
	\scoreitem[3]{100}{Demonstrate a plan has been generated}
	\scoreitem[3]{250}{Solving the command}
	
	\scoreheading{Bonus Rewards}
	\scoreitem{200}{Interleaved Task Bonus}

	\scoreheading{Penalties}
	\penaltyitem[3]{20}{Using a custom operator}
	\penaltyitem[6]{30}{Request a rephrasing}
	\penaltyitem[3]{50}{Bypassing speech recognition}
	\penaltyitem[3]{250}{Human assistance: will apply a percentage penalty according to similar penalties in other tests.}
\end{scorelist}




% Local Variables:
% TeX-master: "Rulebook"
% End:
