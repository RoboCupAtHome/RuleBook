\section{Help Out [Housekeeper]}
Smart speakers have become a staple device in modern households only a few years after their introduction because they fill a niche for a general assistive device in the home. A clear goal of upcoming domestic service robot technologies is to have a device that aids users in day-to-day tasks.

% \subsection{Focus}
% This test focuses on the detection and recognition of objects and their features, as well as object manipulation.

\subsection{Main Goal}
Accept verbal instructions from home occupants and perform those tasks.

\noindent\textbf{Reward:} 600pts (200 points per task)\\

\subsection{Bonus rewards}
\begin{enumerate}[nosep]
	\item Free-form English commands. The team may ask the commands to be re-worded from the generated command, in order to test more advanced natural language processing. (100 pts, each)
	\item Going home when instructed to do so. The robot must park itself in the home position and successfully come back when requested with, \textit {<Robot Name>, I need help!}. (50pts)
	\item Coming when called. Robot should come to the same room as the operator when requested with, \textit {<Robot Name>, I need help!}. (50pts)
\end{enumerate}

% %% %%%%%%%%%%%%%%%%%%%%%%%%%%%%%%%%%%%%%%%%%%%%%%%%%%%%%%
%
% Setup
%
% %% %%%%%%%%%%%%%%%%%%%%%%%%%%%%%%%%%%%%%%%%%%%%%%%%%%%%%%
\subsection{Setup}
\begin{enumerate}
	\item \textbf{Home:} An area in the arena is marked as the \textit{home} for the robot, which it goes to when it is not in use.

	\item \textbf{House Residents:} Three people in the arena will give tasks to the robot.

	\item \textbf{Object and Resident Placements:} Residents will instruct the robot to perform tasks in the various rooms of the home. Tasks should be pre-generated and the residents and items should be in the correct places before the round starts.
\end{enumerate}


% %% %%%%%%%%%%%%%%%%%%%%%%%%%%%%%%%%%%%%%%%%%%%%%%%%%%%%%%
%
% Setup
%
% %% %%%%%%%%%%%%%%%%%%%%%%%%%%%%%%%%%%%%%%%%%%%%%%%%%%%%%%
\subsection{Procedure}
\begin{enumerate}
	\item \textbf{Tasks} Residents may ask the robot to perform any command from the set of capabilities in all rounds of RoboCup@Home. Tasks will come from the generator.
	\item \textbf{Going Home} When requested, the robot should park itself in its \textit{home}. The robot must also wake up to perform the next command.
	\item \textbf{Coming When Called} When requested, the robot should come to another room to perform the next command. The requester will be in that room. To bypass this, an operator may give the command, \textit{Go to <room>}.
  

\end{enumerate}


% %% %%%%%%%%%%%%%%%%%%%%%%%%%%%%%%%%%%%%%%%%%%%%%%%%%%%%%%
%
% Additional Rules
%
% %% %%%%%%%%%%%%%%%%%%%%%%%%%%%%%%%%%%%%%%%%%%%%%%%%%%%%%%
\subsection{Additional rules and remarks}
\begin{enumerate}
	\item \textbf{Task Generation:}  Tasks will be generated using a generator which will be released no fewer than 3 months prior to the competition.
\end{enumerate}

\newpage
\subsection{OC instructions}

\textbf{5 hours+ before the test}
\begin{itemize}
	\item Choose a home location for the robot and inform teams of this location,
\end{itemize}

\textbf{2 hours before the test}
\begin{itemize}
	\item Generate notecards with tasks on them. Hide from the teams.
	\item Recruit home occupants to provide instructions to the robots.
\end{itemize}

\subsection{Referee instructions}
\begin{itemize}
	\item Place people and items in correct starting locations before activating the robot.
	\item Speak clearly. Follow all appropriate rules for alternative methods of activating the robot as necessary.
\end{itemize}


\newpage
\subsection{Score sheet}

The maximum time for this test is 5 minutes.

\begin{scorelist}
	\scoreheading{Main Goal}
	\scoreitem{600}{Perform each task (200 points per task)}

	\scoreheading{Bonus rewards}
	\scoreitem{300}{Respond to free-form English commands (100 points per task)}
	\scoreitem{50}{Go home}
	\scoreitem{50}{Come here}

	\setTotalScore{1000}
\end{scorelist}


% Local Variables:
% TeX-master: "Rulebook"
% End:


% Local Variables:
% TeX-master: "Rulebook"
% End:
