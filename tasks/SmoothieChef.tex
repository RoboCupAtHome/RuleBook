\section{Smoothie Chef [Party Host]}
\label{test:smoothie-chef}
The home owner teaches the robot a procedure to make a smoothie and the robot repeats the task on its own.\\

\noindent \textbf{Main goal:} The robot crafts a smoothie as taught, placing the correct 3 fruits in the right order.\\

\noindent \textbf{Optional goals:}
\begin{enumerate}[nosep]
	\item Pouring \emph{milk} without spilling
	\item Pouring \emph{sugar} without spilling
	\item Activating the blender/stir with the mixer
\end{enumerate}

% We will have the robot place the (fake) fruit corresponding to the desired smoothie into a bowl, which avoids needing blenders and needing to manipulate liquids. However, one of the fruits will be something small, like blueberries, which will need to be measured using an appropriate measuring cup.

\subsection*{Focus}
\emph{Action recognition and understanding}, \emph{demonstration-based learning}.

\subsection*{Setup}
\begin{itemize}[nosep]
	\item \textbf{Location:} The test takes place in the kitchen.
	\item \textbf{Start Llcation:} The robot starts outside the \Arena{} and navigates to the kitchen counter when the door is open.
	\item \textbf{Operator location:} The operator is standing in front of the robot on the opposite side of the counter.
	\item \textbf{Kitchen counter:} Either the kitchen table or counter can be used in this test.
	All ingredients are placed on the kitchen counter near an \emph{Official Blender}.
	A spoon to add sugar from a bowl is also available.
\end{itemize}

\subsection*{Additional Rules and Remarks}
\begin{enumerate}
	\item \textbf{Custom operator:} The training is normally conducted by a \emph{Professional Operator}.
	Teams can alternatively opt for a \emph{Custom Operator}, which causes a score reduction.
	\item \textbf{Smoothie ingredients:} On the counter, there are 6 \emph{fruits}, one bottle of milk, and a bowl with sugar; all of these are standard objects (see~\refsec{rule:scenario_objects}).
	The sugar is contained in a bowl and must be served with a spoon.
	It should be noted that any liquid can be used as milk (such as actual milk, but also water or something else from the \emph{Drinks} category).
	Similarly, any pourable solid can be used as sugar (such as actual sugar, but also muesli or cereal, for example).
	\item \textbf{Smoothie recipe:} A smoothie recipe includes three out of six random fruits to be blended.
	At a team's request, the recipe can also include one teaspoon of sugar and a glass of milk.
	\item \textbf{Official blender and mixer:} Unless a real blender is available (unlikely), teams can expect a glass or bowl to be used in this test.
	Any large object (e.g., a plastic bottle) from the list of standard objects (see~\refsec{rule:scenario_objects}) can be used as a mixer.
	\item \textbf{Deus ex Machina:} The scores are reduced if human assistance is received, in particular:
	\begin{itemize}[nosep]
		\item starting the test in front of the kitchen counter
		\item telling the robot which ingredient comes next (such as by pointing, naming, etc.)
		\item handing an object over to the robot
		\item spilling milk or sugar causes a reduction of the optional bonuses
	\end{itemize}
\end{enumerate}

\subsection*{OC Instructions}
During the \SetupDays:
\begin{itemize}
	\item Announce which objects will be used as a blender and a mixer.
\end{itemize}
2 hours before the test:
\begin{itemize}
	\item Announce the location of the kitchen counter.
	\item Announce the location where the operator and the robot will perform the task.
\end{itemize}

\subsection*{Referee Instructions}

The referee needs to:
\begin{itemize}
	\item Clean any spills as necessary.
	\item Perform a demonstration of the smoothie preparation process for the robot.
	\item Shuffle the fruits and put all utensils within a reachable distance from the robot.
\end{itemize}

\subsection*{Score sheet}
The maximum time for this test is 10 minutes.

\begin{scorelist}
	\scoreheading{Main Goal}
	\scoreitem{750}{Placing all three fruits in the blender in the right order}

	\scoreheading{Bonus Rewards}
	\scoreitem{750}{Pouring sugar into the blender}
	\scoreitem{500}{Pouring milk into the blender}
	\scoreitem{250}{Activating the blender/stir with the mixer}
	\penaltyitem{600}{Spilling sugar}
	\penaltyitem{400}{Spilling milk}

	\scoreheading{Deus Ex Machina Penalties}
	\penaltyitem{150}{Starting the test in front of the kitchen counter}
	\penaltyitem{200}{Handing an object over to the robot}
	\penaltyitem{250}{Telling the robot which ingredient comes next}
	\penaltyitem{250}{Using a custom operator}
	% No longer necessary, computes automatically
	% \setTotalScore{1000}
\end{scorelist}


% Local Variables:
% TeX-master: "Rulebook"
% End:

