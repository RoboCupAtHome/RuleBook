\section{Enhanced General Purpose Service Robot}
\label{test:egpsr}
Similar to a smart speaker, the robot is asked to execute arbitrary commands requested by an operator.\\
\textbf{Note 1:} Due to this task's more open-ended nature, the maximum possible score is higher than the scores of the other \STWO{} tasks.\\
\textbf{Note 2:} The commands in this task are more complex than in the GPSR task in \SONE.\\

\noindent \textbf{Main Goal:} Execute 3 commands requested by the operator.\\

\noindent \textbf{Optional Goals:}
\begin{enumerate}[nosep]
	\item Understand a command given by a non-expert operator.
\end{enumerate}

\subsection*{Focus}
\emph{Task planning}, \emph{object/people detection and recognition}, \emph{object feature recognition}, \emph{object manipulation}


% %% %%%%%%%%%%%%%%%%%%%%%%%%%%%%%%%%%%%%%%%%%%%%%%%%%%%%%%
% Setup
% %% %%%%%%%%%%%%%%%%%%%%%%%%%%%%%%%%%%%%%%%%%%%%%%%%%%%%%%
\subsection*{Setup}
\begin{enumerate}
    \item \textbf{Locations:}
		\begin{itemize}
			\item \textbf{Test location:} The task takes place inside the \Arena{}, but some commands may require the robot to go out. The \Arena{} is in its nominal state for this test.
			\item \textbf{Start location:} The robot starts outside the \Arena{}. When the door opens, the robot moves towards the \textit{Instruction Point}.
			\item \textbf{Instruction point:} At the beginning of the test, as well as after finishing the first and second command, the robot moves to the \textit{Instruction Point}.
		\end{itemize}
	\item \textbf{People:}
		\begin{itemize}
			\item \textbf{Operators:} A \emph{Professional Operator} (the referee) commands the robot to execute a task.
			Optionally, commands can be issued by a \emph{Non-Expert Operator}, i.e. a person from the audience with no robotics background.
			In this case, the referee gives the goal of the command to the non-expert operator, who will then issue it to the robot in their own words. For example, the generated command might be ``Bring me a coke from the kitchen." then the non-expert operator will be told ``The robot should bring you a coke, which is found in the kitchen.", who then tells the robot ``I want a coke. Go to the kitchen and get me one."
			If the robot consistently fails to understand the non-expert operator (e.g. after two retries), teams can default to a custom operator.
		\end{itemize}
\end{enumerate}


% %% %%%%%%%%%%%%%%%%%%%%%%%%%%%%%%%%%%%%%%%%%%%%%%%%%%%%%%
% Procedure
% %% %%%%%%%%%%%%%%%%%%%%%%%%%%%%%%%%%%%%%%%%%%%%%%%%%%%%%%
\subsection*{Procedure}
\begin{enumerate}[nosep]
	\item \textbf{Instruction point:} Two hours before the test, the referees announce the location of the \textit{Instruction Point}.
	\item \textbf{Test start:} The robot moves to the \textit{Instruction Point} when the arena door is open.
	\item \textbf{Command execution:} The operator instructs the robot to execute a command and the robot performs the task.
	\item \textbf{Back to the instruction point:} The robot goes back to the \textit{Instruction Point} and waits for the next command.
\end{enumerate}


% %% %%%%%%%%%%%%%%%%%%%%%%%%%%%%%%%%%%%%%%%%%%%%%%%%%%%%%%
% Additional Rules
% %% %%%%%%%%%%%%%%%%%%%%%%%%%%%%%%%%%%%%%%%%%%%%%%%%%%%%%%
\subsection*{Additional rules and remarks}
\begin{enumerate}[nosep]
	\item \textbf{Partial scoring:} The main task allows partial scoring (per \emph{completed} command).

	\item \textbf{Command generator:} Tasks will be generated using the official \emph{GPSR Command Generator}\footnote{\url{https://github.com/kyordhel/GPSRCmdGen}}.
	Commands for EGPSR are either \STWO{} tasks, complex commands requiring to perform chains of subtasks in a sequence, or \emph{incomplete} commands lacking relevant information to succeed.

	\item \textbf{Non-expert operators:} A \emph{Non-Expert Operator} receives a note-card with the summarized command from the referee and rephrases the command.
	Referees are not allowed to instruct non-expert operators on how to operate the robot.
	Teams attempting to instruct or bias the operator will be disqualified from the task.

	\item \textbf{Custom operator fallback:} If the robot consistently fails to understand the professional or non-expert operator (three times in a row), the referee will request a custom operator to issue commands.

	\item \textbf{Deus ex Machina:} The scores are reduced if human assistance is received, in particular for:
	\begin{itemize}
		\item using a custom operator
		\item bypassing speech recognition by using an alternative HRI
		\item receiving human assistance to accomplish a task (score reduction is applied consistent with the other tasks in this rulebook)
		\item instructing a human assistant to perform the whole task
	\end{itemize}
\end{enumerate}

\subsection*{Referee instructions}
\begin{itemize}
	\item Provide the note-cards to the operators.
\end{itemize}

\subsection*{OC instructions}

2h before the test:
\begin{itemize}
	\item Generate the robot commands (don't reveal them to the teams!).
	\item Announce the location of the instruction point.
	\item Recruit volunteers to assist during the test.
\end{itemize}

\noindent During the test:
\begin{itemize}[nosep]
	\item Rearrange the arena so that it is in its nominal condition before each command.
\end{itemize}

\subsection*{Score sheet}
\begin{scorelist}[timelimit=10]
	\scoreheading{Main Goal (can be repeated unlimited times)}
	\scoreitem[3]{150}{Find and clearly state an encountered problem}
	\scoreitem[3]{650}{Solve a problem}

	\scoreheading{Penalties}
	\penaltyitem[1]{100}{Find repeated problem category}
	\penaltyitem[1]{300}{Solving repeated problem category for the 2nd time}
	\penaltyitem[1]{500}{Solving repeated problem category for the 3rd (or more) time}

	\scoreheading{Deus Ex Machina Penalties}
	\penaltyitem[3]{150}{Asking for location of a problem}
	\penaltyitem[3]{650}{
		Instructing a human to perform parts of the task will apply a \\
		percentage penalty according to similar penalties in other \\
		Stage II tests.
		}

\end{scorelist}

\ifShortScoresheet{}{
	\textbf{Referee Information:}
	Note each problem (category, item, location) and mark if they are stated by the robot in remarks
}%
	


% Local Variables:
% TeX-master: "Rulebook"
% End:



% Local Variables:
% TeX-master: "Rulebook"
% End:
