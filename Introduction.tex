%% %%%%%%%%%%%%%%%%%%%%%%%%%%%%%%%%%%%%%%%%%%%%%%%%%%%%%%%%%%%%%%%%%%%%%%%%%%%
%%
%%    author(s): RoboCupAtHome Technical Committee(s)
%%  description: Introduction
%%
%% %%%%%%%%%%%%%%%%%%%%%%%%%%%%%%%%%%%%%%%%%%%%%%%%%%%%%%%%%%%%%%%%%%%%%%%%%%%
\chapter{Introduction}
\label{chap:introduction}


\section{RoboCup}
\iterm{RoboCup} is an international joint project to promote AI, robotics, and related fields. It is an attempt to foster AI and intelligent robotics research by providing standard problems where a wide range of technologies can be integrated and examined. More information can be found at http://www.robocup.org/.

\section{RoboCup@Home}
The \iterm{RoboCup@Home} league aims to develop service and assistive robot technology with high relevance for future personal domestic applications. It is the largest international annual competition for autonomous service robots and is part of the RoboCup initiative. A set of benchmark tests is used to evaluate the robots abilities and performance in a realistic non-standardized home environment setting. Focus lies on the following domains but is not limited to: Human-Robot-Interaction and Cooperation, Navigation and Mapping in dynamic environments, Computer Vision and Object Recognition under natural light conditions, Object Manipulation, Adaptive Behaviors, Behavior Integration, Ambient Intelligence, Standardization and System Integration. It is collocated with the RoboCup symposium.

%% %%%%%%%%%%%%%%%%%%%%%%%%%%%%%%%%%%%%%%%%%%%%%%%%%%%%%%%%%%%%%%%%%%%%%%%%%%%
%%
%%  author(s): RoboCupAtHome Technical Committee(s)
%%  description: Introduction - Organization
%%
%% %%%%%%%%%%%%%%%%%%%%%%%%%%%%%%%%%%%%%%%%%%%%%%%%%%%%%%%%%%%%%%%%%%%%%%%%%%%
\section{Organization}
\label{sec:introduction:organization}
\AtHome{} is organized into three subcommittees. Current members are listed at: 
\url{https://athome.robocup.org/committees/}.

\paragraph{Executive Committee}
\label{sec:introduction:ec}
The \EC{} consists of members of the board of trustees and representatives of each activity area.

\paragraph{Technical Committee}
\label{sec:introduction:tc}
The \TC{} is responsible for the rules of the league. Main focus is writing the rulebook and refereeing.
Members of the \EC{} are always members of the \TC{} as well.

\paragraph{Organizing Committee}
\label{sec:introduction:oc}
The \OC{} is responsible for the organization of the competition. They create the schedule and provide information about the scenario.
The \LOC{} is responsible for the set up and organization of the venue.


%% %%%%%%%%%%%%%%%%%%%%%%%%%%%%%%%%%%%%%%%%%%%%%%%%%%%%%%%%%%%%%%%%%%%%%%%%%%%
%%
%%    author(s): RoboCupAtHome Technical Committee(s)
%%  description: Introduction - Infrastructure
%%
%% %%%%%%%%%%%%%%%%%%%%%%%%%%%%%%%%%%%%%%%%%%%%%%%%%%%%%%%%%%%%%%%%%%%%%%%%%%%
\section{Infrastructure}
\label{sec:introduction:infrastructure}
\subsection{RoboCup@Home Mailing List}
\label{sec:introduction:mailinglist}
The official \AtHome{} mailing list can be reached at:
\begin{center}
\href{mailto:robocup-athome@lists.robocup.org}{\texttt{robocup-athome@lists.robocup.org}}
\end{center}
You can subscribe to the mailing list at:
\begin{center}
{\small\url{http://lists.robocup.org/cgi-bin/mailman/listinfo/robocup-athome}}
\end{center}

\subsection{RoboCup@Home Web Page}
\label{sec:introduction:webpage}
The official \AtHome{} website that also hosts this rulebook can be found at:
\begin{center}
{\small\url{https://athome.robocup.org/}}
\end{center}

\subsection{RoboCup@Home Rulebook Repository}
\label{sec:introduction:repo}
The official \AtHome{} \RR{} is where rules are publicly discussed before applying changes.
The entire \AtHome{} community is welcome and encouraged to actively participate in creating and discussing the rules. The \RR{} can be reached at:
\begin{center}
{\small\url{https://github.com/RoboCupAtHome/RuleBook/}}
\end{center}

\subsection{RoboCup@Home Telegram Group}
\label{sec:introduction:telegramgroup}
The official \AtHome{} \TG{} is a communication channel for the \AtHome{} community where rules are discussed, announcements are made, and questions are answered.
Beyond supporting the technical aspects of the competition, the group is a meeting point to stay in contact with the community, foster knowledge exchange, and strengthen relationships.
The \TG{} can be reached at:
\begin{center}
{\small\url{https://t.me/RoboCupAtHome}}
\end{center}

\subsection{RoboCup@Home Wiki}
\label{sec:introduction:wiki}
The official \AtHome{} \WIKI{} is meant to be a central place to collect information on all topics related to the \AtHome league. It was set up to simplify and unify the exchange of relevant information.
This includes but is certainly not limited to hardware, software, media, data, and more.
The \WIKI{} can be reached at:
\begin{center}
{\small\url{https://github.com/RoboCupAtHome/AtHomeCommunityWiki/wiki}}
\end{center}



%% %%%%%%%%%%%%%%%%%%%%%%%%%%%%%%%%%%%%%%%%%%%%%%%%%%%%%%%%%%%%%%%%%%%%%%%%%%%
%%
%%    author(s): RoboCupAtHome Technical Committee(s)
%%  description: Introduction - Leagues
%%
%% %%%%%%%%%%%%%%%%%%%%%%%%%%%%%%%%%%%%%%%%%%%%%%%%%%%%%%%%%%%%%%%%%%%%%%%%%%%
\section{Leagues}
\label{sec:leagues}

\AtHome{} is divided into three Leagues. One of these grants complete freedom to all competitors with respect to the robot used, while two of them are \SPLs{}, namely all competitors use the same robot. The official leagues and their names are:
\begin{itemize}
  \item \OPL
  \item \DSPL
  \item \SSPL
\end{itemize}

\begin{wrapfigure}[21]{r}{0.30\textwidth}
	\vspace{-30pt}
	\begin{center}
		\includegraphics[width=0.25\textwidth]{images/toyota_hsr.png}
		\vspace{-10pt}
		\caption{Toyota HSR}
		\label{fig:toyota_hsr}
	\end{center}

	\vspace{-20pt}
	\begin{center}
		\includegraphics[width=0.20\textwidth]{images/softbank_pepper.png}
		\vspace{-10pt}
		\caption{Softbank / Aldebaran Pepper}
		\label{fig:softbank_pepper}
	\end{center}
\end{wrapfigure}
Each league focuses on a different aspect of service robotics by targeting specific abilities.

\subsection{Domestic Standard Platform League (DSPL)}

The main goal of the DSPL is to assist humans in a domestic environment, paying special attention to elderly people and people suffering from illness or disability.
As a consequence, the DSPL focuses on \AmI, \CV, \OM, safe indoor \NAV{} and \MAP, and \TP.
The robot used in the DSPL is the \HSR, shown in Figure \ref{fig:toyota_hsr}.

\subsection{Social Standard Platform League (SSPL)}

The SSPL takes robots away from a traditional passive servant role, as the robot is now the one who actively looks for interaction.
From a party waiter in a domestic environment to a hostess in a museum or shopping mall, in SSPL we look for the next user who may require the robot's services.
Hence, this league focuses on \HRI, \NLP, \PerDet{} and \PerRec, \AB, and safe outdoor \NAV{} and \MAP.
The robot to be used in the SSPL is the \PEPPER, shown in Figure \ref{fig:softbank_pepper}.

\subsection{Open Platform League (OPL)}

The OPL has had the same modus operandi since the foundation of \AtHome.
With no hardware constrains, OPL is the league for teams who want to test their own robot designs and configurations, as well as for old at-homers.
In this league, robots are tested to their limits without having in mind any concrete design restriction, although the scope is similar to the DSPL.


%% %%%%%%%%%%%%%%%%%%%%%%%%%%%%%%%%%%%%%%%%%%%%%%%%%%%%%%%%%%%%%%%%%%%%%%%%%%%
%%
%%    author(s): RoboCupAtHome Technical Committee(s)
%%  description: Introduction - Competition
%%
%% %%%%%%%%%%%%%%%%%%%%%%%%%%%%%%%%%%%%%%%%%%%%%%%%%%%%%%%%%%%%%%%%%%%%%%%%%%%
\section{Competition}
The competition consists of 2 \emph{Stages} and the \iterm{Finals}. Each stage consists of a series of \iterm{Tests} that are being held in a daily life environment. The best teams from \iterm{Stage~I} advance to \iterm{Stage~II} which consists of more difficult tests. The competition ends with the \emph{Finals} where only the two highest ranked teams of each league compete to select the winner.

%% %%%%%%%%%%%%%%%%%%%%%%%%%%%%%%%%%%%%%%%%%%%%%%%%%%%%%%%%%%%%%%%%%%%%%%%%%%%
%%
%%    author(s): RoboCupAtHome Technical Committee(s)
%%  description: Introduction - Awards
%%
%% %%%%%%%%%%%%%%%%%%%%%%%%%%%%%%%%%%%%%%%%%%%%%%%%%%%%%%%%%%%%%%%%%%%%%%%%%%%
\section{Awards}
\label{sec:awards}
The RoboCup@Home league features the following \iterm{awards}.

\subsection{Winner of the competition}
For each sub-league, there will be a 1st, 2nd, and 3rd place award.

%
% As of 2017, the Execs have decided to remove the Innovation Award since
% is rarely given and its discussion is time consuming.
%
% \subsection{Innovation award}
% \label{award:innovation}
% To honour outstanding technical and scientific achievements as well as applicable solutions in the @Home league, a special \iterm{innovation award} may be given to one of the participating teams. Special attention is being paid to making usable robot components and technology available to the @Home community.
% 
% The \iaterm{Executive Committee}{EC} members from the RoboCup@Home league nominate a set of candidates for the award. The \iaterm{Technical Committee}{TC} elects the winner. A TC member whose team is among the nominees is not allowed to vote.
% 
% There is no innovation award in case no outstanding innovation and no nominees, respectively.


%
% As of 2017, the TC have decided to add an award for the best alternate HRI method
% to bypass speech recognition.
%
\subsection{Best Human-Robot Interface award}
\label{award:hri}
To honour outstanding Human-Robot Interfaces developed for interacting with robots in the @Home league, a special \iterm{Best HR Interface award} may be given to one of the participating teams. Special attention is being paid to making the interface open and available to the @Home community.

The \iaterm{Executive Committee}{EC} members from the RoboCup@Home league nominate a set of candidates for the award. The \iaterm{Technical Committee}{TC} elects the winner. A TC member whose team is among the nominees is not allowed to vote.
 
There is no Best HR Interface award in case no outstanding interface and no nominees, respectively.

%
% As of 2013, the Execs have decided to remove the Innovation Award due to
% the lack of interest of the participants
%
% \subsection{Winner of the Technical Challenge}
% In parallel to the regular competition, the RoboCup@Home league features a \iterm{technical challenge}. The winner of the technical challenge is given a special \iterm{award for winning the technical challenge}.
%
% As with the innovation award, the award for winning the technical challenge is not given in case no team shows a \emph{sufficient performance}. The decision which team wins the technical challenge, and if the award is given at all, is conducted by the \iaterm{Technical Committee}{TC}.

\subsection{Skill Certificates}
  \label{award:skill}
  The @Home league features certificates for the robots best at a the skills below:
  \begin{itemize}
   \item Navigation
   \item Manipulation
   \item Speech Recognition
   \item Person Recognition
  \end{itemize}
  
  A team is given the certificate if it scored at least 75\% of the attainable points for that skill.
  This is counted over all challenges, so e.g. if the robot scores manipulation points during the navigation test to open the door, that will count for the Manipulation-certificate.
  The certificate will only be handed out if the team is \emph{not} the overall winner of the competition. 


% Local Variables:
% TeX-master: "Rulebook"
% End:
