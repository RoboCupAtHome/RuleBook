%%%%%%%%%%%%%%%%%%%%%%%%%%%%%%%%%%%%%%%%%%%%%%%%%%%%%%%%%
\section{Organization of the Competition}
\label{sec:procedure_during_competition}

\subsection{Stage System}\label{rule:stages}

The competition features a \iterm{stage system}, namely it is organized in two stages, each consisting of a number of specific tasks, and ends with a \FINAL.
\begin{enumerate}
	\item \textbf{Robot Inspection:} For security, robots are inspected during the \SetupDays.
	A robot must pass the \RobotInspection{} test (see~\refsec{sec:robot_inspection}) so that it is allowed to compete.

	\item \textbf{Stage~I:} The first days of the competition are called \SONE.
	All qualified teams that have passed the \RobotInspection{} can participate in \SONE.

	\item \textbf{Stage~II:} The best \emph{50\% of teams} after \SONE{} advance to \STWO. If the total number of teams is less than 12, up to 6 teams may advance to \STWO.
	In this stage, tasks require more complex abilities or combinations of abilities.

	\item \textbf{Final demonstration:} The best \emph{two teams} of each league, namely the ones with the highest score after \STWO, advance to the \FINAL.
	The final round features only a single task integrating all tested abilities.
	In order to participate in the \FINAL, a team must have solved at least one of the \STWO{} tasks.
\end{enumerate}
Note that the same task can be performed multiple times during \SONE{} and \STWO{} (see~\refsec{rule:score_system}).
In case of having no considerable score deviation between a team advancing to the next stage and a team dropping out, the TC may announce additional teams advancing to the next stage.

%%%%%%%%%%%%%%%%%%%%%%%%%%%%%%%%%%%%%%%%%%%%%%%%%%%%%%%%%
\subsection{Thematic Scenarios}
\label{rule:themes}

Each stage includes a set of tasks that are grouped in two thematic scenarios:
\begin{enumerate}
	\item The \iterm{Housekeeper} scenario features tasks related to cleaning, organization, and maintenance of a household.
	\item The \iterm{Party Host} scenario focuses on providing general assistance during a party by attending to the needs of the guests.
\end{enumerate}

%%%%%%%%%%%%%%%%%%%%%%%%%%%%%%%%%%%%%%%%%%%%%%%%%%%%%%%%%
\subsection{Schedule}
\label{rule:schedule}

\begin{enumerate}
	\item \textbf{Thematic scenario blocks:} Two \Testblocks{} are scheduled per day, lasting between two and three hours.
	All teams have at least 2 \Testslots{} assigned per \Testblock{}, during which they can perform any task of their choice from the block's assigned scenario.

	\item \textbf{Test slots:} In principle, all teams get the same amount of \Testslots{} with a minimum of 2 per block.
	In case there is sufficient unoccupied time, a referee may open an extra testing slot, thereby allowing an extra run for all teams.

	\item \textbf{Tests:} Teams must inform the OC in advance about the tasks that they want to perform in each block.
	Note that only one task can be attempted per \Testslot, but different tasks may be attempted in different \Testslots{} within a \Testblock{} (i.e. one task can be attempted in the first slot, while another one in the following slot, if desired).

	\item \textbf{Participation is default:} Teams have to inform the OC in advance if they are skipping a \Testslot. Without such indication, they may receive a penalty when not attending (see~\refsec{rule:not_attending}).
\end{enumerate}

% Please add the following required packages to your document preamble:
% \usepackage[table,xcdraw]{xcolor}
% If you use beamer only pass "xcolor=table" option, i.e. \documentclass[xcolor=table]{beamer}
\begin{table}[h]
	\centering\small
	\newcommand{\teams}[2]{%
		\tiny
		\begin{tabular}{c}%
			\textit{Test slot 1, team $#1$}\\
			\textit{Test slot 1, team $#2$}\\
			$\vdots$\\
			\textit{Test slot $n$, team $#1$}\\
			\textit{Test slot $n$, team $#2$}\\
		\end{tabular}
	}
	\newcommand{\wcell}[2]{%
		\parbox[c]{2.5cm}{%
			\vspace{#1}%
			\centering%
			#2%
			\vspace{#1}%
		}%
	}
	\newcommand{\cell}[1]{\wcell{0.2\baselineskip}{#1}}
	% \newcommand{\mr}[1]{\multirow{2}{*}{#1}}


	\begin{tabular}{
		>{\centering\arraybackslash}m{2.5cm}|%
		>{\columncolor[HTML]{9AFF99}}c |%
		>{\columncolor[HTML]{9AFF99}}c |%
		>{\columncolor[HTML]{CBCEFB}}c |%
		>{\columncolor[HTML]{FF8D27}}c  %
	}
	\multicolumn{1}{ c }{}
		& \multicolumn{1}{ c }{\cellcolor{white} Day 1 }
		& \multicolumn{1}{ c }{\cellcolor{white} Day 2 }
		& \multicolumn{1}{ c }{\cellcolor{white} Day 3 }
		& \multicolumn{1}{ c }{\cellcolor{white} Day 4 }
		\\\hhline{~---~}

	\cell{Block 1\\\footnotesize(9:00--12:00)}
		& \cell{Housekeeper\\\teams{i}{j}}
		& \cell{Housekeeper\\~\\Party Host}
		& \cell{Restaurant}
		& \cellcolor{white}
		\\\hhline{~----}



	\multicolumn{1}{ c }{}
		& \multicolumn{3}{ c }{\wcell{0.5\baselineskip}{\color{gray}Lunch}}
		& \multicolumn{1}{|c|}{\cellcolor[HTML]{FF8D27}\cell{\textbf{Finals}}}
		\\\hhline{~----}

	\cell{Block 2\\\footnotesize(14:00--17:00)}
		& \cell{Party Host\\\teams{i}{k}}
		& \cellcolor[HTML]{CBCEFB}\cell{Party Host}
		& \cell{Housekeeper}
		& \cellcolor{white}
		\\\hhline{~---~}

	\multicolumn{1}{ c }{}
		& \multicolumn{1}{ c }{\wcell{0.5\baselineskip}{\color[HTML]{029734}Stage 1}}
		& \multicolumn{1}{ c }{\cellcolor{white}}
		& \multicolumn{1}{ c }{\wcell{0.5\baselineskip}{\color[HTML]{6668e5}Stage 2}}\\
	\end{tabular}

	\caption{Example schedule.
		Each team has at least two \Testslots{} assigned in every \Testblock.
		At least two \Testblocks{} are scheduled per day with an assigned theme.
		A team can choose a different task in each \Testslot, which means at least four different tests per stage.
	}
	\label{tbl:schedule}
\end{table}

\noindent Note that the actual allocation of blocks will be announced by the OC during the \SetupDays{} (see Table \ref{tbl:schedule}).

\subsection{Scoring System}
\label{rule:score_system}

Each task has a main objective and a set of bonus scores.
To score in a test, a team must successfully accomplish the main objective of the task; bonuses are not awarded otherwise.

The scoring system has the following constrains:
\begin{enumerate}
	\item \textbf{\SONE:} The maximum total score per task in \SONE{} is \scoring{1000 points}.
	\item \textbf{\STWO:} The maximum total score per task in \STWO{} is \scoring{2000 points}.
	\item \textbf{\FINAL:} The final score is normalized.
	\item \textbf{Minimum score:} The minimum total score per test in \SONE{} and \STWO{} is \scoring{0 points}.
	In principle, teams cannot receive negative points, except if they receive penalties.
	In particular, both penalties for not attending (see~\refsec{rule:not_attending}) and extraordinary penalties (see~\refsec{rule:extraordinary_penalties}) can result in a total negative score.
\end{enumerate}

% Local Variables:
% TeX-master: "../Rulebook"
% End:
