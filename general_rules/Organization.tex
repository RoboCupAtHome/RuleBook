%%%%%%%%%%%%%%%%%%%%%%%%%%%%%%%%%%%%%%%%%%%%%%%%%%%%%%%%%
\section{Competition Procedure}
\label{sec:rules:competitionprocedure}

A \RoboCup\AtHome{} competition consists of the following stages:

\begin{enumerate}
	\item \textbf{\RobotInspection{}:} For security, robots are inspected during \SetupDays{}.
	All registered teams can participate.

	\item \textbf{\SONE{}:} First set of tests, assessing the robot's basic abilities.
	Only teams that passed the \RobotInspection{} can participate.

	\item \textbf{\STWO{}:} Second set of tests, assessing more complex abilities and behaviors.
	The best \SI{50}{\percent} of teams\footnotemark{} (after \SONE{}) can participate.
	\footnotetext{If the total number of teams is less than 12, up to 6 teams may advance to \STWO{}}

	\item \textbf{\FINAL{}:} An open demonstration, asking teams to showcase complex behaviors and novel approaches. The two best scoring teams (\SONE{} and \STWO{} combined) can participate.
\end{enumerate}

\begin{table}[h]
	\newcolumntype{C}[1]{>{\centering\let\newline\\\arraybackslash\hspace{0pt}}m{#1}}
	\newcolumntype{S}{C{1.6cm}}
	\newcolumntype{M}{C{3.2cm}}
	\begin{center}
		\begin{tabularx}{14.56cm}{S|S|S|S|S|S|S|S}
			\hline
			\multicolumn{2}{|M|}{ \cellcolor[HTML]{FFFFC7}Setup Days} &
			\multicolumn{2}{M|}{ \cellcolor[HTML]{67FD9A}\iterm{Stage~I}} &
			\multicolumn{2}{M|}{ \cellcolor[HTML]{9698ED}\iterm{Stage~II}} &
			\multicolumn{2}{M|}{ \cellcolor[HTML]{FFCCC9}\iterm{Finals}}\\
			\hline
			%Second row
			\multicolumn{1}{S|}{} &
			\multicolumn{2}{M|}{$\xrightarrow{advance}$\newline All teams that \newline passed Inspection} &
			\multicolumn{2}{M|}{$\xrightarrow{advance}$\newline Best 6 ($<12$) \newline or best 50\% ($\geq 12$)} &
			\multicolumn{2}{M|}{$\xrightarrow{advance}$\newline Best 2 \newline teams} &
			\multicolumn{1}{C{1.2cm}}{~}
			\\ \cline{2-7}
		\end{tabularx}
	\end{center}
\end{table}

\noindent In case of having no considerable score deviation between a team advancing to the next stage and a team dropping out, the \TC{} may announce additional teams advancing to the next stage.

\subsection{Scenarios}
\label{sec:rules:scenarios}
The tests in \SONE{} and \STWO{} are divided in two thematic scenarios:
\begin{itemize}
	\item \textbf{\Housekeeper{}:} Features tests related to cleaning, organizing, and maintenance.
	
	\item \textbf{\Partyhost{}:} Focuses on providing general assistance during a party by attending the needs of the guests.
\end{itemize}


\subsection{Schedule}
\label{sec:rules:schedule}
There are two \Testblocks{} in a competition day. Each block has a stage and one or two thematic scenarios assigned. An exception is the \emph{Restaurant} test (see~\ref{test:restaurant}) which has its own block. During a block, each team has at least two \Testslots{} available, where they can choose which test, fitting the stage and scenario, they want to perform. The teams must inform the \OC{} which tests they will perform a day prior, usually in the \TLM{} (see~\ref{sec:rules:teamleadermeeting}). Teams have to indicate to the \abb{OC} when they are skipping a \Testslot{}. Without such indication, they may receive a penalty (see~\ref{sec:rules:missingslot}).

% Please add the following required packages to your document preamble:
% \usepackage[table,xcdraw]{xcolor}
% If you use beamer only pass "xcolor=table" option, i.e. \documentclass[xcolor=table]{beamer}
\begin{table}[H]
	\centering\small
	\newcommand{\teams}[2]{%
		\tiny
		\begin{tabular}{c}%
			\textit{Slot $1$, Team $#1$}\\
			$\vdots$\\
			\textit{Slot $N$, Team $#2$}\\
			\textit{Slot $N+1$, Team $#1$}\\
			$\vdots$\\
			\textit{Slot $2N$, Team $#2$}\\
		\end{tabular}
	}
	\newcommand{\wcell}[2]{%
		\parbox[c]{2.5cm}{%
			\vspace{#1}%
			\centering%
			#2%
			\vspace{#1}%
		}%
	}
	\newcommand{\cell}[1]{\wcell{0.2\baselineskip}{#1}}
	% \newcommand{\mr}[1]{\multirow{2}{*}{#1}}


	\begin{tabular}{
		>{\centering\arraybackslash}m{2.5cm}|%
		>{\columncolor[HTML]{9AFF99}}c |%
		>{\columncolor[HTML]{9AFF99}}c |%
		>{\columncolor[HTML]{CBCEFB}}c |%
		>{\columncolor[HTML]{FF8D27}}c  %
	}
	\multicolumn{1}{ c }{}
		& \multicolumn{1}{ c }{\cellcolor{white} Day 1 }
		& \multicolumn{1}{ c }{\cellcolor{white} Day 2 }
		& \multicolumn{1}{ c }{\cellcolor{white} Day 3 }
		& \multicolumn{1}{ c }{\cellcolor{white} Day 4 }
		\\\hhline{~---~}

	\cell{Block 1\\\footnotesize(9:00--12:00)}
		& \cell{Housekeeper\\\teams{i}{j}}
		& \cell{Housekeeper\\~\\Party Host}
		& \cell{Restaurant}
		& \cellcolor{white}
		\\\hhline{~----}



	\multicolumn{1}{ c }{}
		& \multicolumn{3}{ c }{\wcell{0.5\baselineskip}{\color{gray}Lunch}}
		& \multicolumn{1}{|c|}{\cellcolor[HTML]{FF8D27}\cell{\textbf{Finals}}}
		\\\hhline{~----}

	\cell{Block 2\\\footnotesize(14:00--17:00)}
		& \cell{Party Host\\\teams{k}{l}}
		& \cellcolor[HTML]{CBCEFB}\cell{Party Host}
		& \cell{Housekeeper}
		& \cellcolor{white}
		\\\hhline{~---~}

	\multicolumn{1}{ c }{}
		& \multicolumn{1}{ c }{\wcell{0.5\baselineskip}{\color[HTML]{029734}Stage 1}}
		& \multicolumn{1}{ c }{\cellcolor{white}}
		& \multicolumn{1}{ c }{\wcell{0.5\baselineskip}{\color[HTML]{6668e5}Stage 2}}\\
	\end{tabular}

	\caption{Example schedule.
		Each of the $N$ teams has two slots assigned per block.
		At least two blocks are scheduled per day with assigned themes.
	}
	\label{tbl:schedule}
\end{table}

\noindent\textbf{Note:} The schedule will be announced during \SetupDays{} (see~\ref{chap:setupdays}) by the \abb{OC}.


\subsection{Team Leader Meeting}
\label{sec:rules:teamleadermeeting}
In the evening before each competition day, a \TLM{} is held. Attendance from all teams participating in the next day's tests is mandatory. During the meeting, teams can ask questions and discuss the upcoming tests with the \abb{TC} and \abb{OC}. The starting time will be announced by the \abb{OC}. Decisions made in the \abb{TLM} are binding. The \abb{TC} and referees on site will decide on anything coming up during or after a test.

\subsection{Scoring System}
\label{sec:rules:scoringsystem}
Each test has a main objective and a set of bonuses.
To score in a test, a team must accomplish the main goal (in parts if allowed). Bonuses are only given if at least \SI{50}{\percent} of the points for the main goal are achieved. Overall scoring in a stage is calculated as the sum of the maximum score obtained in each test. The final score is calculated differently and is normalized (see~\ref{sec:finals:scoring}). A team cannot get a negative score for a test unless a penalty was received.

\paragraph*{Note: } Once a scoresheet has been signed by the team leader or the scores have been published, the \abb{TC} decision is irrevocable.

% Local Variables:
% TeX-master: "../Rulebook"
% End:
