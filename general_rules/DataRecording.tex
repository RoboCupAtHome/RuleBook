\section{Data Recording}
  In order to benchmark robots and software outside the RoboCup@Home arena, the teams are asked to contribute to a public dataset.
  This will consist of audio, imagery and other data obtained and generated by the robots during RoboCup@Home challenges.
  
  \subsection{Collected data}
    During a challenge, specific data is to be gathered and stored on a USB stick. 
    After all attempts at a challenge are made, the USB stick must be given to the TC, which will copy the data to the public dataset.
    The recordings themselves are not used for scoring and may be post-processed manually to be more useful, before handing over to the TC. 
    Not all types of data are interesting for each challenge and thus each challenge will list which data to record. 
    
    \begin{itemize}
    \item \textbf{Audio: } A .wav file of conversation or commands given by any operator and the result of the automatic speech recognition, if applicable.
      The recording must be made of the same signals that are input to the automatic speech recognition software. 
    \item \textbf{Commands: } A text file with the commands as received by the robot. 
      This may be the output of speech recognition or the outcome of any form of the continue rule.
    \item \textbf{Images: } 2D and/or 3D RGB(D) images or pointclouds from the robot's camera(s) while doing any sort of recognition task.
			    Record the full field of view and label the images with the classification result, if appliccable.
    \item \textbf{Mapping data: } Record the data the robot uses for mapping its surroundings and obstacle avoidance plus the resulting map. 
      For many robots this will be 2D laser scans of an Laser Range Finder but other means are possible. 
    \item \textbf{Plans: } Any plan generated by the robot. This includes navigation paths, arm trajectories and action plans. 
      If possible, plans are preferably annoted with whether is was succesfully executed or not.
    \end{itemize}
    
    For ROS-based robots, the most convenient data format for mapping data (laser scans, occupancy grids etc.) and motion plans are their ROS messages recorded into a ROS Bag file.
  
%   
%   5
% In the following, ‘offline’ identifies data produced by the robot that will be collected by the referees when
% the execution of the benchmark ends (e.g., as files on a USB stick), while ‘online’ identifies data that the robot
% has to transmit to the testbed during the execution of the benchmark. Data marked neither with ‘offline’ nor
% with ‘online’ is generated outside the robot. NOTE: the online data should also be displayed by the robot on its
% computer screen, for redundancy purposes, in case problems with wireless communications arise.
% 
% 6Speech files from all teams and all benchmarks (both Task benchmarks and Functional benchmarks) will be
% collected and used to build a public dataset. The audio files in the dataset will therefore include all the defects of
% real-world audio capture using robot hardware (e.g., electrical and mechanical noise, limited bandwidth, harmonic
% distortion). Such files will be usable to test speech recognition software, or (possibly) to act as input during the
% execution of speech recognition benchmarks.
% 
% Catering to Granny_annie's comfort:
% • On the robot, the audio signals of the conversations between Annie and the robot. [offline]
% • The final command produced during the natural language analysis process. [online]
% • The pose of the robot while moving in the environment.
% • The pose of the robot while moving in the environment, as perceived by the robot. [offline]
% • The sensorial data of the robot when recognizing the object to be operated. [offline]
% • The results of the robot’s attempts to execute Annie’s commands.
% 
% Welcoming Visitors:
% The event/command causing the activation of the robot. [online]
% • The video signal from the door camera.
% • The pose of the robot during the execution of the task.
% • The pose of the robot while moving in the environment, as perceived by the robot. [offline]
% • The results of any attempts by the robot to detect and classify a visitor. [online]
% • The audio signals of the conversations with the visitors. [offline]
% • Any notifications from the robot (e.g., alarm if a visitor shows anomalous behavior). [online]
% • The results of any actions taken by the robot, including opening or closing the front door,
% guiding visitors into and around the apartment, manipulating objects, etc.
% 
% Getting to know my home:
% The output files produced by the robot, as described by section 4.3.4. [offline]
% • The pose of the robot during the execution of the task.
% • The pose of the robot while moving in the environment, as perceived by the robot. [offline]
% • The result (success/failure) of the command issued to the robot.