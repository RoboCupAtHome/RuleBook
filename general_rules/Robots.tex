%%%%%%%%%%%%%%%%%%%%%%%%%%%%%%%%%%%%%%%%%%%%%%%%%%%%%%%%%
\section{Robots}
\label{rule:robots}

\subsection{Number of robots}
\label{rule:robots_number}

\begin{enumerate}
	\item \textbf{Registration:} The maximum \term{number of robots} per team is \emph{two} (2).
	\item \textbf{Regular Tests:} Only one robot is allowed per test. For different tests different robots can be used.
	% \item \textbf{Open Demonstrations:} In the \iterm{Open Challenge} and the \iterm{Finals} both robots can be used simultaneously.
\end{enumerate}

\subsection{Appearance and safety}
\label{rule:robot_appearance}

Robots should have a nice product-like appearance, be safe to operate, and should not annoy people. The following rules apply to all robots and are part of the \iterm{Robot Inspection} test (see~\refsec{sec:robot_inspection}).
\begin{enumerate}
	\item \textbf{Cover:} The robot's internal hardware (electronics and cables) should be covered in an appealing way. The use of (visible) duct tape is strictly prohibited.
	\item \textbf{Loose cables:} Loose cables hanging out of the robot are not permitted.
	\item \textbf{Safety:} The robot must not have sharp edges or elements that might harm people.
	\item \textbf{Annoyance:} The robot must not be continuously making loud noises or use blinding lights.
	\item \textbf{Marks:} The robot may not exhibit any kind of artificial marks or patterns.
	\item \textbf{Driving:} To be safe, the robots should be careful when driving (obstacle avoidance is mandatory).
\end{enumerate}

\subsection{Standard Platform Leagues}
RoboCup@Home features two Standard Platform Leagues adhering to the rules listed above.

\subsubsection{Modifications}
\label{rule:spl-mods}
Standardized platforms allow teams to compete in equality of conditions by eliminating all hardware-dependent variables.
Therefore, modifications and alterations to the robots are strictly forbidden; including, but not limited to attaching, connecting, plugging, gluing, and taping components into and onto the robot, as well as modifying or altering the robot structure.
Voiding this rule leads to immediate disqualification from the competition and penalty for the team (see~\refsec{rule:extraordinary_penalties}).

During the \iterm{Robot Inspection} test (see~\refsec{sec:robot_inspection}), the TC will verify that the robot is in proper state for the competition; presenting no alterations and a neat condition.
EC and TC members may request re-inspection of a SPL robot at any time during the competition.

\textbf{Clothing, coloring, and stickers:} Robots are allowed to \enquote{wear} clothes, as well as have stickers (e.g., a sticker exhibiting the logo of an sponsor).
Painting the robot with another color or design is also allowed. 
However, artificial markers (e.g. bar codes, QR codes, OpenCV markers) are strictly forbidden. 
Teams should contact the robot providet before altering the robot's appearance.

% \subsubsection{Domestic Standard Platform League}
% The characteristics of the Toyota Human Support Robot are detailed below.

% \begin{itemize}
	% \item Aimed at human support tasks, elderly care et cetera
	% \item Omni-directional base, maximum speed 0.8km/h
	% \item 1 arm with multifunctional gripper through a vacuum pad. The wrist is equipped with a force-torque sensor. Capable of lifting 1.2kg.
	% \item RGB-D, stereo cameras and wide-angle camera
	% \item Display mounted in head, separate tablet interface
	% \item Access to cloud-based services
	% \item Equipped with a microphone array
	% \item Gravity compensated arm
	% \item Height-adjusting torso
% \end{itemize}
% 
% \subsubsection{Social Standard Platform League}
% The characteristics of the Softbank Robotics/Aldebaran Pepper are detailed below.
% 
% \begin{itemize}
	% \item Aimed at social interaction, public environments, explainable artificial intelligence
	% \item Omni-directional base, maximum speed 3km/h
	% \item 2 arms mostly intended for social gesturing.
	% \item 3D and 2 HD cameras
	% \item Equipped with a built-in tablet
	% \item Access to cloud-based services
	% \item Equipped with a 4-microphone array in the head
	% \item Emotion recognition by voice and images
	% \item Emotion engine to adapt it's attitude
% \end{itemize}

\subsection{Open Platform League}
\label{sec:rules:robotappearance_opl}
Robots competing in the \OPL{} must comply with security specifications in order to avoid causing any harm while operating.

\subsubsection{Size and Weight}
\label{sec:rules:robotappearance_opl:size}

\begin{itemize}
	\item \textbf{Dimensions:} The dimensions of a robot should not exceed the limits of an average door (\SI{200}{\centi\meter} by \SI{70}{\centi\meter}). The \TC{} may allow the qualification and registration of larger robots, but it cannot be guaranteed that the robots can actually enter the arena. In doubt, contact the \LOC{}.
\end{itemize}


\noindent\textbf{Note:} All robot requirements will be tested during \RobotInspection{} (see~\ref{sec:setupdays:inspection}).



% Local Variables:
% TeX-master: "../Rulebook"
% End:
