%%%%%%%%%%%%%%%%%%%%%%%%%%%%%%%%%%%%%%%%%%%%%%%%%%%%%%%%%
\section{Scenario}
\label{sec:scenario}

The tests take place in the \iterm{RoboCup@Home arena}. In addition, particular tests are situated outside the arena, e.g., in a previously unknown public place. The following rules are related to the \iterm{RoboCup@Home arena} and its contents. 

\subsection{RoboCup@Home arena}
The \iterm{RoboCup@Home arena} is a realistic home setting consisting of inter-connected rooms like, for instance, a living room, a kitchen, a bath room, and a bed room. 

% \subsection{Team area}\label{rule:scenario_team_area}

% \todo{remove? does not depend on the rules, but on local organization }
% The maximum number of people to register per team is unlimited, but
% the organization only provides space for \emph{four} (4) persons to
% work at tables in the team area. 
% \todo{this is actually more an additional note for the registration information}

\subsection{Walls, doors and floor}
\label{rule:scenario_walls}

The indoor home setting will be surrounded by high and low \Term{walls}{Arena walls}. These walls will be built up using standard fair construction material.

\begin{enumerate}
	{\bf\item Walls:} Walls have a minimum height of \SI{60}{\centi\meter}. A maximum height is not specified, but should be chosen so that the audience is able to watch the competition.\\
	Walls will be fixed and are likely to be not modified during the competition (see \refsec{rule:scenario_changes}). 

	{\bf\item Doors:} There will be at least two entry/exit \Term{doors}{Arena doors} connecting the outside of the scenario. These doors are used as starting points for the robots (see \refsec{rule:start_position}).
	% At least one of the entrances will be a door with a handle (not a knob).\
	There will be also another door inside the scenario with a handle (not a knob) between any two rooms. Doors with handle (not a knob) may be closed at any time, it is expected robots be able to open them.

	{\bf\item Floor:} The floor of the arena as well as the doorways of the arena are even. That is, there will be no significant steps or even stairways. However, minor unevenness such as carpets, transitions in floor covering between different areas, and minor gaps (especially at doorways) must be expected.

	{\bf\item Appearance:} Floor and walls are mainly uni-colored but can contain texture, e.g., a carpet on the floor, or a poster or picture on the wall.\\
	Although being unlikely at the moment, transparent elements are also possible. 
\end{enumerate}


\subsection{Furniture}
\label{rule:scenario_furniture}

The arena will be equipped with typical objects (furniture) that are not specified in quantity and kind. The minimal configuration consists of 
\begin{itemize}
	\item a small dinner table with two chairs, 
	\item a couch, 
	\item an open cupboard or small table with a television and remote control, 
	\item a cupboard or shelf (with some books inside), and
	\item a refrigerator in the kitchen (with some cans and plastic bottles inside). 
\end{itemize}
A typical arena setup is shown in \reffig{fig:scenario_arena}.

\begin{figure}[tbp]
	\centering
	\subfloat[Typical arena]{\label{fig:scenario_arena}\includegraphics[height=46mm]{images/typical_arena.jpg}} ~ 
	\subfloat[Typical objects]{\label{fig:scenario_objects}\includegraphics[height=46mm]{images/typical_objects.jpg}}
	\caption{Scenario examples: (a) a typical arena, and (b) typical objects.}
	\label{fig:arena}
\end{figure}



\subsection{Changes to the arena}
\label{rule:scenario_changes}

Since the robots should be able to function in the real world the scenario is not fixed and might change without further notice.
\begin{enumerate}
	{\bf\item Major changes:} Changes will primarily influence the position of objects such as furniture inside the arena while walls are likely to stay fixed. Multiple changes may take place up to completely restructuring the internals of the apartment. The position of named locations (see \refsec{rule:scenario_names}) are not changed when used in a test, e.g., as navigation goal. \\
	In addition, passages may be blocked and cleared, respectively. One hour before a test slot begins no \iterm{major changes} will be made.

	{\bf\item Minor changes:} In contrast to major changes, \iterm{minor changes} like, for instance, slightly moved chairs cannot be avoided and may happen at any time (even during a test). 
\end{enumerate}


\def\NumObjects{10\ }
\def\NumLocations{20\ }
\def\NumNames{20\ }

%%%%%%%%%%%%%%%%%%%%%%%%%%%%%%%%%%%%%%%%%%%%%%%%%%%%%%%%%%%%%%%%%%
%
% Objects section.
%
% Revisited by Mauricio Matamoros for 2015
%
%%%%%%%%%%%%%%%%%%%%%%%%%%%%%%%%%%%%%%%%%%%%%%%%%%%%%%%%%%%%%%%%%%

\subsection{Objects}
\label{rule:scenario_objects}
Some tests in the RoboCup@Home league involve the manipulation of objects. These objects resemble items usually found in household environments like, for instances, soda cans, coffee mugs or books. An example of objects used in a previous competition can be seen in \reffig{fig:scenario_objects}.

There are three main categories for the kind of objects used during the competition:

\begin{enumerate}
	{\bf\item Predefined objects:} 
	\begin{enumerate}
		{\bf\item Known objects:} These objects are known beforehand and a copy is available for training before the competition, with no noticeable difference among peers (e.g.~coke cans).

		{\bf\item Alike objects:} These objects are partially known beforehand and a specimen is available for training before the competition, however there are slight differences among peers (e.g.~apples).

	\end{enumerate}

	{\bf\item Containers}: These objects can contain smaller objects or be filled with its content. As with \iterm{known objects}, are known beforehand and a copy is available for training before the competition, with no noticeable difference among peers (e.g.~bowl).

	{\bf\item Unknown objects}: Any other object that does not belong to the set of \iterm{container objects} nor \iterm{predefined objects}.

	{\bf\item Special objects}: Objects whose recognition is not important but can be moved or operated by the robot such as: door handles, chairs, walking sticks, poles, etc.

\end{enumerate}

The following general rules for objects apply:

\begin{enumerate}
	{\bf\item Object classes:} Each object will be assigned to an \iterm{object class}. The objects \quotes{apple} and \quotes{banana} may be of class \quotes{fruits} for example.

	{\bf\item Object (class) locations:} Each object (class) will be assigned to an \iterm{object location}. Objects of class \quotes{fruits} may be usually found on the \quotes{kitchen table}, and objects of class \quotes{unknown} may be usually found on the \quotes{trash bin}, for example.

	{\bf\item Announcement:} The TC makes the set of \iterm{containers} and \iterm{predefined objects}, including their names, classes, and usual locations; available during the setup days. 
	
	{\bf\item Placement:} \nterm{object placement} Unless stated otherwise, in manipulation tasks, the objects will be positioned at \iterm{manipulation locations} and less than \SI{15}{\centi\meter} away from the border of the surface they are located at. There will be at least \SI{5}{\centi\meter} space around each object.
\end{enumerate}

It is not allowed to modify any of the objects provided for training.

\subsubsection{Containers}
A container is any object capable of store, transport or contain, any other (often smaller) object within, such as trays, bowls, baskets, etc.~The TC will provide at least two containers (a transport container such as a tray and a pouring container such as a bowl) which will be available for training during the setup days.

There are no restrictions on a container size, appearance or weight; however, it can be expected that the selected containers be lightweight, with handles, and easily manipulable by a human using both hands.

\paragraph*{Custom containers.} It is allowed that a team provide a \iterm{custom container} adapted to be used by the robot, considering the following:
\begin{enumerate}
	\item Custom containers must be approved by the TC during during the \iterm{Robot Inspection} (see \refsec{sec_robot_inspection}).
	\item Custom containers must \emph{not} have any kind of artificial marks, sensors, or electronic devices.
	\item Penalties may apply for the use of custom containers.
\end{enumerate}

\subsubsection{Predefined objects}
The TC will compile a list of at least \NumObjects objects which will be available for training during the setup days and are referred as \iterm{known objects} in the test specifications. There are no restrictions on an object size, appearance or weight; however, it can be expected that the selected objects are easily manipulable by a human using a single hand.

For \iterm{known objects}, at least one exact copy of the object will be available for training during setup days (as exact as any other \quotes{equal} object you may buy in any store.

For \iterm{alike objects}, at least one specimen of the object will be available for training during setup days, but the one being use during the competition may differ. For instance, if the predefined objects list includes an \quotes{apple} you may expect any size and color for the apple (even marbled apples); or any brand of local newspaper if one is specified. However, the TC will try to reduce the incertitude for this kind of objects. 

\subsubsection{Unknown objects}
Objects, referred as \iterm{unknown objects} in the test specifications are those that can be easily manipulable by a human using a single hand but does not belong to the \iterm{containers} and \iterm{predefined objects} sets.

Note that, any object not being listed in the set of \iterm{containers} nor in the set of \iterm{predefined objects} is automatically considered an unknown object for scoring purposes (e.g.~ornamentation).

\subsubsection{Special objects}
For some tests, special objects are used for the robot to interact with. This interaction includes, but is not limited to: pushing, moving, grasping, hand-grasping, delivering, handling and opening.

With \iterm{special objects} the recognition is not as important as its detection and proper interaction. For instance, when opening a door, having the door open is important, not the door handle itself. The same applies if the robot needs to push a chair in order to navigate through a previously blocked path, or handle a walking stick to the granny.

When possible, the TC in coordination with the OC and the local organization, will try to provide an object of the kind for teams to practice, but this cannot be warrantied.





\subsection{Predefined locations}
\label{rule:scenario_locations}

Some tests in the RoboCup@Home league involve \iterm{predefined locations}. 
These may include places like a \quotes{bookshelf} or a \quotes{dining table}, as well as certain objects such as a \quotes{television}, or the \quotes{front door}. 

\begin{enumerate}
	{\bf\item Definition:} The TC will compile a list of predefined locations. There are no restrictions on which parts of the arena will be selected as a predefined location.

	{\bf\item Location classes:} Each location will be assigned to a \iterm{location class}. The objects \quotes{couch} and \quotes{arm chair} may be of class \quotes{seat} for example. 

	{\bf\item Announcement:} The TC makes the set of locations (and their names and classes) available during the setup days.

	{\bf\item Position:} The positions of locations are \emph{not} necessarily fixed (see \refsec{rule:scenario_changes}).

	{\bf\item Manipulation locations:} The TC will mark at least \NumLocations locations out of the set of predefined locations as being \iterm{manipulation locations}. Whenever a test involves manipulation, the object to manipulate will be placed at one of the manipulation locations. 
\end{enumerate}



\subsection{Predefined rooms}
\label{rule:scenario_rooms}
Some tests in the RoboCup@Home league involve \iterm{predefined rooms}. 
\begin{enumerate}
	{\bf\item Definition:} The TC will compile a list of room names.
	{\bf\item Announcement:} The TC makes the set of rooms available during the setup days.
\end{enumerate}



\subsection{Predefined (person) names}\label{rule:scenario_names}

Some tests in the RoboCup@Home league involve \iterm{predefined names} of people. 

\begin{enumerate}
	{\bf\item Definition:} The TC will compile a list of \NumNames predefined names. The names are \SI{50}{\percent} male and \SI{50}{\percent} female, and taken from the (current) most common first names in the United States.\\
	In order to ease speech recognition, it is tried to select names to be phonetically different from each other.

	{\bf\item Announcement:} The TC makes the set of names available during the setup days.
	{\bf\item Assignment:} When a test involves interacting with persons (using a person's name), all involved persons are assigned names by the referees before the test. 
\end{enumerate}

Typical names are, for example, James, John, Robert, Michael and William as male names; Mary, Patricia, Linda, Barbara and Elizabeth as female names.


%% %%%%%%%%%%%%%%%%%%%%%%%%
\subsection{Wireless network}
\label{rule:scenario_wifi}

For wireless communication, an \iterm{arena network} is provided. The actual infrastructure depends on the local organization. 

\begin{itemize}
	\item To avoid interference with other leagues, this WiFi has to be used for communication only. It is not allowed to use the above or any other WiFi network for personal use at the venue.
	\item During the competitions, only the active team is allowed to use the \iterm{arena network}. 
	\item The organizers cannot guarantee reliability and performance of wireless communication. Therefore, teams are required to be ready to setup, start their robots and run the tests even if, for any reason, network is not working properly.
\end{itemize}

Preferably the organizers will try to provide one LAN cable on the desk of each participating team for Internet connection. However, this cannot be guaranteed. If multiple LAN connections are needed, each team has to bring its own LAN hub/switch and cables.

\paragraph*{Important note:} Any unapproved wireless device may be removed by the TC at any time.

\subsection{Smart Home Devices}
\label{rule:smarthomedevices}

The Organizing and Technical Committees in coordination with the Local Organization will compile a list of \iterm{Smart~Home} official devices that will be available in the arena and can be used in some tests for additional score.

At any time, only the Smart~Home devices provided by the Local Organization and approved by the Technical Committee may be used during competition.

\subsubsection{Smart~Home devices list announcement}
The list if Smart~Home devices will be provided to teams as soon as it becomes available and has been granted by the Local Organization and approved by the Technical Committee. 

This list must be announced at least one month prior the competition. In case that this list does not become available for that date, Smart~Home devices may still be present at the arena for testing, but no additional score can be achieved for its use. This is to maintain fair conditions among all teams.

\subsubsection{Technical specifications}
The list of \iterm{Smart~Home} official devices will include as much technical information as possible. However, before it becomes available you may assume the following considerations:

\begin{enumerate}
	{\bf\item Interface:} Most Smart~Home devices interface wireless, often via R/F transmitters. When possible, the OC will provide an official interface via the \iterm{arena network}.
	{\bf\item Operating voltage:} The operating voltage used will be the standard for the place of the competition (e.g.120V/60Hz for North America and 220V/50Hz for Europe). Please note that devices designed for other voltages/frecuencies may burn when plugged to the outlet.
	{\bf\item Type of devices:} Mostly Smart~Home switches will be used (set on/off, read can not be guaranteed). For high bandwidth devices such as microphones or video cameras, an official interface (such as a ROS topic or web service) will be provided via the \iterm{arena network}.
\end{enumerate}

\subsubsection{Availability \& Scoring}
All test has been designed to optionally allow the use of Smart~Home devices and even grant bonus scoring for using this option. However, robots must be able to continue operating normally when there are no Smart~Home devices available. Therefore, it is unacceptable that a robot stuck while trying to operate those devices.

As stated in \refsec{rule:scenario_wifi}, organizers cannot guarantee reliability and performance of wireless communication. Therefore, in case of malfunction or communication problems with the Smart~Home devices, or any other issue which may affect scoring, no claims will be accepted by the EC/TC/OC, nor test will be repeated. The decision on if a team given points for using \iterm{Smart~Home} devices, is conducted by the \iaterm{Technical Committee}{TC}, and it reserves the rights of discarding Smart~Home related scoring.
