\subsection{Vizbox}
\label{vizbox}

The objective of RoboCup is to \enquote{promote robotics and AI research, by offering a publicly appealing, but formidable challenge} \footnote{\url{http://robocup.org/objective}}.

Part of making RoboCup@Home appealing, is to show the audience what is going on, what the robots should do and what they are doing.

To this end, robots in RoboCup@Home are expected run the RoboCup@Home \href{https://github.com/LoyVanBeek/vizbox}{VizBox}\footnote{\url{https://github.com/LoyVanBeek/vizbox}}.

This is a web server to be run on a robot during a challenge. The page it serves can be displayed on a screen, visible to the audience, via a secondary computer in or around the arena, connected to the web server via the wireless network.

All robots are expected to run the \iterm{VizBox}; the audience expects to know what all the robots are doing and what each challenge entails.

The \iterm{VizBox}'s code is hosted \url{https://github.com/LoyVanBeek/vizbox}.
We want to show the audience a consistent presentation, so ideally, all teams run the same VizBox code.
Sharing your changes back in the form of a Pull Request is much appreciated so all teams can benefit.

The \iterm{VizBox} has the following visualization capabilities:
\begin{itemize}
	\item Images of what the robot sees or a visualization of the robot's world model, eg. camera images, it's map, anything to make clear what is going on to the audience.
	\item Show an outline of the current challenge and where the robot is in the story of the current challenge.
	\item Subtitles of what the robot and operator just said; their conversation
\end{itemize}

Additionally, the \iterm{VizBox} offers a way to \textbf{input} a text command to the robot, to bypass automatic speech recognition if need be.

The exact documentation is maintained in the repository of the \iterm{VizBox} itself.